\documentclass[a4paper,UKenglish]{lipics-v2016}

\usepackage{amssymb,amsmath}
\usepackage{amsthm}
\usepackage{cmll}
\usepackage{txfonts}
\usepackage{graphicx}
\usepackage{stmaryrd}
\usepackage{todonotes}
\usepackage{mathpartir}
\usepackage{hyperref}
\usepackage{mdframed}
\usepackage[barr]{xy}
\usepackage{comment}
\usepackage{graphicx}
\usepackage[inline]{enumitem}

\usepackage{caption}
\captionsetup[figure]{name=Diagram}

%% This renames Barr's \to to \mto.  This allows us to use \to for imp
%% and \mto for a inline morphism.
\let\mto\to
\let\to\relax
\newcommand{\to}{\rightarrow}
\newcommand{\ndto}[1]{\to_{#1}}
\newcommand{\ndwedge}[1]{\wedge_{#1}}
\newcommand{\rto}{\leftharpoonup}
\newcommand{\lto}{\rightharpoonup}

% Commands that are useful for writing about type theory and programming language design.
%% \newcommand{\case}[4]{\text{case}\ #1\ \text{of}\ #2\text{.}#3\text{,}#2\text{.}#4}
\newcommand{\interp}[1]{\llbracket #1 \rrbracket}
\newcommand{\normto}[0]{\rightsquigarrow^{!}}
\newcommand{\join}[0]{\downarrow}
\newcommand{\redto}[0]{\rightsquigarrow}
\newcommand{\nat}[0]{\mathbb{N}}
\newcommand{\fun}[2]{\lambda #1.#2}
\newcommand{\CRI}[0]{\text{CR-Norm}}
\newcommand{\CRII}[0]{\text{CR-Pres}}
\newcommand{\CRIII}[0]{\text{CR-Prog}}
\newcommand{\subexp}[0]{\sqsubseteq}
%% Must include \usepackage{mathrsfs} for this to work.

\date{}

\let\b\relax
\let\d\relax
\let\t\relax
\let\r\relax
\let\c\relax
\let\j\relax
\let\wn\relax
\let\H\relax

% Cat commands.
\newcommand{\powerset}[1]{\mathcal{P}(#1)}
\newcommand{\cat}[1]{\mathcal{#1}}
\newcommand{\func}[1]{\mathsf{#1}}
\newcommand{\iso}[0]{\mathsf{iso}}
\newcommand{\H}[0]{\func{H}}
\newcommand{\J}[0]{\func{J}}
\newcommand{\catop}[1]{\cat{#1}^{\mathsf{op}}}
\newcommand{\Hom}[3]{\mathsf{Hom}_{\cat{#1}}(#2,#3)}
\newcommand{\limp}[0]{\multimap}
\newcommand{\colimp}[0]{\multimapdotinv}
\newcommand{\dial}[1]{\mathsf{Dial_{#1}}(\mathsf{Sets^{op}})}
\newcommand{\dialSets}[1]{\mathsf{Dial_{#1}}(\mathsf{Sets})}
\newcommand{\dcSets}[1]{\mathsf{DC_{#1}}(\mathsf{Sets})}
\newcommand{\sets}[0]{\mathsf{Sets}}
\newcommand{\obj}[1]{\mathsf{Obj}(#1)}
\newcommand{\mor}[1]{\mathsf{Mor(#1)}}
\newcommand{\id}[0]{\mathsf{id}}
\newcommand{\lett}[0]{\mathsf{let}\,}
\newcommand{\inn}[0]{\,\mathsf{in}\,}
\newcommand{\cur}[1]{\mathsf{cur}(#1)}
\newcommand{\curi}[1]{\mathsf{cur}^{-1}(#1)}

\newcommand{\w}[1]{\mathsf{weak}_{#1}}
\newcommand{\c}[1]{\mathsf{contra}_{#1}}
\newcommand{\cL}[1]{\mathsf{contraL}_{#1}}
\newcommand{\cR}[1]{\mathsf{contraR}_{#1}}
\newcommand{\e}[1]{\mathsf{ex}_{#1}}

\newcommand{\m}[1]{\mathsf{m}_{#1}}
\newcommand{\n}[1]{\mathsf{n}_{#1}}
\newcommand{\b}[1]{\mathsf{b}_{#1}}
\newcommand{\d}[1]{\mathsf{d}_{#1}}
\newcommand{\h}[1]{\mathsf{h}_{#1}}
\newcommand{\p}[1]{\mathsf{p}_{#1}}
\newcommand{\q}[1]{\mathsf{q}_{#1}}
\newcommand{\t}[0]{\mathsf{t}}
\newcommand{\r}[1]{\mathsf{r}_{#1}}
\newcommand{\s}[1]{\mathsf{s}_{#1}}
\newcommand{\j}[1]{\mathsf{j}_{#1}}
\newcommand{\jinv}[1]{\mathsf{j}^{-1}_{#1}}
\newcommand{\wn}[0]{\mathop{?}}
\newcommand{\codiag}[1]{\bigtriangledown_{#1}}

\newenvironment{changemargin}[2]{%
  \begin{list}{}{%
    \setlength{\topsep}{0pt}%
    \setlength{\leftmargin}{#1}%
    \setlength{\rightmargin}{#2}%
    \setlength{\listparindent}{\parindent}%
    \setlength{\itemindent}{\parindent}%
    \setlength{\parsep}{\parskip}%
  }%
  \item[]}{\end{list}}

\newenvironment{diagram}{
  \begin{center}
    \begin{math}
      \bfig
}{
      \efig
    \end{math}
  \end{center}
}

%% %% Ott
%% \input{BiLNL-inc}

\title{On Linear Modalities for Exchange, Weakening, and Contraction}
\author[1]{Jiaming Jiang}
\author[2]{Harley Eades III}
\affil[1]{Computer Science, North Carolina State University, Raleigh, North Carolina, USA\\
  \texttt{jjiang13@ncsu.edu}}
\affil[2]{Computer Science, Augusta University, Augusta, Georgia, USA\\
  \texttt{heades@augusta.edu}}

\Copyright{Jiaming J. Open and Harley E. Access}

\subjclass{TODO}
\keywords{TODO}

\begin{document}

\maketitle 

\begin{abstract}

  TODO

\end{abstract}

\section{Introduction}
\label{sec:introduction}
TODO \cite{Benton:1994}
% section introduction (end)

\section{Categorical Models}
\label{sec:categorical_models}

\subsection{Lambek Categories}
\label{subsec:lambek_categories}
\begin{definition}
  \label{def:Lambek-category}
  A \textbf{monoidal category}, $(\cat{L}, \otimes, I, \lambda,
  \rho)$, is a category, $\cat{L}$, equipped with a bifunctor,
  $\otimes : \cat{L} \times \cat{L} \mto \cat{L}$, called the tensor
  product, a distinguished object $I$ of $\cat{L}$ called the unit,
  and three natural isomorphisms $\lambda_A : I \otimes A \mto A$,
  $\rho_A : A \otimes I \mto A$, and $\alpha_{A,B,C} : A \otimes (B
  \otimes C) \mto (A \otimes B) \otimes C$ called the left and right
  unitors and the associator respectively.  Finally, these are subject
  to the following coherence diagrams:
  \begin{mathpar}
    \bfig
    \hSquares|aammmma|/->`->`->``->``/<400>[
      ((A \otimes B) \otimes C) \otimes D`
      (A \otimes (B \otimes C)) \otimes D`
      A \otimes ((B \otimes C) \otimes D)`
      (A \otimes B) \otimes (C \otimes D)``
      A \otimes (B \otimes (C \otimes D));
      \alpha_{A,B,C} \otimes \id_D`
      \alpha_{A,B \otimes C,D}`
      \alpha_{A \otimes B,C,D}``
      \id_A \otimes \alpha_{B,C,D}``]

    \morphism(-200,0)<2700,0>[
      (A \otimes B) \otimes (C \otimes D)`
      A \otimes (B \otimes (C \otimes D));
      \alpha_{A,B,C \otimes D}]
    \efig
    \and
    \bfig
    \Vtriangle[
      (A \otimes I) \otimes B`
      A \otimes (I \otimes B)`
      A \otimes B;
      \alpha_{A,I,B}`
      \rho_{A}\otimes id_B`
      id_A\otimes\lambda_{B}]
    \efig
  \end{mathpar}
\end{definition}

\begin{definition}
  \label{def:Lambek-category}
  A \textbf{Lambek category} is a monoidal category $(\cat{L},
  \otimes, I, \lambda, \rho,\alpha)$ equipped with two bifunctors
  $\lto : \catop{L} \times \cat{L} \mto \cat{L}$ and $\rto : \cat{L}
  \times \catop{L} \mto \cat{L}$ that are both right adjoint to the
  tensor product.  That is, the following natural bijections hold:
  \begin{center}
    \begin{math}
      \begin{array}{lll}
        \Hom{L}{X \otimes A}{B} \cong \Hom{L}{X}{A \lto B} & \quad\quad\quad\quad & 
        \Hom{L}{A \otimes X}{B} \cong \Hom{L}{X}{B \rto A}\\
      \end{array}
    \end{math}
  \end{center}
\end{definition}
One might call Lambek categories biclosed monoidal categories, but we
name them in homage to Lambek for his contributions to non-commutative
linear logic.

\begin{definition}
  \label{def:sym-monoidal-category}
  A monoidal category $(\cat{L},\otimes,I,\lambda,\rho,\alpha)$ is
  \textbf{symmetric} if there is a natural transformation $\beta_{A,B}
  : A \otimes B \mto B \otimes A$ such that $\beta_{B,A} \circ
  \beta_{A,B} = \id_{A \otimes B}$ and the following commute:
  \begin{center}
    \begin{math}
      \small
      \begin{array}{lll}
        \bfig
        \hSquares|aammmaa|/->`->`->``->`->`->/[
        (A \otimes B) \otimes C`
        A \otimes (B \otimes C)`
        (B \otimes C) \otimes A`
        (B \otimes A) \otimes C`
        B \otimes (A \otimes C)`
        B \otimes (C \otimes A);
        \alpha_{A,B,C}`
        \beta_{A,B \otimes C}`
        \beta_{A,B} \otimes \id_C``
        \alpha_{B,C,A}`
        \alpha_{B,A,C}`
        \id_B \otimes \beta_{A,C}]
        \efig
        & \quad &
        \bfig
          \Vtriangle[
            I \otimes A`
            A \otimes I`
            A;
            \beta_{I,A}`
            \lambda_A`
            \rho_A]
          \efig
      \end{array}
    \end{math}
  \end{center}
\end{definition}

\begin{definition}
  \label{def:sym-monoidal-closed}
  A symmetric monoidal category $(\cat{L}, \otimes, I, \lambda, \rho,
  \alpha, \beta)$ is \textbf{closed} if it comes equipped with a
  bifunctor $\limp : \catop{L} \times \cat{L} \mto \cat{L}$ that is
  right adjoint to the tensor product.  That is, the following natural
  bijection $\Hom{L}{X \otimes A}{B} \cong \Hom{L}{X}{A \limp B}$ holds.
\end{definition}

\begin{definition}
  \label{def:weakening}
  A \textbf{Lambek category with weakening}, $(\cat{L}, \otimes, I,
  \lambda, \rho, \alpha, \w{})$, is a Lambek category $(\cat{L}, \otimes, I,
  \lambda, \rho,\alpha)$ equipped with a natural transformation
  $\w{A} : A \mto I$.
\end{definition}

\begin{definition}
  \label{def:contraction}
  A \textbf{Lambek category with contraction}, $(\cat{L}, \otimes, I,
  \lambda, \rho, \alpha, \cL{},\cR{})$, is a Lambek category $(\cat{L}, \otimes, I,
  \lambda, \rho,\alpha)$ equipped with natural transformations:
  \[
  \begin{array}{lll}
    \cL{A,B} : (A \otimes B) \mto (A \otimes B) \otimes A & \quad &
    \cR{A,B} : (B \otimes A) \mto A \otimes (B \otimes A)\\
  \end{array}
  \]
  Furthermore, the following diagrams must commute:
    \begin{mathpar}
      \bfig
      \square/<-`->``/<1050,400>[
	A\otimes I`
        A`
        (A\otimes I)\otimes A`;
	\rho_{A}^{-1}`
	\cL{A,I}``]
      \square(1050,0)/->``->`/<1050,400>[
        A`
        I\otimes A``
        A\otimes(I\otimes A);
        \lambda_{A}^{-1}``
	\cR{A,I}`]
        \morphism(0,0)|b|<2100,0>[(A\otimes I)\otimes A`A\otimes(I\otimes A);\alpha_{A,I,A}]
      \efig
    \end{mathpar}
    \begin{mathpar}
    \bfig
      \square/->`->``->/<1300,800>[
        A\otimes A`
        A\otimes(A\otimes I)`
        (I\otimes A)\otimes A`
        (A\otimes(I\otimes A))\otimes A;
        id_{A}\otimes\rho_{A}^{-1}`
        \lambda_{A}^{-1}\otimes id_{A}``
        \cR{A,I}\otimes id_{A}]
      \qtriangle(1300,400)/->``->/<1300,400>[
        A\otimes(A\otimes I)`
        A\otimes((A\otimes I)\otimes A)`
        A\otimes(A\otimes A);
        id_{A}\otimes\cL{A,I}``
        id_{A}\otimes(\rho_{A}\otimes id_{A})]
      \dtriangle(1300,0)/`<-`->/<1300,400>[
        A\otimes(A\otimes A)`
        (A\otimes(I\otimes A))\otimes A`
        (A\otimes A)\otimes A;
        `
        \alpha_{A,A,A}`
        (id_{A}\otimes\lambda_{A})\otimes id_{A}]
    \efig
    \end{mathpar}
\end{definition}

\begin{definition}
  \label{def:exchange}
  A \textbf{Lambek category with exchange}, $(\cat{L}, \otimes, I,
  \lambda, \rho, \alpha, \e{})$, is a Lambek category, $(\cat{L},
  \otimes, I, \lambda, \rho, \alpha)$, such that $\cat{L}$ is
  symmetric monoidal, where $\e{A,B} : A \otimes B \mto B \otimes A$
  is the symmetry.
\end{definition}

% subsection lambek_categories (end)

\subsection{Lambek Categories with Weakening and Contraction}
\label{subsec:lambek_categories_with_weakening_and_contraction}
\begin{definition}
  \label{def:weakening}
  A \textbf{Lambek category with weakening}, $(\cat{L},w,\w{})$, is a
  Lambek category equipped with a monoidal comonad
  $(w,\varepsilon,\delta)$, and a monoidal natural transformation
  $\w{A}:wA\mto I$.  Furthermore, $\w{}$ must be a coalgebra morphism.
  That is, the following digram must commute:
  \begin{mathpar}
    \bfig
    \square/->`->`->`->/<800,400>[
      wA`
      I`
      w^2A`
      wI;
      \w{A}`
      \delta_A`
      \q{I}`
      w\w{A}]
    \efig
  \end{mathpar}
\end{definition}



\begin{definition}
  \label{def:contraction}
  A \textbf{Lambek category with contraction},
  $(\cat{L},c,\cL{},\cR{})$, is a Lambek category equipped with a
  monoidal comonad $(c,\varepsilon,\delta)$, and two monoidal natural
  transformations:
  \[
  \begin{array}{l}
    \cL{A,B}:cA\otimes B\mto (cA\otimes B)\otimes cA\\
    \cR{A,B}:B\otimes cA\mto cA\otimes (B\otimes cA)
  \end{array}
  \]
  Furthermore, the following diagrams must commute:
    \begin{mathpar}
      \bfig
      \square/<-`->``/<1050,400>[
	cA\otimes I`
        cA`
        (cA\otimes I)\otimes cA`;
	\rho_{cA}^{-1}`
	\cL{A,I}``]
      \square(1050,0)/->``->`/<1050,400>[
        cA`
        I\otimes cA``
        cA\otimes(I\otimes cA);
        \lambda_{cA}^{-1}``
	\cR{A,I}`]
        \morphism(0,0)|b|<2100,0>[(cA\otimes I)\otimes cA`cA\otimes(I\otimes cA);\alpha_{cA,I,cA}]
      \efig
    \end{mathpar}
    \begin{mathpar}
    \bfig
      \square/->`->``->/<1300,800>[
        cA\otimes cA`
        cA\otimes(cA\otimes I)`
        (I\otimes cA)\otimes cA`
        (cA\otimes(I\otimes cA))\otimes cA;
        id_{cA}\otimes\rho_{cA}^{-1}`
        \lambda_{cA}^{-1}\otimes id_{cA}``
        \cR{A,I}\otimes id_{cA}]
      \qtriangle(1300,400)/->``->/<1300,400>[
        cA\otimes(cA\otimes I)`
        cA\otimes((cA\otimes I)\otimes cA)`
        cA\otimes(cA\otimes cA);
        id_{cA}\otimes\cL{A,I}``
        id_{cA}\otimes(\rho_{cA}\otimes id_{cA})]
      \dtriangle(1300,0)/`<-`->/<1300,400>[
        cA\otimes(cA\otimes cA)`
        (cA\otimes(I\otimes cA))\otimes cA`
        (cA\otimes cA)\otimes cA;
        `
        \alpha_{cA,cA,cA}`
        (id_{cA}\otimes\lambda_{cA})\otimes id_{cA}]
    \efig
    \end{mathpar}
\end{definition}

% subsection lambek_categories_with_weakening_and_contraction (end)

\subsection{Lambek Categories with Exchange}
\label{subsec:lambek_categories_with_exchange}

\begin{definition}
  \label{def:exchange}
  A \textbf{Lambek category with exchange}, $(\cat{L},e,\e{})$, is a
  Lambek category equipped with a monoidal comonad
  $(e,\varepsilon,\delta)$ on $\cat{L}$, and a monoidal natural
  transformation $\e{A,B}:eA \otimes eB \mto eB \otimes eA$.  We
  require $\e{}$ to be a coalgebra morphism, and the following
  diagrams must commute:
  \[
  \begin{array}{cccc}
    \bfig
    \qtriangle|amm|/->`=`->/<1000,500>[
      eA \otimes eB`
      eB \otimes eA`
      eA \otimes eB;
      \e{A,B}``
      \e{B,A}]
    \efig
    &
    \quad
    &
    \bfig
    \square|amma|<1000,500>[
      e^2A \otimes e^2B`
      e^2B \otimes e^2A`
      e(eA \otimes eB)`
      e(eB \otimes eA);
      \e{eA,eB}`
      \q{eA,eB}`
      \q{eB,eA}`
      e\e{A,B}]
    \efig          
  \end{array}
  \]
  \[
  \begin{array}{cccccccc}
    \bfig
    \square|amma|/->`->``/<1100,400>[
      (eA \otimes eB) \otimes eC`
      eA \otimes (eB \otimes eC)`
      (eB \otimes eA) \otimes eC`;
      \alpha_{eA,eB,eC}`
      \e{A,B} \otimes \id_{eC}``]
    
    \morphism(0,0)|m|/->/<0,-800>[
      (eB \otimes eA) \otimes eC`
      eB \otimes (eA \otimes eC);
      \alpha_{eB,eA,eC}]

    \morphism(0,-800)|a|/->/<1100,0>[
      eB \otimes (eA \otimes eC)`
      eB \otimes (eC \otimes eA);
      \id_{eB} \otimes \e{A,C}]
    
    \morphism(1100,400)|a|/->/<1150,0>[
      eA \otimes (eB \otimes eC)`
      eA \otimes (e^2B \otimes e^2C);
      \id_{eA} \otimes (\delta_{B} \otimes \delta_C)]
    \morphism(2250,400)|m|/->/<0,-400>[
      eA \otimes (e^2B \otimes e^2C)`
      eA \otimes e(eB \otimes eC);
      \id_{eA} \otimes \q{eB,eC}]
    \morphism(2250,0)|m|/->/<0,-400>[
      eA \otimes e(eB \otimes eC)`
      e(eB \otimes eC) \otimes eA;
      \e{eA,eB \otimes eC}]
    \morphism(2250,-400)|m|/->/<0,-400>[
      e(eB \otimes eC) \otimes eA`
      (eB \otimes eC) \otimes eA;
      \varepsilon_{eB \otimes eC} \otimes \id_{eA}]
    \morphism(2250,-800)|a|/->/<-1150,0>[
      (eB \otimes eC) \otimes eA`
      eB \otimes (eC \otimes eA);
      \alpha_{eB,eC,eA}]
    \efig
  \end{array}
  \]
  Furthermore, for any coalgebra morphisms $f : (eA, \delta) \mto
  (eB,\delta)$ and $g : (eC,\delta) \mto (eD,\delta)$ between free
  coalgebras the following diagram must commute:
  \[
  \bfig
  \square(0,-500)|amma|<800,500>[
    eA \otimes eC`
    eB \otimes eD`
    eC \otimes eA`
    eD \otimes eB;
    f \otimes g`
    \e{A,C}`
    \e{B,D}`
    g \otimes f]      
  \efig
  \]
  The morphism $\q{A,B} : eA \otimes eB \mto e(A \otimes B)$ makes
  $(e,\q{})$ a monoidal functor.
\end{definition}
The first diagram in the previous definition makes $\e{}$ an
involution, and the second and third diagrams are required in the
proof that the Eilenberg-Moore category is symmetric; see the proofs of
Lemma~\ref{lemma:beta-coalgebra-morph} and
Lemma~\ref{lemma:the_eilenberg-moore_category_is_symmetric_monoidal}.

\begin{definition}
  \label{def:eilenberg-moore-cat}
  Suppose $(\cat{L},e,\e{})$ is a Lambek category with exchange.  Then
  the \textbf{Eilenberg Moore category}, $\cat{L}^e$, of the comonad
  $(e, \varepsilon, \delta)$ has as objects all the e-coalgebras $(A,
  h_A : A \mto eA)$, and as morphisms all the coalgebra morphisms.  We
  call $h_A$ the action of the coalgebra.  Furthermore, the following
  (action) diagrams must commute:
  \begin{mathpar}
    \bfig
    \square[A`eA`eA`e^2A;h_A`h_A`eh_A`\delta_A]    
    \efig
    \and
    \bfig
    \btriangle/->`=`->/[A`eA`A;h_A``\varepsilon_A]
    \efig
  \end{mathpar}
\end{definition}

\begin{lemma}[The Eilenberg Moore Category is Monoidal]
  \label{lemma:the_eilenberg_moore_category_is_monoidal}
  Then the category $\cat{L}^e$ is monoidal.
\end{lemma}
\begin{proof}
  We must first define the unitors, and then the associator.  Then we
  show that they respect the symmetry monoidal coherence diagrams.
  Throughout this proof we will make use of the coalgebra $(A,h_A)$,
  $(B,h_B)$, and $(C,h_C)$.

  The tensor product of $(A, h_A)$ and $(B, h_b)$ is $(A \otimes
  B,q_{A,B} \circ (h_A \otimes h_B))$, and the unit of the tensor
  product is $(I, q_I)$; both actions are easily shown to satisfies
  the action diagrams of the Eilenberg Moore category. The left and
  right unitors are $\lambda : I \otimes A \mto A$ and $\rho : A
  \otimes I \mto A$, because they are indeed coalgebra morphisms.

  \ \\
  \noindent
  The respective diagram for the right unitor is as follows:
  \begin{center}
    \begin{math}
      \begin{array}{lll}
        \bfig
        \square|amma|/->`->``/<650,500>[A \otimes I`eA \otimes I`A`;h_A \otimes \id{}`\rho``]
        \morphism(650,500)<650,0>[eA \otimes I`eA \otimes eI;\id{} \otimes \q{I}]
        \square(1300,0)|amma|/->``->`/<650,500>[eA \otimes eI`e(A \otimes I)``eA;\q{A,I}``e\rho`]

        \morphism(650,500)<1300,-500>[eA \otimes I`eA;\rho]
        \morphism<1950,0>[A`eA;h_A]
        \efig
      \end{array}
    \end{math}
  \end{center}
  The left diagram commutes by naturality of $\rho$, the right diagram
  commutes by the fact that $e$ is a monoidal functor.  Showing the
  left unitor is a coalgebra morphism is similar.

  The unitors are natural and isomorphisms, because they are
  essentially inherited from the underlying Lambek category.

  The associator $\alpha : (A \otimes B) \otimes C \mto A \otimes (B
  \otimes C)$ is also a coalgebra morphism.  First, notice that:
  \[\q{A \otimes B,C} \circ ((\q{A,B} \circ (h_A \otimes h_B)) \otimes h_c) = \q{A \otimes B,C} \circ (\q{A,B} \otimes \id{}) \circ ((h_A \otimes h_B) \otimes h_C)\]
  where the left-hand side is the action of the coalgebra $(A \otimes B)
  \otimes C$. Similarly, the following is the action of the coalgebra
  $A \otimes (B \otimes C)$:
  \[
  \q{A,B \otimes C} \circ (h_A \otimes (\q{B,C} \circ (h_B \otimes h_C))) = \q{A,B \otimes C} \circ (\id{} \otimes \q{B,C}) \circ (h_A \otimes (h_B \otimes h_C))
  \]
  The following diagram must commute:
  \begin{center}
    \rotatebox{90}{$    
      \bfig
      \square|amma|<1200,500>[
        (A \otimes B) \otimes C`
        (eA \otimes eB) \otimes eC`
        A \otimes (B \otimes C)`
        eA \otimes (eB \otimes eC);
        (h_A \otimes h_B) \otimes h_C`
        \alpha`
        \alpha`
        h_A \otimes (h_B \otimes h_C)]

      \square(1200,0)|amma|/->`->``->/<1200,500>[
        (eA \otimes eB) \otimes eC`
        e(A \otimes B) \otimes eC`
        eA \otimes (eB \otimes eC)`
        eA \otimes e(B \otimes C);
        \q{} \otimes \id{}`
        \alpha``
        \id{} \otimes \q{}]

      \square(2400,0)|amma|/->``->`->/<1200,500>[
        e(A \otimes B) \otimes eC`
        e((A \otimes B) \otimes C)`
        eA \otimes e(B \otimes C)`
        e(A \otimes (B \otimes C));
        \q{}``
        e\alpha`
       \q{}]
      \efig
      $}
  \end{center}
  The left diagram commutes by naturality of $\alpha$, and the right
  diagram commutes because $e$ is a monoidal functor.

  Composition in $\cat{L}^e$ is the same as $\cat{L}$, and thus, the
  monoidal coherence diagrams hold in $\cat{L}^e$ as well.  Thus,
  $\cat{L}^e$ is monoidal.  We now show that it is symmetric.    
\end{proof}

\begin{lemma}
  \label{lemma:pseudo-braided}
  In $\cat{L}^e$ there is a natural transformation $\beta_{A,B} : A \otimes B \mto B \otimes A$.
\end{lemma}
\begin{proof}
  We define $\beta$ as follows:
  \[
  \beta_{A,B} := A \otimes B \mto^{h_A \otimes h_B} eA \otimes eB \mto^{ex_{A,B}} eB \otimes eA \mto^{\varepsilon_B \otimes \varepsilon_A} B \otimes A
  \]
  Suppose $f : A \mto A'$ and $g : B \mto B'$ are two coalgebra
  morphisms.  Then the following diagram shows that $\beta_{A,B}$ is a
  natural transformation:
  \[
  \bfig
  \square|amma|<900,500>[
    A \otimes B`
    eA \otimes eB`
    A' \otimes B'`
    eA' \otimes eB';
    h_A \otimes h_B`
    f \otimes g`
    ef \otimes eg`
    h_{A'} \otimes h_{B'}]

  \square(900,0)|amma|<900,500>[
    eA \otimes eB`
    eB \otimes eA`
    eA' \otimes eB'`
    eB' \otimes eA';
    \e{A,B}`
    ef \otimes eg`
    eg \otimes ef`
    \e{A',B'}]

  \square(1800,0)|amma|<900,500>[
    eB \otimes eA`
    B \otimes A`
    eB' \otimes eA'`
    B' \otimes A';
    \varepsilon_{B} \otimes \varepsilon_{A}`
    eg \otimes ef`
    g \otimes f`
    \varepsilon_{B'} \otimes \varepsilon_{A'}]
  \efig
  \]
  The left diagram commutes because $f$ and $g$ are both coalgebra
  morphisms, the middle diagram commutes because $\e{A,B}$ is a
  natural transformation, and the right diagram commutes by naturality
  of $\varepsilon$.
\end{proof}

\begin{corollary}
  \label{corollary:ex-simple}
  For any coalgebras $(A,h_A)$ and $(B,h_B)$ the following commutes:
    \[
    \bfig
    \square|aaaa|/->`=``->/<700,500>[
      A \otimes B`
      eA \otimes eB`
      A \otimes B`
      eA \otimes eB;
      h_A \otimes h_B```
      h_A \otimes h_B]

    \square(700,0)|aaaa|/->```->/<700,500>[
      eA \otimes eB`
      eB \otimes eA`
      eA \otimes eB`
      eB \otimes eA;
      \e{A,B}```
      \e{A,B}]

    \square(1400,0)|aama|/->``->`=/<700,500>[
      eB \otimes eA`
      B \otimes A`
      eB \otimes eA`
      eB \otimes eA;
      \varepsilon_B \otimes \varepsilon_A``
      h_B \otimes h_A`]
      \efig
    \]
\end{corollary}
\begin{proof}
  This proof follows by the fact that the following diagram commutes:
  \[
  \bfig
  \square|mmmm|/=`->`->`=/<900,500>[
    A \otimes B`
    A \otimes B`
    eA \otimes eB`
    eA \otimes eB;`
    h_A \otimes h_B`
    h_A \otimes h_B`]

  \qtriangle(0,-500)|mmm|/=`<-`->/<900,500>[
    eA \otimes eB`
    eA \otimes eB`
    e^2A \otimes e^2B;`
    \varepsilon_{eA} \otimes \varepsilon_{eB}`
    h_{eA} \otimes h_{eB}]

  \dtriangle(0,-1000)|mmm|/`->`<-/<900,500>[
    e^2A \otimes e^2B`
    eB \otimes eA`
    e^2B \otimes e^2A;`
    \e{eA,eB}`
    \varepsilon_{eB} \otimes \varepsilon_{eA}]

  \morphism(0,0)<0,-1000>[eA \otimes eB`eB \otimes eA;\e{A,B}]

  \square(900,0)|mmmm|/=`->`->`/<900,500>[
    A \otimes B`
    A \otimes B`
    eA \otimes eB`
    eA \otimes eB;`
    h_A \otimes h_B`
    h_A \otimes h_B`]

  \dtriangle(900,-1500)|mmm|/`->`<-/<900,1500>[
    eA \otimes eB`
    B \otimes A`
    eB \otimes eA;`
    \e{A,B}`
    \varepsilon_{B} \otimes \varepsilon_A]

  \morphism(900,-1500)|m|<-900,500>[B \otimes A`eB \otimes eA;h_B \otimes h_A]
  \efig
  \]
  The diagram on the right commutes because $\beta_{A,B}$ is a natural
  transformation, and the other diagrams commute either because
  $\cat{L}$ is a Lambek category with exchange, or by the action
  diagrams.
\end{proof}

\begin{definition}
  \label{def:cofork}
  Given two parallel arrows $f,g : B \mto C$ in a category $\cat{C}$,
  a \textbf{cofork} is a morphism $c : A \mto B$ such that
  the following diagram commutes:
  \[
  A \mto^c B \two^f_g C
  \]
  That is, $f \circ c = g \circ c$.
\end{definition}

\begin{lemma}
  \label{lemma:cofork-for-ex}
  The morphism $\e{A,B} \circ (\h{A} \otimes \h{B})$ is a cofork of
  the morphisms $(\h{B} \otimes \h{A}) \circ (\varepsilon_B \otimes
  \varepsilon_A)$ and $(e\varepsilon_B \otimes e\varepsilon_A) \circ
  (\delta_B \otimes \delta_A)$.
\end{lemma}
\begin{proof}
  We prove this by equational reasoning as follows:
  \[
  \small
  \begin{array}{lll}
    (\h{B} \otimes \h{A}) \circ (\varepsilon_B \otimes \varepsilon_A) \circ \e{A,B} \circ (\h{A} \otimes \h{B})
    \\\,\,\,\,\,\,\,\,\,\,\,
    = (\h{B} \otimes \h{A}) \circ (\varepsilon_B \otimes \varepsilon_A) \circ (\h{B} \otimes \h{A}) \circ \beta_{A,B}
    & \text{(Corollary~\ref{corollary:ex-simple})}\\
    \,\,\,\,\,\,\,\,\,\,\,= (\h{B} \otimes \h{A}) \circ ((\varepsilon_B \circ \h{B}) \otimes (\varepsilon_A \circ \h{A})) \circ \beta_{A,B}
    & \text{}\\
    \,\,\,\,\,\,\,\,\,\,\,= (\h{B} \otimes \h{A}) \circ (\id_B \otimes \id_A) \circ \beta_{A,B}
    & \text{(Action diagrams)}\\
    \,\,\,\,\,\,\,\,\,\,\,= (\h{B} \otimes \h{A}) \circ \beta_{A,B}
    & \text{}\\
    \,\,\,\,\,\,\,\,\,\,\,= \e{A,B} \circ (\h{A} \otimes \h{B})
    & \text{(Corollary~\ref{corollary:ex-simple})}\\
    \,\,\,\,\,\,\,\,\,\,\,= (\id_B \otimes \id_A) \circ \e{A,B} \circ (\h{A} \otimes \h{B})
    & \text{}\\
    \,\,\,\,\,\,\,\,\,\,\,= ((e\varepsilon_B \circ \delta_B) \otimes (e\varepsilon_A \circ \delta_A)) \circ \e{A,B} \circ (\h{A} \otimes \h{B})
    & \text{(Monoidal Comonad)}\\
    \,\,\,\,\,\,\,\,\,\,\,= (e\varepsilon_B \otimes e\varepsilon_A) \circ (\delta_B \otimes \delta_A) \circ \e{A,B} \circ (\h{A} \otimes \h{B})
    & \text{}\\
  \end{array}
  \]
\end{proof}

\begin{lemma}
  \label{lemma:beta-coalgebra-morph}
  In $\cat{L}^e$, $\beta$ is a coalgebra morphism.
\end{lemma}
\begin{proof}
  The proof follows from the commutativity of the following diagram:
  \[
  \small
  \bfig
  \square|amma|<800,500>[
    A \otimes B`
    eA \otimes eB`
    eA \otimes eB`
    e^2A \otimes e^2B;
    h_A \otimes h_B`
    h_A \otimes h_B`
    \delta_A \otimes \delta_B`
    eh_A \otimes eh_B]

  \square(800,0)|amma|<800,500>[
    eA \otimes eB`
    eB \otimes eA`
    e^2A \otimes e^2B`
    e^2B \otimes e^2A;
    \e{A,B}`
    \delta_A \otimes \delta_B`
    \delta_B \otimes \delta_A`
    \e{eA,eB}]

  \square(1600,0)|amma|<800,500>[
    eB \otimes eA`
    B \otimes A`
    e^2B \otimes e^2A`
    eA \otimes eB;
    \varepsilon_B \otimes \varepsilon_A`
    \delta_B \otimes \delta_A`
    h_B \otimes h_A`
    e\varepsilon_B \otimes e\varepsilon_A]

  \square(0,-500)|amma|<800,500>[
    eA \otimes eB`
    e^2A \otimes e^2B`
    e(A \otimes B)`
    e(eA \otimes eB);
    eh_A \otimes eh_B`
    \q{A,B}`
    \q{eA,eB}`
    e(h_A \otimes h_B)]

  \square(800,-500)|amma|<800,500>[
    e^2A \otimes e^2B`
    e^2B \otimes e^2A`
    e(eA \otimes eB)`
    e(eB \otimes eA);
    \e{eA,eB}`
    \q{eA,eB}`
    \q{eB,eA}`
    e\e{A,B}]

  \square(1600,-500)|amma|<800,500>[
    e^2B \otimes e^2A`
    eA \otimes eB`
    e(eB \otimes eA)`
    e(B \otimes A);
    e\varepsilon_B \otimes e\varepsilon_A`
    \q{eB,eA}`
    \q{B,A}`
    e(\varepsilon_B \otimes \varepsilon_A)]

  \place(400,250)[\text{(1)}]
  \place(1200,250)[\text{(2)}]
  \place(2000,250)[\text{(3)}]
  \place(400,-250)[\text{(4)}]
  \place(1200,-250)[\text{(5)}]
  \place(2000,-250)[\text{(6)}]
  \efig 
  \]
  Diagram one commutes by the action diagrams for the coalgebras
  $(A,h_A)$ and $(B,h_B)$, diagram two commutes because $\cat{L}$ is a
  Lambek category with exchange, diagram three does not commute, but
  holds by Lemma~\ref{lemma:cofork-for-ex}, diagram four and six
  commute by naturality of $\q{}$, and diagram five commutes because
  $\cat{L}$ is a Lambek category with exchange.
\end{proof}

\begin{lemma}[The Eilenberg-Moore Category is Symmetric Monoidal]
  \label{lemma:the_eilenberg-moore_category_is_symmetric_monoidal}
  The category $\cat{L}^e$ is symmetric monoidal.
\end{lemma}
\begin{proof}
  The following diagram shows that $\beta_{B,A} \circ \beta_{A,B} = \id_{A \otimes B}$:
  \[
  \bfig
  \square|amma|<800,500>[
    A \otimes B`
    eA \otimes eB`
    eA \otimes eB`
    e^2A \otimes e^2 B;
    h_A \otimes h_B`
    h_A \otimes h_B`
    \delta_A \otimes \delta_B`
    \delta_A \otimes \delta_B]

  \square(0,-500)|amma|<800,500>[
    eA \otimes eB`
    e^2A \otimes e^2B`
    eB \otimes eA`
    e^2B \otimes e^2A;
    \delta_A \otimes \delta_B`
    \e{A,B}`
    \e{eA,eB}`
    \delta_B \otimes \delta_A]

  \square(0,-1000)|ammm|<800,500>[
    eB \otimes eA`
    e^2B \otimes e^2A`
    B \otimes A`
    eB \otimes eA;
    \delta_B \otimes \delta_A`
    \varepsilon_B \otimes \varepsilon_A`
    e\varepsilon_{B} \otimes e\varepsilon_{A}`
    h_B \otimes h_A]

  \square(800,-1000)|ammm|<1200,500>[
    e^2B \otimes e^2A`
    e^2A \otimes e^2B`
    eB \otimes eA`
    eA \otimes eB;
    \e{eB,eA}`
    e\varepsilon_{B} \otimes e\varepsilon_{A}`
    e\varepsilon_{A} \otimes e\varepsilon_{B}`
    \e{B,A}]

  \btriangle(800,-500)|mmm|/`=`/<1200,500>[
    e^2A \otimes e^2B`
    e^2B \otimes e^2A`
    e^2A \otimes e^2B;``]

  \morphism(2000,-1000)<600,0>[
    eA \otimes eB`
    A \otimes B;
    \varepsilon_A \otimes \varepsilon_B]

  \morphism(0,500)/{@{=}@/^10em/}/<2600,-1500>[
    A \otimes B`
    A \otimes B;]

  \place(400,250)[(1)]
  \place(400,-250)[(2)]
  \place(400,-750)[(3)]
  \place(1100,-300)[(4)]
  \place(1400,-750)[(5)]
  \place(1400,250)[(6)]
  \efig
  \]
  Diagram one trivially commutes, diagram two, four, and five commute
  because $\cat{L}$ is a Lambek category with exchange, diagram three
  does not commute, but holds by Lemma~\ref{lemma:cofork-for-ex},
  diagrams six, seven, and eight commute by the fact that
  $(e,\varepsilon,\delta)$ is a comonad and the action diagrams of the
  Eilenberg Moore category.
  
  At this point we must verify that $\beta$ respects the coherence
  diagrams of a symmetric monoidal category; see
  Definition~\ref{def:sym-monoidal-category}.  Thus, we must show that
  each of the following diagrams hold:
  \begin{description}
  \item[Case]
    \[
    \bfig
      \hSquares|aammmaa|/->`->`->``->`->`->/[
        (A \otimes B) \otimes C`
        A \otimes (B \otimes C)`
        (B \otimes C) \otimes A`
        (B \otimes A) \otimes C`
        B \otimes (A \otimes C)`
        B \otimes (C \otimes A);
        \alpha_{A,B,C}`
        \beta_{A,B \otimes C}`
        \beta_{A,B} \otimes \id_C``
        \alpha_{B,C,A}`
        \alpha_{B,A,C}`
        \id_B \otimes \beta_{A,C}]
      \efig      
      \]
      We can show that this diagram commutes, by reducing it to the
      corresponding diagram on free coalgebras which we know holds by
      the assumption that $\cat{L}$ is a Lambek category with
      exchange.  This reduction is as follows (due to the size of the
      diagram it is broken up into three diagrams that can be
      straightforwardly composed):
      \begin{enumerate}
      \item[] Diagram 1:
        \[
        \bfig
        \square|amma|<1000,500>[
          (eB \otimes eA) \otimes C`
          (e^2B \otimes e^2A) \otimes C`
          (B \otimes A) \otimes C`
          (eB \otimes eA) \otimes C;
          (\delta \otimes \delta) \otimes \id`
          (\varepsilon \otimes \varepsilon) \otimes \id`
          (e\varepsilon \otimes e\varepsilon) \otimes \id`
          (h_B \otimes h_A) \otimes \id]


        \square|amma|/{@{->}@/^2em/}`->`->`/<2000,500>[
          (eB \otimes eA) \otimes C`
          (eB \otimes eA) \otimes eC`
          (B \otimes A) \otimes C`
          eB \otimes (eA \otimes eC);
          \id \otimes h_C`
          (\varepsilon \otimes \varepsilon) \otimes \id`
          \alpha`]

        \morphism(1000,0)|m|<1000,500>[
          (eB \otimes eA) \otimes C`
          (eB \otimes eA) \otimes eC;
          \id \otimes h_C]

        \square(0,-500)|amma|/`=`<-`->/<2000,500>[
          (B \otimes A) \otimes C`
          eB \otimes (eA \otimes eC)`
          (B \otimes A) \otimes C`
          B \otimes (A \otimes C);``
          h_B \otimes (h_A \otimes h_C)`
          \alpha]

        \square(0,-1000)|amma|/->`=`->`/<2000,500>[
          (B \otimes A) \otimes C`
          B \otimes (A \otimes C)`
          (B \otimes A) \otimes C`
          B \otimes (eA \otimes eC);
          \alpha``
          \id \otimes (h_A \otimes h_C)`]

        \morphism(0,-1000)<1000,0>[
          (B \otimes A) \otimes C`
          B \otimes (A \otimes C);
          \alpha]

        \morphism(1000,-1000)<1000,0>[
          B \otimes (A \otimes C)`
          B \otimes (eA \otimes eC);
          \id \otimes (h_A \otimes h_C)]

        \square(0,500)|amma|/->`->`->`/<2000,500>[
          (eA \otimes eB) \otimes C`
          (eA \otimes eB) \otimes eC`
          (eB \otimes eA) \otimes C`
          (eB \otimes eA) \otimes eC;
          \id \otimes h_C`
          \e{} \otimes \id`
          \e{} \otimes \id`]

        \square(0,1000)|amma|/=`->`->`->/<2000,500>[
          (A \otimes B) \otimes C`
          (A \otimes B) \otimes C`
          (eA \otimes eB) \otimes C`
          (eA \otimes eB) \otimes eC;`
          (h_A \otimes h_B) \otimes \id`
          (h_A \otimes h_B) \otimes h_C`
          \id \otimes h_C]

        \place(500,250)[(2)]
        \place(1450,450)[(1)]
        \place(1000,-250)[(3)]
        \efig
        \]        
        Diagram one commutes because $(e,\varepsilon,\delta)$ is a
        comonad, diagram two does not commute, but holds by
        Lemma~\ref{lemma:cofork-for-ex}, and diagram 3 commutes by
        naturality of $\alpha$.

      \item[] Diagram 2:
      \item[] Diagram 3:        
      \end{enumerate}

    \item[Case]
      \[
      \bfig
      \btriangle[
        A \otimes B`
        B \otimes A`
        A \otimes B;
        \beta_{A,B}`
        \id_{A \otimes B}`
        \beta_{B,A}]
      \efig
      \]

    \item[Case]
      \[
      \bfig
      \Vtriangle[
        \top \otimes A`
        A \otimes \top`
        A;
        \beta_{\top,A}`
        \lambda_A`
        \rho_A]
      \efig
      \]
  \end{description}
\end{proof}

\begin{definition}
  \label{def:cokleisli-exchange}
  Let $(\cat{L},e,\e{})$ be a Lambek category with exchange. The
  \textbf{coKleisli Category of $e$}, $\cat{L}_e$, is a category with the
  same objects as $\cat{L}$. There is an arrow $\hat{f}:A\mto B$ in
  $\cat{L}_e$ if there is an arrow $f:eA\mto B$ in $\cat{L}$. The
  identity arrow $\hat{id_A}:A\mto A$ is the arrow
  $\varepsilon_A:eA\mto A$ in $\cat{L}$. Given $\hat{f}:A\mto B$ and
  $\hat{g}:B\mto C$ in $\cat{L}_e$, which are arrows
  $f:eA\mto B$ and $g:eB\mto C$ in $\cat{L}$, the composition
  $\hat{g}\circ\hat{f}:A\mto C$ is defined as $g\circ ef\circ\delta_A$.
\end{definition}


% subsection lambek_categories_with_exchange (end)

\subsection{Linear Categories}
\label{subsec:linear_categories}

\begin{definition}
  \label{def:linear-category}
  A \textbf{linear category}, $(\cat{L},!,\w{},\c{})$, is a symmetric
  monoidal closed category $(\cat{L},I,\otimes,\limp)$ equipped with a
  symmetric monoidal comonad $(!,\varepsilon,\delta)$ with
  $\q{A,B}:!A\otimes !B\mto !(A\otimes B)$ and $\q{I}:I\mto !I$, and two
  monoidal natural transformations with components $\w{A}:!A\mto I$ and
  $\c{A}:!A\mto !A\otimes !A$, satisfying the following conditions:
  \begin{itemize}
  \item each $(!A,\w{A},\c{A})$ is a commutative comonoid, i.e.
    the following diagrams commute and $\beta\circ\c{A}=\c{A}$ where
    $\beta_{B,C}:B\otimes C\mto C\otimes B$ is the symmetry natural
    transformation of $\cat{L}$;
    \begin{mathpar}
      \bfig
      \square/->`->``/<1050,400>[
        !A`
        !A\otimes !A`
        !A\otimes !A`;
        \c{A}`
        \c{A}``]
      \square(1050,0)/->``<-`/<1050,400>[
        !A\otimes !A`
        !A\otimes(!A\otimes !A)``
        (!A\otimes !A)\otimes !A;
        id_{!A}\otimes\c{A}``
        \alpha_{!A,!A,!A}`]
        \morphism(0,0)|b|<2100,0>[
          !A\otimes !A`
          (!A\otimes !A)\otimes !A;
          \c{A}\otimes id_{!A}]
      \efig
    \end{mathpar}
    \begin{mathpar}
    \bfig
    \Atrianglepair/->`->`->`<-`->/<800,400>[
      !A`
      I\otimes !A`
      !A\otimes !A`
      !A\otimes I;
      \lambda^{-1}`
      \c{A}`
      \rho^{-1}`
      \w{A}\otimes id_{!A}`
      id_{!A}\otimes\w{A}]
    \efig
    \end{mathpar}

  \item $\w{A}$ and $\c{A}$ are coalgebra morphisms, i.e. the
    following diagrams commute;
    \begin{mathpar}
    \bfig
      \square/->`->`->`->/<1000,500>[
      !A`
      I`
      !!A`
      !I;
      \w{A}`
      \delta{A}`
      \q{I}`
      !\w{A}]
    \efig
    \end{mathpar}
    \begin{mathpar}
    \bfig
      \square/->`->``/<1050,400>[
        !A`
        !A\otimes !A`
        !!A`;
        \c{A}`
        \delta_A``]
      \square(1050,0)/->``->`/<1050,400>[
        !A\otimes !A`
        !!A\otimes !!A``
        !(!A\otimes !A);
        \delta_A\otimes\delta_A``
        \q{!A,!A}`]
        \morphism(0,0)|b|<2100,0>[!!A`!(!A\otimes !A);!\c{A}]
    \efig
    \end{mathpar}

  \item any coalgebra morphism $f:(!A,\delta_A)\mto (!B,\delta_B)$
    between free coalgebras preserve the comonoid structure given
    by $\w{}$ and $\c{}$, i.e. the following diagrams commute.
    \begin{mathpar}
    \bfig
      \Vtriangle/->`->`->/<600,400>[
        !A`
        !B`
        I;
        f`
        \w{A}`
        \w{B}]
    \efig
    \and
    \bfig
      \square/->`->`->`->/<800,400>[
      !A`
      !A\otimes !A`
      !B`
      !B\otimes !B;
      \c{A}`
      f`
      f\otimes f`
      \c{B}]
    \efig
    \end{mathpar}

  \end{itemize}
\end{definition}

\begin{definition}
  \label{def:dist}
  Given two comonads $(c,\varepsilon^c,\delta^c)$ and
  $(w,\varepsilon^w,\delta^w)$ on a category $\cat{L}$ such that
  $(\cat{L},c,\cL{},\cR{})$ is a Lambek category with contraction and
  $(\cat{L},w,\w{})$ is a Lambek category with weakening, we define a
  \textbf{distributive law} of $c$ over $w$ to be a natural
  transformation with components $dist_A:cwA\mto wcA$, subject to the
  following coherence diagrams:
  \begin{mathpar}
    \bfig
      \Vtriangle/<-`<-`->/<500,400>[
        wA`
        cwA`
        wcA;
        \varepsilon_{wA}^c`
        w\varepsilon_A^c`
        dist_A]
    \efig
    \and
    \bfig
      \Vtriangle/<-`<-`->/<500,400>[
        cA`
        cwA`
        wcA;
        c\varepsilon_A^w`
        \varepsilon_{cA}^w`
        dist_A]
    \efig
  \end{mathpar}
\end{definition}

\begin{lemma}
  \label{lem:dist}
  Given two comonads $(c,\varepsilon^c,\delta^c)$ and
  $(w,\varepsilon^w,\delta^w)$ on a category $\cat{L}$ such that
  $(\cat{L},c,\cL{},\cR{})$ is a Lambek category with contraction and
  $(\cat{L},w,\w{})$ is a Lambek category with weakening, the following
  two diagrams commute:
  \begin{mathpar}
  \bfig
    \Vtriangle/->`->`<-/<500,400>[
      cwcA`
      cwc^2A`
      c^2wcA;
      cw\delta_A^c`
      \delta_{wcA}^c`
      cdist_{cA}]
  \efig
  \and
  \bfig
    \Vtriangle/->`->`->/<500,400>[
      wcwA`
      wcw^2A`
      w^2cwA;
      wc\delta_A^w`
      \delta_{cwA}^w`
      wdist_{wA}]
  \efig
  \end{mathpar}
\end{lemma}

\begin{proof}
  The two diagrams above commute because the following ones commute by the
  distributive law and the comonad laws for $c$ and $w$.
  \begin{mathpar}
  \bfig
    \Vtriangle|amm|/->`=`->/<600,300>[
      cwcA`
      cwc^2A`
      cwcA;
      cw\delta_A^c``
      cw\varepsilon_{cA}^c]
    \morphism(0,300)|b|<600,-700>[cwcA`c^2wcA;\delta_{wA}^c]
    \morphism(600,-400)|m|<0,400>[c^2wcA`cwcA;c\varepsilon_{wcA}^c]
    \morphism(600,-400)|b|<600,700>[c^2wcA`cwc^2A;cdist_{cA}]
  \efig
  \and
  \bfig
    \Vtriangle|amm|/->`=`->/<600,300>[
      wcwA`
      wcw^2A`
      wcwA;
      wc\delta_A^w``
      wc\varepsilon_{wA}^w]
    \morphism(0,300)|b|<600,-700>[wcwA`w^2cwA;\delta_{cwA}^w]
    \morphism(600,-400)|m|<0,400>[w^2cwA`wcwA;w\varepsilon_{cwA}^w]
    \morphism(1200,300)|b|<-600,-700>[wcw^2A`w^2cwA;wdist_{wA}]
  \efig
  \end{mathpar}
\end{proof}

% Lemma for composing contraction and weakening
\begin{lemma}[Composition of Weakening and Contraction]
  \label{lem:compose-cw}
  Suppose \\ $(\cat{L},I,\otimes,w,\w{}^w,c,\cL{},\cR{})$ is a Lambek
  category with weakening and contraction, where
  $(w,\varepsilon^w,\delta^w)$ and $(c,\varepsilon^c,\delta^c)$ are
  the respective monoidal comonads. Then the composition of $c$ and
  $w$ using the distributive law $dist_A:cwA\mto wcA$ is a monoidal
  comonad on $\cat{L}$.
\end{lemma}
\begin{proof}
  For the complete proof see
  Appendix~\ref{subsec:proof_of_composition_of_weakening_and_contraction_lem:compose-cw}.
\end{proof}

% Definition: Lambek category with cw
\begin{definition}
  \label{def:Lambek-cw}
  A \textbf{Lambek category with $cw$},
  $(\cat{L},cw,\w{}^w,\cL{},\cR{}, dist)$, is a Lambek category with
  weakening and contraction, and a distributive law.  Furthermore, the
  following coherence diagrams commute:
  \begin{mathpar}
    \bfig
      \square|amma|/->`->`<-`->/<1200,400>[
        I\otimes cwA`
        I\otimes (I\otimes cwA)`
        cwA\otimes(I\otimes cwA)`
        wA\otimes(I\otimes cwA);
        \lambda_{I\otimes cwA}^{-1}`
        \cR{wA,I}`
        \w{A}^w\otimes id_{I\otimes cwA}`
        \varepsilon_{wA}^c\otimes id_{I\otimes cwA}]
    \efig
    \and
    \bfig
      \square|amma|/->`->`<-`->/<1200,400>[
        cwA\otimes I`
        (cwA\otimes I)\otimes I`
        (cwA\otimes I)\otimes cwA`
        (cwA\otimes I)\otimes wA;
        \rho_{cwA\otimes I}^{-1}`
        \cL{wA,I}`
        id_{cwA\otimes I}\otimes\w{A}^w`
        id_{cwA\otimes I}\otimes\varepsilon_{wA}^c]
    \efig
    \and
    \bfig
    \square|amma|/->`->`->`/<1000,400>[
      cwA`
      cwB`
      wA`
      wB;
      f`
      \varepsilon_{wA}^c`
      \varepsilon_{wB}^c`]
    \morphism<500,0>[wA`I;\w{A}^w]
    \morphism(1000,0)<-500,0>[wB`I;\w{B}^w]
    \efig
  \end{mathpar}  
\iffalse
  \begin{mathpar}
  \bfig
    \Vtriangle/->`->`->/<600,400>[
      cwA`
      cwB`
      I;
      f`
      \w{A}`
      \w{B}]
  \efig
  \end{mathpar}
\fi
  where $f:(cwA,\delta_A)\mto(cwB,\delta_B)$ is any coalgebra morphism between
  free coalgebras.
\end{definition}



% Conditions for weakening and contraction
\begin{lemma}
  \label{lem:compose-cw-2}
  Let $(\cat{L},cw,\w{}^w,\cL{},\cR{})$ be a Lambek category with $cw$.
  Then the following conditions are satisfied:
  \begin{itemize}
    \item[1.] There exist two natural transformations $\w{A}:cwA\mto I$
      and $\c{A}:cwA\mto cwA\otimes cwA$.
    \item[2.] Each $(cwA,\w{A},\c{A})$ is a comonoid.
    \item[3.] $\w{A}$ and $\c{A}$ are coalgebra morphisms.
    \item[4.] Any coalgebra morphism $f:(cwA,\delta_A)\mto(cwB,\delta_B)$
      between free coalgebras preserves the comonoid structure given by
      $\w{}$ and $\c{}$.
  \end{itemize}
\end{lemma}

\begin{proof}
  We will only prove the first condition by defining $\w{}$ and $\c{}$. For the complete proof see
  Appendix~\ref{subsec:proof_of_conditions_of_lambek_with_cw_lem:compose-cw-2}.
    Each of $\w{}$ and $\c{}$can be given two equivalent definitions.
    $\w{A}:cwA\mto I$ is defined as in the diagram below. The left
    triangle commutes by the definition of $dist$ and the right triangle
    commutes by the definition of $\w{}^w$.
    \begin{mathpar}
    \bfig
    \Atrianglepair/<-`->`->`->`->/<600,400>[
      wcA`
      cwA`
      wA`
      I;
      dist_A`
      w\varepsilon_A^c`
      \w{cA}^w`
      \varepsilon_{wA}^c`
      \w{A}^w]
    \efig
    \end{mathpar}

    $\c{A}:cwA\mto cwA\otimes cwA$ is defined as below. The left part of
    the diagram commutes by the definitions of $\cL{}$ and of $\cR{}$, and
    the right part commutes because $\cat{L}$ is monoidal.
    \begin{mathpar}
      \bfig
      \square/->`->``->/<1050,400>[
        cwA`
        cwA\otimes I`
        I\otimes cwA`
        cwA\otimes(I\otimes cwA);
        \rho_{cwA}^{-1}`
        \lambda_{cwA}^{-1}``
        \cR{wA,I}]
      \ptriangle(1050,0)/->``/<1050,400>[
        cwA\otimes I`
        (cwA\otimes I)\otimes cwA`
        cwA\otimes(I\otimes cwA);
        \cL{wA,I}``]
      \dtriangle(1050,0)|mrb|/->`->`->/<1050,400>[
        (cwA\otimes I)\otimes cwA`
        cwA\otimes(I\otimes cwA)`
        cwA\otimes cwA;
        \alpha_{cwA,I,cwA}`
        \rho_{cwA}\otimes id_{cwA}`
        id_{cwA}\otimes\lambda_{cwA}]
      \efig
    \end{mathpar}
\end{proof}


% Definition: distributive law for exchange
\begin{definition}
  \label{def:distEx}
  Given two comonads $(cw,\varepsilon^{cw},\delta^{cw})$ and
  $(e,\varepsilon^e,\delta^e)$ on a category $\cat{L}$ such that
  $(\cat{L},cw,\w{},\c{})$ is a Lambek category with $cw$ and
  $(\cat{L},e,\e{})$ is a Lambek category with exchange, we define a
  \textbf{distributive law for exchange} of $cw$ over $e$ to be a natural
  isomorphism with components $distEx_A:cweA\mto ecwA$, subject to the
  following coherence diagrams:
  \begin{mathpar}
    \bfig
      \Vtriangle/<-`<-`->/<500,400>[
        eA`
        cweA`
        ecwA;
        \varepsilon_{eA}^{cw}`
        e\varepsilon_A^{cw}`
        distEx_A]
    \efig
    \and
    \bfig
      \Vtriangle/<-`<-`->/<500,400>[
        cwA`
        cweA`
        ecwA;
        cw\varepsilon_A^e`
        \varepsilon_{cwA}^e`
        distEx_A]
    \efig
  \end{mathpar}
\end{definition}

\begin{lemma}
  \label{lem:distEx}
  Given two comonads $(cw,\varepsilon^{cw},\delta^{cw})$ and
  $(e,\varepsilon^e,\delta^e)$ on a category $\cat{L}$ such that
  $(\cat{L},cw,\w{},\c{})$ is a Lambek category with $cw$ and
  $(\cat{L},e,\e{})$ is a Lambek category with exchange, the following
  two digrams also commute:
  \begin{mathpar}
  \bfig
    \Vtriangle/->`->`<-/<500,400>[
      cwecwA`
      cwe(cw)^2A`
      (cw)^2ecwA;
      cwe\delta_A^{cw}`
      \delta_{cweA}^{cw}`
      cwdistEx_{cwA}]
  \efig
  \and
  \bfig
    \Vtriangle/->`->`->/<500,400>[
      ecweA`
      ecwe^2A`
      e^2cweA;
      ecw\delta_A^e`
      \delta_{cweA}^e`
      edistEx_{eA}]
  \efig
  \end{mathpar}
\end{lemma}

  The proof is similar with the proof of Lemma~\ref{lem:dist} and we will
  not elaborate it here. Also, notice the difference between $dist$ of $c$
  over $w$ and $distEx$ of $cw$ over $e$. While $dist$ is a natural
  transformation, $distEx$ is a natural isomorphism.


% Lemma: cwe is a monoidal comonad
\begin{lemma}
  \label{lem:compose-cwe}
  let $(cw,\varepsilon^{cw},\delta^{cw})$ and $(e,\varepsilon^e,\delta^e)$
  be two monoidal comonads on a Lambek category with $cw$ and exchange
  $(\cat{L},I,\otimes,cw,\w{},\c{},e,\e{})$. Then the composition of $cw$
  and $e$ using the distributive law for exchange $distEx_A:cweA\mto ecwA$
  is a monoidal comonad $(cwe,\varepsilon,\delta)$ on $\cat{L}$.
\end{lemma}

\begin{proof}
  Suppose $(cw,\varepsilon^{cw},\delta^{cw})$ and
  $(e,\varepsilon^e,\delta^e)$ are monoidal comonads, and \\
  $(\cat{L},I,\otimes,cw,\w{},\c{},e,\e{})$ is a
  Lambek category with $cw$ and exchange. Since by definition
  $cw,e:\cat{L}\mto\cat{L}$ are monoidal functors, we know that their
  composition \\
  $cwe:\cat{L}\mto\cat{L}$ is a monoidal functor:
  \begin{align*}
    \q{A,B} &: cweA\otimes cweB\mto cwe(A\otimes B)   \\
    \q{A,B} &= cw\q{A,B}^e\circ\q{eA,eB}^{cw}         \\
    \q{I}   &: I\mto cweI                             \\
    \q{I}   &= cw\q{I}^e\circ\q{I}^{cw}
  \end{align*}
  Analogous to the proof of Lemma~\ref{lem:compose-cw}, each of
  $\varepsilon$ and $\delta$ can be given two equivalent definitions:
  \begin{mathpar}
  \bfig
    \square<800,400>[
      cweA`
      eA`
      cwA`
      A;
      \varepsilon_{eA}^{cw}`
      cw\varepsilon_A^e`
      \varepsilon_A^e`
      \varepsilon_A^{cw}]
  \efig
  \and
  \bfig
    \square|almb|/->`->`->`->/<800,500>[
      cweA`
      cwe^2A`
      (cw)^2eA`
      (cw)^2e^2A;
      cw\delta_A^e`
      \delta_{eA}^{cw}`
      \delta_{e^2A}^{cw}`
      (cw)^2\delta_A^e]
    \square(800,0)|amrb|/->``->`->/<800,500>[
      cwe^2A`
      (cw)^2e^2A`
      (cw)^2e^2A`
      cwecweA;
      \delta_{e^2A}^{cw}``
      cwdist_{eA}`
      cwdist_{eA}]
  \efig
  \end{mathpar}
  And the comonad laws can be proved similarly, which we will not elaborate
  for simplicity.
\end{proof}



% Lemma: co-Kleisli of cwe is symmetric monoidal
\begin{lemma}
  \label{lem:cokleisli-cwe}
  Let $(cwe,\varepsilon,\delta)$ be a monoidal comonad over a monoidal
  category $(\cat{L},I,\otimes)$ such that
  $(\cat{L},I,\otimes,cw,\w{},\c{},e,\e{})$ is a Lambek category with $cw$
  and exchange. Then the co-Kleisli category of $\cat{L}$, $\cat{L}_{cwe}$,
  is a linear category.
  \todo{changed to liner category. Finish the proof when lemma 5 is proved.}
\end{lemma}
\begin{proof}
  The identity object of $\cat{L}_{cwe}$ is still $I$.

  The left and right unitors, $\hat\lambda_A:I\otimes A\mto A$ and
  $\hat\rho_A:A\otimes I\mto A$, in $\cat{L}_{cwe}$ are morphisms
  $cwe(I\otimes A)\mto A$ and $cwe(A\otimes I)\mto A$ in $\cat{L}$,
  respectively. Then we define $\hat\lambda$ and $\hat\rho$ as:
  \begin{align*}
    \hat\lambda_A &= \varepsilon_A\circ cwe\lambda_A     \\
    \hat\rho_A    &= \varepsilon_A\circ cwe\rho_A,
  \end{align*}
  where $\lambda$ and $\rho$ are the left and right unitors in $\cat{L}$,
  respectively. And we define their inverses as:
  \begin{align*}
    \hat\lambda_A^{-1} &= \varepsilon_{I\otimes A}\circ cwe\lambda_A^{-1} \\
    \hat\rho_A^{-1}    &= \varepsilon_{A\otimes I}\circ cwe\rho_A^{-1}
  \end{align*}
  $\hat\lambda$ is a nautral isomorphism with inverse $\hat\lambda^{-1}$
  because the following diagram chasing commutes:
  \begin{mathpar}
  \bfig
    \Vtrianglepair|aammm|/->`->`->``<-/<800,400>[
      cwe(I\otimes A)`
      (cwe)^2(I\otimes A)`
      (cwe)^2A`
      cweA;
      \delta_{I\otimes A}`
      (cwe)^2\lambda_A`
      cwe\lambda_A``
      \delta_A]
    \btriangle(0,-400)/->`=`<-/<800,800>[
      cwe(I\otimes A)`
      I\otimes A`
      cwe(I\otimes A);
      \varepsilon_{I\otimes A}``
      \varepsilon_{I\otimes A}]
    \btriangle(800,-400)|mmb|/->`=`<-/<800,400>[
      cweA`
      cwe(I\otimes A)`
      cweA;
      cwe\lambda_A^{-1}``
      cwe\lambda_A^{-1}]
    \morphism(1600,400)|r|<0,-800>[(cwe)^2A`cweA;cwe\varepsilon_A]
    \ptriangle(800,100)/``/<100,100>[(1)``;``]
    \ptriangle(300,-250)/``/<100,100>[(2)``;``]
    \ptriangle(600,-100)/``/<100,100>[(3)``;``]
    \ptriangle(1000,-350)/``/<100,100>[(4)``;``]
    \ptriangle(1300,-100)/``/<100,100>[(5)``;``]
  \efig
  \end{mathpar}
  (1) commutes by the naturality of $\delta$. (2), (3) and (4) commute
  trivially. And (5) commutes because $cwe$ is a comonad.

  Similarly, $\hat\rho$ is a natural isomorphism with inverse
  $\hat\rho^{-1}$ by the following diagram chasing:
  \begin{mathpar}
  \bfig
    \Vtrianglepair|aammm|/->`->`->``<-/<800,400>[
      cwe(A\otimes I)`
      (cwe)^2(A\otimes I)`
      (cwe)^2A`
      cweA;
      \delta_{A\otimes I}`
      (cwe)^2\rho_A`
      cwe\rho_A``
      \delta_A]
    \btriangle(0,-400)/->`=`<-/<800,800>[
      cwe(A\otimes I)`
      A\otimes I`
      cwe(A\otimes I);
      \varepsilon_{A\otimes I}``
      \varepsilon_{A\otimes I}]
    \btriangle(800,-400)|mmb|/->`=`<-/<800,400>[
      cweA`
      cwe(A\otimes I)`
      cweA;
      cwe\rho_A^{-1}``
      cwe\rho_A^{-1}]
    \morphism(1600,400)|r|<0,-800>[(cwe)^2A`cweA;cwe\varepsilon_A]
  \efig
  \end{mathpar}

  The associator
  $\hat\alpha_A:(A\otimes B)\otimes C)\mto A\otimes(B\otimes C)$ in
  $\cat{L}_{cwe}$ is the morphism
  $cwe((A\otimes B)\otimes C)\mto A\otimes(B\otimes C)$ in $\cat{L}$.
  We define $\hat\alpha$ as:
  $$\hat\alpha_{A,B,C}=\varepsilon_{A\otimes(B\otimes C)}\circ cwe\alpha_{A,B,C},$$
  where $\alpha$ is the associator of $\cat{L}$. And its inverse is
  $$\hat\alpha_{A,B,C}^{-1} =
    \varepsilon_{(A\otimes B)\otimes C}\circ cwe\alpha_{A,B,C}^{-1}$$
  $\hat\alpha$ is a natural isomorphism with inverse $\hat\alpha^{-1}$
  because the following diagram chasing commutes:
  \begin{mathpar}
  \bfig
    \Vtrianglepair|aammm|/->`->`->``<-/<1200,400>[
      cwe((A\otimes B)\otimes C)`
      (cwe)^2((A\otimes B)\otimes C)`
      (cwe)^2(A\otimes (B\otimes C))`
      cwe(A\otimes(B\otimes C));
      \delta_{(A\otimes B)\otimes C}`
      (cwe)^2\alpha_{A,B,C}`
      cwe\alpha_{A,B,C}``
      \delta_{A\otimes (B\otimes C)}]
    \btriangle(0,-400)/->`=`<-/<1200,800>[
      cwe((A\otimes B)\otimes C)`
      (A\otimes B)\otimes C`
      cwe((A\otimes B)\otimes C);
      \varepsilon_{(A\otimes B)\otimes C}``
      \varepsilon_{(A\otimes B)\otimes C}]
    \btriangle(1200,-400)|mmb|/->`=`<-/<1200,400>[
      cwe(A\otimes(B\otimes C))`
      cwe((A\otimes B)\otimes C)`
      cwe(A\otimes(B\otimes C));
      cwe\alpha_{A,B,C}^{-1}``
      cwe\alpha_{A,B,C}^{-1}]
    \morphism(2400,400)|r|<0,-800>[
      (cwe)^2(A\otimes (B\otimes C))`
      cwe(A\otimes(B\otimes C));
      cwe\varepsilon_{A\otimes(B\otimes C)}]
  \efig
  \end{mathpar}

  Therefore, $\cat{L}_{cwe}$ is a monoidal category.

  The symmetry, $\hat\beta_{A,B}:A\otimes B\mto B\otimes A$, in
  $\cat{L}_{cwe}$ is the morphism $cwe(A\otimes B)\mto B\otimes A$ in
  $\cat{L}$, which is defined as:
  $$\hat\beta_{A,B}=\varepsilon_{B\otimes A}^{cw}\circ cw\gamma_{A,B},$$
  where $\varepsilon_A^{cw}:cwA\mto A$ is a natural transformation
  associated with the comonad $cw$, and $\gamma$ is the natural
  isomorphism defined in Lemma~\ref{lem:cokleisli-exchange}. Then its
  inverse is
  $$\hat\beta_{A,B}^{-1}=\varepsilon_{A\otimes B}^{cw}\circ cw\gamma_{B,A}$$
  $\hat\beta$ is a natural isomorphism with inverse $\hat\beta^{-1}$
  because the following diagram chasing commutes:
  \begin{mathpar}
  \bfig
    \square|almm|/<-`<-`->`<-/<850,400>[
      A\otimes B`
      cw(A\otimes B)`
      cw(A\otimes B)`
      (cw)^2(A\otimes B);
      \varepsilon_{A\otimes B}^{cw}`
      \varepsilon_{A\otimes B}^{cw}`
      \delta_{A\otimes B}^{cw}`
      cw\varepsilon_{A\otimes B}^{cw}]
    \morphism/=/<850,400>[cw(A\otimes B)`cw(A\otimes B);]
    \square(850,0)|amrm|/<-``->`<-/<1850,400>[
      cw(A\otimes B)`
      cwe(A\otimes B)`
      (cw)^2(A\otimes B)`
      (cw)^2e(A\otimes B);
      cw\varepsilon_{A\otimes B}^e``
      \delta_{e(A\otimes B}^{cw}`
      (cw)^2\varepsilon_{A\otimes B}^e]
    \btriangle(850,-400)|mmm|/<-``<-/<1050,400>[
      (cw)^2(A\otimes B)`
      (cw)^2e(B\otimes A)`
      (cw)^2e^2(A\otimes B);
      (cw)^2\gamma_{B,A}``
      (cw)^2e\gamma_{A,B}]
    \morphism(2700,0)<0,-400>[
      (cw)^2e(A\otimes B)`
      (cw)^2e^2(A\otimes B);
      (cw)^2\delta_{A\otimes B}^e]
    \qtriangle(1900,-800)|mmr|/=`<-`->/<800,400>[
      (cw)^2e^2(A\otimes B)`
      (cw)^2e^2(A\otimes B)`
      (cwe)^2(A\otimes B);
      `
      cwdistEx_{e(A\otimes B)}^{-1}`
      cwdistEx_{e(A\otimes B)}]
    \morphism(850,-400)|m|/=/<0,-400>[
      (cw)^2e(B\otimes A)`
      (cw)^2e(B\otimes A);]
    \morphism(2700,-1200)|m|<-1850,800>[
      cwecw(B\otimes A)`
      (cw)^2e(B\otimes A);
      cwdistEx_{B\otimes A}^{-1}]
    \morphism(850,-800)|m|<1850,-400>[
      (cw)^2e(B\otimes A)`
      cwecw(B\otimes A);
      cwdistEx_{B\otimes A}]
    \morphism(2700,-800)|r|<0,-400>[
      (cwe)^2(A\otimes B)`
      cwecw(B\otimes A);
      cwecw\gamma_{A,B}]
    \btriangle(0,-1200)/<-``<-/<2700,1200>[
      cw(A\otimes B)`
      cwe(B\otimes A)`
      cwecw(B\otimes A);
      cw\gamma_{B,A}``
      cwe\varepsilon_{B\otimes A}^{cw}]
    \morphism(850,-400)|m|<-850,-800>[
      (cw)^2e(B\otimes A)`
      cwe(B\otimes A);
      cw\varepsilon_{e(B\otimes A)}^{cw}]
    \ptriangle(200,250)/``/<100,0>[(1)``;``]
    \ptriangle(550,100)/``/<100,0>[(2)``;``]
    \ptriangle(1775,200)/``/<100,0>[(3)``;``]
    \ptriangle(350,-400)/``/<100,0>[(4)``;``]
    \ptriangle(1775,-200)/``/<100,0>[(5)``;``]
    \ptriangle(850,-1000)/``/<100,0>[(6)``;``]
    \ptriangle(1200,-700)/``/<100,0>[(7)``;``]
    \ptriangle(1775,-600)/``/<100,0>[(8)``;``]
    \ptriangle(2550,-550)/``/<100,0>[(9)``;``]
  \efig
  \end{mathpar}
  (1), (7) and (9) commute trivially. (2) is the comonad law for $cw$. (3)
  commutes by the naturality of $\delta^{cw}$. (4) commutes by the
  naturality of $\varepsilon^{cw}$. (5) commutes because $\gamma$ is a
  natural isomorphism (Lemma~\ref{lem:cokleisli-exchange}). (6) is the
  definition of $distEx$. (8) is the naturality of $distEx$. 
  
  In conclusion, $\cat{L}_{cwe}$ is a symmetric monoidal category.
\end{proof}


% subsection linear_categories (end)

% section categorical_models (end)

\section{Related Work}
\label{sec:related_work}
TODO
% section related_work (end)

\section{Conclusion}
\label{sec:conclusion}
TODO
% section conclusion (end)

\bibliographystyle{plainurl} \bibliography{ref}

\appendix
\section{Appendix}
\label{sec:appendix}
\subsection{Symmetric Monoidal Categories}
\label{subsec:symmetric_monoidal_categories}

\begin{definition}
  \label{def:monoidal-category}
  A \textbf{monoidal category} is a category, $\cat{M}$,
  with the following data:
  \begin{itemize}
  \item An object $\top$ of $\cat{M}$,
  \item A bi-functor $\otimes : \cat{M} \times \cat{M} \mto \cat{M}$,
  \item The following natural isomorphisms:
    \[
    \begin{array}{lll}
      \lambda_A : \top \otimes A \mto A\\
      \rho_A : A \otimes \top \mto A\\      
      \alpha_{A,B,C} : (A \otimes B) \otimes C \mto A \otimes (B \otimes C)\\
    \end{array}
    \]
  \item Subject to the following coherence diagrams:
    \begin{mathpar}
      \bfig
      \vSquares|ammmmma|/->`->```->``<-/[
        ((A \otimes B) \otimes C) \otimes D`
        (A \otimes (B \otimes C)) \otimes D`
        (A \otimes B) \otimes (C \otimes D)``
        A \otimes (B \otimes (C \otimes D))`
        A \otimes ((B \otimes C) \otimes D);
        \alpha_{A,B,C} \otimes \id_D`
        \alpha_{A \otimes B,C,D}```
        \alpha_{A,B,C \otimes D}``
        \id_A \otimes \alpha_{B,C,D}]      
      
      \morphism(1185,1000)|m|<0,-1000>[
        (A \otimes (B \otimes C)) \otimes D`
        A \otimes ((B \otimes C) \otimes D);
        \alpha_{A,B \otimes C,D}]
      \efig
    \end{mathpar}
    \begin{mathpar}
      \bfig
      \Vtriangle[
        (A \otimes \top) \otimes B`
        A \otimes (\top \otimes B)`
        A \otimes B;
        \alpha_{A,\top,B}`
        \rho_{A}\otimes id_B`
        id_A\otimes\lambda_{B}]
      \efig
    \end{mathpar}
  \end{itemize}
\end{definition}

% Definition: symmetric monoidal category
\begin{definition}
  \label{def:sym-monoidal-category}
  A \textbf{symmetric monoidal category (SMC)} is a category, $\cat{M}$,
  with the following data:
  \begin{itemize}
  \item An object $\top$ of $\cat{M}$,
  \item A bi-functor $\otimes : \cat{M} \times \cat{M} \mto \cat{M}$,
  \item The following natural isomorphisms:
    \[
    \begin{array}{lll}
      \lambda_A : \top \otimes A \mto A\\
      \rho_A : A \otimes \top \mto A\\      
      \alpha_{A,B,C} : (A \otimes B) \otimes C \mto A \otimes (B \otimes C)\\
    \end{array}
    \]
  \item A symmetry natural isomorphism:
    \[
    \beta_{A,B} : A \otimes B \mto B \otimes A
    \]
  \item Subject to the following coherence diagrams:
    \begin{mathpar}
      \bfig
      \vSquares|ammmmma|/->`->```->``<-/[
        ((A \otimes B) \otimes C) \otimes D`
        (A \otimes (B \otimes C)) \otimes D`
        (A \otimes B) \otimes (C \otimes D)``
        A \otimes (B \otimes (C \otimes D))`
        A \otimes ((B \otimes C) \otimes D);
        \alpha_{A,B,C} \otimes \id_D`
        \alpha_{A \otimes B,C,D}```
        \alpha_{A,B,C \otimes D}``
        \id_A \otimes \alpha_{B,C,D}]      
      
      \morphism(1185,1000)|m|<0,-1000>[
        (A \otimes (B \otimes C)) \otimes D`
        A \otimes ((B \otimes C) \otimes D);
        \alpha_{A,B \otimes C,D}]
      \efig
      \and
      \bfig
      \hSquares|aammmaa|/->`->`->``->`->`->/[
        (A \otimes B) \otimes C`
        A \otimes (B \otimes C)`
        (B \otimes C) \otimes A`
        (B \otimes A) \otimes C`
        B \otimes (A \otimes C)`
        B \otimes (C \otimes A);
        \alpha_{A,B,C}`
        \beta_{A,B \otimes C}`
        \beta_{A,B} \otimes \id_C``
        \alpha_{B,C,A}`
        \alpha_{B,A,C}`
        \id_B \otimes \beta_{A,C}]
      \efig      
    \end{mathpar}
    \begin{mathpar}
      \bfig
      \Vtriangle[
        (A \otimes \top) \otimes B`
        A \otimes (\top \otimes B)`
        A \otimes B;
        \alpha_{A,\top,B}`
        \rho_{A}\otimes id_B`
        id_A\otimes\lambda_{B}]
      \efig
      \and
      \bfig
      \btriangle[
        A \otimes B`
        B \otimes A`
        A \otimes B;
        \beta_{A,B}`
        \id_{A \otimes B}`
        \beta_{B,A}]
      \efig
      \and
      \bfig
      \Vtriangle[
        \top \otimes A`
        A \otimes \top`
        A;
        \beta_{\top,A}`
        \lambda_A`
        \rho_A]
      \efig
    \end{mathpar}    
  \end{itemize}
\end{definition}

% Definition: monoidal biclosed category
\begin{definition}
  \label{def:monoidal-biclosed-category}
  A \textbf{monoidal biclosed category} is a monoidal category
  $(\cat{M},\top,\otimes)$, such that, for any object $B$ of $\cat{M}$,
  each of the functors $-\otimes B:\cat{M}\mto\cat{M}$ and
  $B\otimes -:\cat{M}\mto\cat{M}$ has a specified right adjoint. Hence,
  for any object $A$ and $C$ of $\cat{M}$, there are two objects
  $C\leftharpoonup B$ and $B\rightharpoonup C$ of $\cat{M}$ and two
  natural bijections:
  \begin{align*}
  \Hom{\cat{M}}{A\otimes B}{C} &\cong
    \Hom{\cat{M}}{A}{C\leftharpoonup B} \\
  \Hom{\cat{M}}{B \otimes A}{C} &\cong
    \Hom{\cat{M}}{A}{B \rightharpoonup C}
  \end{align*}
\end{definition}

% Definition: symmetric monoidal closed category
\begin{definition}
  \label{def:SMCC}
  A \textbf{symmetric monoidal closed category (SMCC)} is a symmetric
  monoidal category, $(\cat{M},\top,\otimes)$, such that, for any object
  $B$ of $\cat{M}$, the functor $- \otimes B : \cat{M} \mto \cat{M}$
  has a specified right adjoint.  Hence, for any objects $A$ and $C$
  of $\cat{M}$ there is an object $B \limp C$ of $\cat{M}$ and a
  natural bijection:
  \[
  \Hom{\cat{M}}{A \otimes B}{C} \cong \Hom{\cat{M}}{A}{B \limp C}
  \]
  We call the functor $\limp : \cat{M} \times \cat{M} \mto \cat{M}$
  the internal hom of $\cat{M}$.
\end{definition}

\begin{definition}
  \label{def:MCFUN}
  Suppose we are given two monoidal categories
  $(\cat{M}_1,\top_1,\otimes_1,\alpha_1,\lambda_1,\rho_1,\beta_1)$ and
  $(\cat{M}_2,\top_2,\otimes_2,\alpha_2,\lambda_2,\rho_2,\beta_2)$.  Then a
  \textbf{monoidal functor} is a functor $F : \cat{M}_1 \mto
  \cat{M}_2$, a map $m_{\top_1} : \top_2 \mto F\top_1$ and a natural transformation
  $m_{A,B} : FA \otimes_2 FB \mto F(A \otimes_1 B)$ subject to the
  following coherence conditions:
  \begin{mathpar}
    \bfig
    \vSquares|ammmmma|/->`->`->``->`->`->/[
      (FA \otimes_2 FB) \otimes_2 FC`
      FA \otimes_2 (FB \otimes_2 FC)`
      F(A \otimes_1 B) \otimes_2 FC`
      FA \otimes_2 F(B \otimes_1 C)`
      F((A \otimes_1 B) \otimes_1 C)`
      F(A \otimes_1 (B \otimes_1 C));
      {\alpha_2}_{FA,FB,FC}`
      m_{A,B} \otimes \id_{FC}`
      \id_{FA} \otimes m_{B,C}``
      m_{A \otimes_1 B,C}`
      m_{A,B \otimes_1 C}`
      F{\alpha_1}_{A,B,C}]
    \efig
    \end{mathpar}
  \begin{mathpar}
    \bfig
    \square|amma|/->`->`<-`->/<1000,500>[
      \top_2 \otimes_2 FA`
      FA`
      F\top_1 \otimes_2 FA`
      F(\top_1 \otimes_1 A);
      {\lambda_2}_{FA}`
      m_{\top_1} \otimes \id_{FA}`
      F{\lambda_1}_{A}`
      m_{\top_1,A}]
    \efig
    \and
    \bfig
    \square|amma|/->`->`<-`->/<1000,500>[
      FA \otimes_2 \top_2`
      FA`
      FA \otimes_2 F\top_1`
      F(A \otimes_1 \top_1);
      {\rho_2}_{FA}`
      \id_{FA} \otimes m_{\top_1}`
      F{\rho_1}_{A}`
      m_{A,\top_1}]
    \efig
    \end{mathpar}
  Need to notice that the composition of monoidal functors is also monoidal,
  subject to the above coherence conditions.

\end{definition}

\begin{definition}
  \label{def:SMCFUN}
  Suppose we are given two symmetric monoidal closed categories\\
  $(\cat{M}_1,\top_1,\otimes_1,\alpha_1,\lambda_1,\rho_1,\beta_1)$ and
  $(\cat{M}_2,\top_2,\otimes_2,\alpha_2,\lambda_2,\rho_2,\beta_2)$.  Then a
  \textbf{symmetric monoidal functor} is a functor $F : \cat{M}_1 \mto
  \cat{M}_2$, a map $m_{\top_1} : \top_2 \mto F\top_1$ and a natural
  transformation $m_{A,B} : FA \otimes_2 FB \mto F(A \otimes_1 B)$ subject
  to the following coherence conditions:
  \begin{mathpar}
    \bfig
    \vSquares|ammmmma|/->`->`->``->`->`->/[
      (FA \otimes_2 FB) \otimes_2 FC`
      FA \otimes_2 (FB \otimes_2 FC)`
      F(A \otimes_1 B) \otimes_2 FC`
      FA \otimes_2 F(B \otimes_1 C)`
      F((A \otimes_1 B) \otimes_1 C)`
      F(A \otimes_1 (B \otimes_1 C));
      {\alpha_2}_{FA,FB,FC}`
      m_{A,B} \otimes \id_{FC}`
      \id_{FA} \otimes m_{B,C}``
      m_{A \otimes_1 B,C}`
      m_{A,B \otimes_1 C}`
      F{\alpha_1}_{A,B,C}]
    \efig
    \end{mathpar}
%    \and
\begin{mathpar}
    \bfig
    \square/->`->`<-`->/<1000,500>[
      \top_2 \otimes_2 FA`
      FA`
      F\top_1 \otimes_2 FA`
      F(\top_1 \otimes_1 A);
      {\lambda_2}_{FA}`
      m_{\top_1} \otimes \id_{FA}`
      F{\lambda_1}_{A}`
      m_{\top_1,A}]
    \efig
    \and
    \bfig
    \square/->`->`<-`->/<1000,500>[
      FA \otimes_2 \top_2`
      FA`
      FA \otimes_2 F\top_1`
      F(A \otimes_1 \top_1);
      {\rho_2}_{FA}`
      \id_{FA} \otimes m_{\top_1}`
      F{\rho_1}_{A}`
      m_{A,\top_1}]
    \efig
     \end{mathpar}
     
      \begin{mathpar}
    \bfig
    \square/->`->`->`->/<1000,500>[
      FA \otimes_2 FB`
      FB \otimes_2 FA`
      F(A \otimes_1 B)`
      F(B \otimes_1 A);
      {\beta_2}_{FA,FB}`
      m_{A,B}`
      m_{B,A}`
      F{\beta_1}_{A,B}]
    \efig
  \end{mathpar}
\end{definition}

\begin{definition}
  \label{def:MCNAT}
  Suppose $(\cat{M}_1,\top_1,\otimes_1)$ and $(\cat{M}_2,\top_2,\otimes_2)$
  are monoidal categories, and $(F,m)$ and $(G,n)$ are monoidal functors
  between $\cat{M}_1$ and $\cat{M}_2$.  Then a \textbf{
    monoidal natural transformation} is a natural transformation,
  $f : F \mto G$, subject to the following coherence diagrams:
  \begin{mathpar}
    \bfig
    \square<1000,500>[
      FA \otimes_2 FB`
      F(A \otimes_1 B)`
      GA \otimes_2 GB`
      G(A \otimes_1 B);
      m_{A,B}`
      f_A \otimes_2 f_B`
      f_{A \otimes_1 B}`
      n_{A,B}]
    \efig
    \and
    \bfig
    \Vtriangle/->`<-`<-/[
      F\top_1`
      G\top_1`
      \top_2;
      f_{\top_1}`
      m_{\top_1}`
      n_{\top_1}]
    \efig
  \end{mathpar}  
\end{definition}

\begin{definition}
  \label{def:SMCNAT}
  Suppose $(\cat{M}_1,\top_1,\otimes_1)$ and $(\cat{M}_2,\top_2,\otimes_2)$
  are SMCs, and $(F,m)$ and $(G,n)$ are symmetric monoidal functors
  between $\cat{M}_1$ and $\cat{M}_2$.  Then a \textbf{symmetric
    monoidal natural transformation} is a natural transformation,
  $f : F \mto G$, subject to the following coherence diagrams:
  \begin{mathpar}
    \bfig
    \square<1000,500>[
      FA \otimes_2 FB`
      F(A \otimes_1 B)`
      GA \otimes_2 GB`
      G(A \otimes_1 B);
      m_{A,B}`
      f_A \otimes_2 f_B`
      f_{A \otimes_1 B}`
      n_{A,B}]
    \efig
    \and
    \bfig
    \Vtriangle/->`<-`<-/[
      F\top_1`
      G\top_1`
      \top_2;
      f_{\top_1}`
      m_{\top_1}`
      n_{\top_1}]
    \efig
  \end{mathpar}  
\end{definition}

\begin{definition}
  \label{def:MCADJ}
  Suppose $(\cat{M}_1,\top_1,\otimes_1)$ and $(\cat{M}_2,\top_2,\otimes_2)$
  are monoidal categories, and $(F,m)$ is a monoidal functor between
  $\cat{M}_1$ and $\cat{M}_2$ and $(G,n)$ is a monoidal
  functor between $\cat{M}_2$ and $\cat{M}_1$.  Then a
  \textbf{monoidal adjunction} is an ordinary adjunction
  $\cat{M}_1 : F \dashv G : \cat{M}_2$ such that the unit,
  $\eta_A : A \to GFA$, and the counit, $\varepsilon_A : FGA \to A$, are
  monoidal natural transformations.  Thus, the following
  diagrams must commute:
  \begin{mathpar}
    \bfig
    \square/->`->`->`<-/<1000,500>[
      FGA \otimes_2 FGB`
      F(GA \otimes_1 GB)`
      A \otimes_2 B`
      FGA \otimes_2 FGB;
      m_{GA,GB}`
      \varepsilon_A \otimes_1 \varepsilon_B`
      Fn_{A,B}`
      \varepsilon_{A \otimes_1 B}]
    \efig
    \and
    \bfig
    %% \Vtriangle|amm|/->`<-`=/[
    %%   FG\top_1`
    %%   \top_1`
    %%   \top_1;
    %%   \varepsilon_{\top_1}`
    %%   \q{\top_1}`]
    \square/->`<-`->`=/<1000,500>[
      F\top_1`
      FG\top_2`
      \top_2`
      \top_2;
      Fn_{\top_2}`
      m_{\top_1}`
      \varepsilon_{\top_1}`]    
    \efig
    \and
    \bfig
    %% \dtriangle|mmb|<1000,500>[
    %%   A \otimes_2 B`
    %%   GFA \otimes_2 GFB`
    %%   GF(A \otimes_2 B);
    %%   \eta_A \otimes_2 \eta_B`
    %%   \eta_{A \otimes_2 B}`
    %%   \p{A,B}]
    \square/<-`->`->`->/<1000,500>[
      GFA \otimes_1 GFB`
      A \otimes_1 B`
      G(FA \otimes_2 FB)`
      GF(A \otimes_1 B);
      \eta_A \otimes_1 \eta_B`
      n_{FA,FB}`
      \eta_{A \otimes_1 B}`
      m_{A,B}]
    \efig
    \and
    \bfig
    %% \Vtriangle|amm|/->`=`<-/[
    %%   \top_1`
    %%   GF\top_1`
    %%   \top_1;
    %%   \eta_{\top_1}``
    %%   p_{\top_1}]
    \square/->`<-`<-`=/<1000,500>[
      G\top_2`
      GF\top_1`
      \top_1`
      \top_1;
      Gm_{\top_1}`
      n_{\top_2}`
      \eta_{\top_1}`]      
    \efig
  \end{mathpar} 
\end{definition}

\begin{definition}
  \label{def:SMCADJ}
  Suppose $(\cat{M}_1,\top_1,\otimes_1)$ and $(\cat{M}_2,\top_2,\otimes_2)$
  are SMCs, and $(F,m)$ is a symmetric monoidal functor between
  $\cat{M}_1$ and $\cat{M}_2$ and $(G,n)$ is a symmetric monoidal
  functor between $\cat{M}_2$ and $\cat{M}_1$.  Then a
  \textbf{symmetric monoidal adjunction} is an ordinary adjunction
  $\cat{M}_1 : F \dashv G : \cat{M}_2$ such that the unit,
  $\eta_A : A \to GFA$, and the counit, $\varepsilon_A : FGA \to A$, are
  symmetric monoidal natural transformations.  Thus, the following
  diagrams must commute:
  \begin{mathpar}
    \bfig
    \square/->`->`->`<-/<1000,500>[
      FGA \otimes_2 FGB`
      F(GA \otimes_1 GB)`
      A \otimes_2 B`
      FGA \otimes_2 FGB;
      m_{GA,GB}`
      \varepsilon_A \otimes_1 \varepsilon_B`
      Fn_{A,B}`
      \varepsilon_{A \otimes_1 B}]
    \efig
    \and
    \bfig
    %% \Vtriangle|amm|/->`<-`=/[
    %%   FG\top_1`
    %%   \top_1`
    %%   \top_1;
    %%   \varepsilon_{\top_1}`
    %%   \q{\top_1}`]
    \square/->`<-`->`=/<1000,500>[
      F\top_1`
      FG\top_2`
      \top_2`
      \top_2;
      Fn_{\top_2}`
      m_{\top_1}`
      \varepsilon_{\top_1}`]    
    \efig
    \and
    \bfig
    %% \dtriangle|mmb|<1000,500>[
    %%   A \otimes_2 B`
    %%   GFA \otimes_2 GFB`
    %%   GF(A \otimes_2 B);
    %%   \eta_A \otimes_2 \eta_B`
    %%   \eta_{A \otimes_2 B}`
    %%   \p{A,B}]
    \square/<-`->`->`->/<1000,500>[
      GFA \otimes_1 GFB`
      A \otimes_1 B`
      G(FA \otimes_2 FB)`
      GF(A \otimes_1 B);
      \eta_A \otimes_1 \eta_B`
      n_{FA,FB}`
      \eta_{A \otimes_1 B}`
      m_{A,B}]
    \efig
    \and
    \bfig
    %% \Vtriangle|amm|/->`=`<-/[
    %%   \top_1`
    %%   GF\top_1`
    %%   \top_1;
    %%   \eta_{\top_1}``
    %%   p_{\top_1}]
    \square/->`<-`<-`=/<1000,500>[
      G\top_2`
      GF\top_1`
      \top_1`
      \top_1;
      Gm_{\top_1}`
      n_{\top_2}`
      \eta_{\top_1}`]      
    \efig
  \end{mathpar} 
\end{definition}

\begin{definition}
  \label{def:monoidal-comonad}
  A \textbf{monoidal comonad} on a monoidal
  category $\cat{C}$ is a triple $(T,\varepsilon, \delta)$, where
  $(T,\m{})$ is a monoidal endofunctor on $\cat{C}$,
  $\varepsilon_A : TA \mto A$ and $\delta_A : TA \to T^2 A$ are
  monoidal natural transformations, which make the following
  diagrams commute:
  \begin{mathpar}
    \bfig
    \square<600,600>[
      TA`
      T^2A`
      T^2A`
      T^3A;
      \delta_A`
      \delta_A`
      T\delta_A`
      \delta_{TA}]
    \efig
    \and
    \bfig
    \Atrianglepair/=`->`=`<-`->/<600,600>[
      TA`
      TA`
      T^2 A`
      TA;`
      \delta_A``
      \varepsilon_{TA}`
      T\varepsilon_A]
    \efig
  \end{mathpar}
  The assumption that $\varepsilon$ and $\delta$ are 
  monoidal natural transformations amount to the following diagrams
  commuting:
  \begin{mathpar}
    \bfig
    \qtriangle/->`->`->/<1000,600>[
      TA \otimes TB`
      T(A \otimes B)`
      A \otimes B;
      \m{A,B}`
      \varepsilon_A \otimes \varepsilon_B`
    \varepsilon_{A \otimes B}]
    \efig
    \and
    \bfig
    \Vtriangle/<-`->`=/<600,600>[
      T\top`
      \top`
      \top;
      \m{\top}`
      \varepsilon_\top`]
    \efig    
  \end{mathpar}
  \begin{mathpar}
    \bfig
    \square|alab|/`->``->/<1050,600>[
      TA \otimes TB``
      T^2A \otimes T^2B`
      T(TA \otimes TB);`
      \delta_A \otimes \delta_B``
      \m{TA,TB}]
    \square(1050,0)|mmrb|/``->`->/<1050,600>[`
      T(A \otimes B)`
      T(TA \otimes TB)`
      T^2(A \otimes B);``
      \delta_{A \otimes B}`
      T\m{A,B}]
    \morphism(0,600)<2100,0>[TA \otimes TB`T(A \otimes B);\m{A,B}]
    \efig
    \and
    \bfig
    \square<600,600>[
      \top`
      T\top`
      T\top`
      T^2\top;
      \m{\top}`
      \m{\top}`
      \delta_\top`
      T\m{\top}]
    \efig
  \end{mathpar}
\end{definition}


\begin{definition}
  \label{def:symm-monoidal-comonad}
  A \textbf{symmetric monoidal comonad} on a symmetric monoidal
  category $\cat{C}$ is a triple $(T,\varepsilon, \delta)$, where
  $(T,\m{})$ is a symmetric monoidal endofunctor on $\cat{C}$,
  $\varepsilon_A : TA \mto A$ and $\delta_A : TA \to T^2 A$ are
  symmetric monoidal natural transformations, which make the following
  diagrams commute:
  \begin{mathpar}
    \bfig
    \square<600,600>[
      TA`
      T^2A`
      T^2A`
      T^3A;
      \delta_A`
      \delta_A`
      T\delta_A`
      \delta_{TA}]
    \efig
    \and
    \bfig
    \Atrianglepair/=`->`=`<-`->/<600,600>[
      TA`
      TA`
      T^2 A`
      TA;`
      \delta_A``
      \varepsilon_{TA}`
      T\varepsilon_A]
    \efig
  \end{mathpar}
  The assumption that $\varepsilon$ and $\delta$ are symmetric
  monoidal natural transformations amount to the following diagrams
  commuting:
  \begin{mathpar}
    \bfig
    \qtriangle/->`->`->/<1000,600>[
      TA \otimes TB`
      T(A \otimes B)`
      A \otimes B;
      \m{A,B}`
      \varepsilon_A \otimes \varepsilon_B`
    \varepsilon_{A \otimes B}]
    \efig
    \and
    \bfig
    \Vtriangle/<-`->`=/<600,600>[
      T\top`
      \top`
      \top;
      \m{\top}`
      \varepsilon_\top`]
    \efig    
  \end{mathpar}
  \begin{mathpar}
    \bfig
    \square|alab|/`->``->/<1050,600>[
      TA \otimes TB``
      T^2A \otimes T^2B`
      T(TA \otimes TB);`
      \delta_A \otimes \delta_B``
      \m{TA,TB}]
    \square(1050,0)|mmrb|/``->`->/<1050,600>[`
      T(A \otimes B)`
      T(TA \otimes TB)`
      T^2(A \otimes B);``
      \delta_{A \otimes B}`
      T\m{A,B}]
    \morphism(0,600)<2100,0>[TA \otimes TB`T(A \otimes B);\m{A,B}]
    \efig
    \and
    \bfig
    \square<600,600>[
      \top`
      T\top`
      T\top`
      T^2\top;
      \m{\top}`
      \m{\top}`
      \delta_\top`
      T\m{\top}]
    \efig
  \end{mathpar}
\end{definition}

\section{Proofs}
\label{sec:proofs}
\subsection{Proof of Composition of Weakening and Contraction (Lemma~\ref{lem:compose-cw})}
\label{subsec:proof_of_composition_of_weakening_and_contraction_lem:compose-cw}
Since by definition $w:\cat{L} \mto \cat{L}$ and $c:\cat{L} \mto
\cat{L}$ are monoidal functors we know that their composition
$cw:\cat{L} \mto \cat{L}$ is a monoidal functor:
\[
\begin{array}{ll}
  \q{A,B} : cwA\otimes cwB\mto cw(A\otimes B)   \\
  \q{A,B} = c\q{A,B}^w\circ\q{wA,wB}^c        \\
  \q{I} : I\mto cwI                             \\
  \q{I} = c\q{I}^w\circ\q{I}^c
\end{array}
\]

We must now define both $\varepsilon_A:cwA\mto A$ and
$\delta_A:cwA\mto cwcwA$, and then show that they are monoidal
natural transformations subject to the comonad laws. Since we are
composing two comonads each of $\varepsilon$ and $\delta$ can be
given two definitions, but they are equivalent:
\begin{itemize}
\item $\varepsilon_A:cwA\mto A$ is defined as in the diagram
  below, which commutes by the naturality of $\varepsilon^c$.
  \begin{mathpar}
    \bfig
    \square(1050,0)/->`->`->`->/<1050,600>[
      cwA`
      wA`
      cA`
      A;
      \varepsilon_{wA}^c`
      c\varepsilon_A^w`
      \varepsilon_A^w`
      \varepsilon_A^c]
    \efig
  \end{mathpar}

\item $\delta_A:cwA\mto cwcwA$ is defined as in the diagram:
  \begin{mathpar}
    \bfig
    \square|almb|/->`->`->`->/<1050,600>[
      cwA`
      cw^2A`
      c^2wA`
      c^2w^2A;
      c\delta_A^w`
      \delta_{wA}^c`
      \delta_{w^2A}^c`
      c^2\delta_A^w]
    \square(1050,0)|amrb|/->``->`->/<1050,600>[
      cw^2A`
      c^2w^2A`
      c^2w^2A`
      cwcwA;
      \delta_{w^2A}^c``
      cdist_{wA}`
      cdist_{wA}]
    \efig
  \end{mathpar}
  The left part of the diagram commutes by the naturality
  of $\delta^c$ and the right part commutes trivially.
\end{itemize}

The remainder of the proof shows that the comonad laws hold.

\begin{itemize}
\item[] \textbf{Case 1:}
  \begin{mathpar}
    \bfig
    \square/->`->`->`->/<1050,600>[
      cwA`
      cwcwA`
      cwcwA`
      cwcwcwA;
      \delta_A`
      \delta_A`
      cw\delta_A`
      \delta_{cwA}]
    \efig
  \end{mathpar}

  The previous diagram commutes because the following one does.

  \begin{mathpar}
    \bfig
    \ptriangle/->`->`=/<700,400>[
      cwA`
      cwcwA`
      cwcwA;
      \delta_A`
      \delta_A`]
    \square(700,0)|amm|/->`->`->`/<900,400>[
      cwcwA`
      cwcw^2A`
      c^2wcwA`
      c^2wcw^2A;
      cwc\delta_A^w`
      \delta_{wcwA}^c`
      \delta_{wcw^2A}^c`]
    \ptriangle(1600,0)|amm|/->``<-/<1100,400>[
      cwcw^2A`
      cwc^2w^2A`
      c^2wcw^2A;
      cw\delta_{w^2A}^c`
      `
      cdist_{cw^2A}]
    \qtriangle(700,-600)|mmm|/->`->`->/<900,600>[
      c^2wcwA`
      c^2wcw^2A`
      c^2w^2cwA;
      c^2wc\delta_A^w`
      c^2\delta_{cwA}^w`
      c^2wdist_{wA}]
    \btriangle(0,-600)/->``->/<1600,600>[
      cwcwA`
      cw^2cwA`
      c^2w^2cwA;
      c\delta_{cwA}^w`
      `
      \delta_{w^2cwA}^c]
    \dtriangle(1600,-600)/`->`->/<1100,1000>[
      cwc^2w^2A`
      c^2w^2cwA`
      cwcwcwA;
      `
      cwcdist_{wA}`
      cdist_{wcwA}]
    % To show texts in each subdiagram:
    \ptriangle(200,-150)/``/<400,400>[(1)``;``]
    \ptriangle(1000,-200)/``/<400,400>[(2)``;``]
    \ptriangle(1300,-600)/``/<400,400>[(3)``;``]
    \ptriangle(500,-700)/``/<400,400>[(4)``;``]
    \ptriangle(1850,-150)/``/<400,400>[(5)``;``]
    \ptriangle(2100,-700)/``/<400,400>[(6)``;``]
    \efig
  \end{mathpar}

  (1) commutes by equality and we will not expand $\delta_A$ for
  simplicity. (2) and (4) commutes by the naturality of $\delta^c$. (3),
  (5) commutes by the conditions of $dist$. (6) commutes by the naturality of
  $dist$.

\item[] \textbf{Case 2}:
  \begin{mathpar}
    \bfig
    \qtriangle/->`=`->/<600,600>[
      cwA`
      cwcwA`
      cwA;
      \delta_A``
      cw\varepsilon_A]
    \efig
  \end{mathpar}

  The triangle commutes because of the following diagram chasing.

  \begin{mathpar}
    \bfig
    \qtriangle|amm|/->`<-`=/<1200,600>[
      cwA`
      cw^2A`
      cw^2A;
      c\delta_A^w`
      c\varepsilon_{wA}^w`]
    \ptriangle(1200,0)|amm|/->``->/<600,600>[
      cw^2A`
      c^2w^2A`
      cw^2A;
      \delta_{w^2A}^c``
      c\varepsilon_{w^2A}^c]
    \btriangle(0,-1200)/=``<-/<1200,1800>[
      cwA`
      cwA`
      wcwA;
      ``\varepsilon_{cwA}^w]
    \btriangle(1200,-1200)|mmb|/`<-`<-/<600,1200>[
      cw^2A`
      wcwA`
      cwcwA;
      `
      cw\varepsilon_{wA}^c`
      \varepsilon_{wcwA}^c]
    \dtriangle(600,-600)|mmm|/`->`<-/<600,600>[
      cw^2A`
      wA`
      w^2A;
      `
      \varepsilon_{w^2A}^c`
      \varepsilon_{wA}^w]
    \morphism(0,600)|m|<600,-1200>[cwA`wA;\varepsilon_{wA}^c]
    \morphism(0,-1200)|m|<600,600>[cwA`wA;\varepsilon_{wA}^c]
    \morphism(1200,-1200)|m|<0,600>[wcwA`w^2A;w\varepsilon_{wA}^c]
    \morphism(1800,600)|r|<0,-1800>[c^2w^2A`cwcwA;cdist_{wA}]
    \ptriangle(800,300)/``/<100,100>[(1)``;``]
    \ptriangle(1350,300)/``/<100,100>[(2)``;``]
    \ptriangle(700,-300)/``/<100,100>[(3)``;``]
    \ptriangle(1600,-300)/``/<100,100>[(4)``;``]
    \ptriangle(300,-700)/``/<100,100>[(5)``;``]
    \ptriangle(700,-1000)/``/<100,100>[(6)``;``]
    \ptriangle(1450,-1000)/``/<100,100>[(7)``;``]
    \efig
  \end{mathpar}
  (1) commutes by the comonad law for $w$ with components $\delta_A^w$
  and $\varepsilon_{wA}^w$. (2) commutes by the comonad law for $c$ with
  components $\delta_{w^2A}^c$ and $\varepsilon_{w^2A}^c$. (3) and (7)
  commute by the naturality of $\varepsilon^c$. (4) commutes by the condition
  of $dist$. (5) commutes trivially. And (6) commutes by the naturality of
  $\varepsilon^w$.
  
\item[] \textbf{Case 3}:
  \begin{mathpar}
    \bfig
    \btriangle/->`=`->/<600,600>[
      cwA`
      cwcwA`
      cwA;
      \delta_A``
      \varepsilon_{cwA}]
    \efig
  \end{mathpar}

  The previous triangle commutes because the following diagram chasing
  does.

  \begin{mathpar}
    \bfig
    \qtriangle|amm|/->`->`/<800,400>[
      cwA`
      cw^2A`
      c^2wA;
      c\delta_A^w`
      \delta_{wA}^c`]
    \morphism(0,400)|m|/<-/<800,-800>[cwA`c^2wA;c\varepsilon_{wA}^c]
    \ptriangle(800,0)|amm|/->``<-/<800,400>[
      cw^2A`
      c^2w^2A`
      c^2wA;
      \delta_{w^2A}^c``
      c^2\delta_A^w]
    \morphism(800,-400)|m|/<-/<800,800>[c^2wA`c^2w^2A;c^2w\varepsilon_A^w]
    \morphism(800,0)/=/<0,-400>[c^2wA`c^2wA;]
    \btriangle(0,-800)/=``<-/<800,1200>[
      cwA`
      cwA`
      cwcA;
      ``
      cw\varepsilon_A^c]
    \dtriangle(800,-800)/`->`<-/<800,1200>[
      c^2w^2A`
      cwcA`
      cwcwA;
      `
      cdist_{wA}`
      cwc\varepsilon_A^w]
    \morphism(800,-400)|m|/->/<0,-400>[c^2wA`cwcA;cdist_A]
    \ptriangle(800,100)/``/<100,100>[(1)``;``]
    \ptriangle(600,-100)/``/<100,100>[(2)``;``]
    \ptriangle(1000,-100)/``/<100,100>[(3)``;``]
    \ptriangle(400,-600)/``/<100,100>[(4)``;``]
    \ptriangle(1200,-600)/``/<100,100>[(5)``;``]
    \efig
  \end{mathpar}

  (1) commutes by the naturality of $\delta^c$. (2) is the comonad law
  for $c$ with components $\delta_{wA}^c$ and $\varepsilon_{wA}^c$. (3)
  is the comonad law for $w$ with components $\delta_A^w$ and
  $\varepsilon_A^w$. (4) commutes by the condition of $dist$. And (5)
  commute by the naturality of $dist$.

\end{itemize}
% subsection proof_of_composition_of_weakening_and_contraction_(lemma~\ref{lem:compose-cw}) (end)
% section proofs (end)



\subsection{Proof of Conditions of Lambek category with $cw$ (Lemma~\ref{lem:compose-cw-2})}
\label{subsec:proof_of_conditions_of_lambek_with_cw_lem:compose-cw-2}
  \begin{itemize}
  \item[1.] As shown in the paper.
  % Conditions for each cwA being a comonoid
  \item[2.] Each $(cwA,\w{A},\c{A})$ is a comonoid.
    \begin{itemize}
    \item[] \textbf{Case 1:}
      \begin{mathpar}
      \bfig
      \square/->`->``/<1050,400>[
        cwA`
        cwA\otimes cwA`
        cwA\otimes cwA`;
        \c{A}`
        \c{A}``]
      \square(1050,0)/->``<-`/<1150,400>[
        cwA\otimes cwA`
        cwA\otimes(cwA\otimes cwA)``
        (cwA\otimes cwA)\otimes cwA;
        id_{cwA}\otimes\c{A}``
        \alpha_{cwA,cwA,cwA}`]
        \morphism(0,0)|b|<2200,0>[
          cwA\otimes cwA`
          (cwA\otimes cwA)\otimes cwA;
          \c{A}\otimes id_{cwA}]
      \efig
      \end{mathpar}

      The previous diagram commutes by the following diagram chasing.

      \begin{mathpar}
      \bfig
        \ptriangle/->`->`=/<1100,400>[
          cwA`
          cwA\otimes cwA`
          cwA\otimes cwA;
          \c{A}`
          \c{A}`]
        \qtriangle(1100,0)/->``->/<1700,400>[
          cwA\otimes cwA`
          cwA\otimes(cwA\otimes I)`
          cwA\otimes((cwA\otimes I)\otimes cwA);
          id_{cwA}\otimes\rho_{cwA}^{-1}``
          id_{cwA}\otimes\cL{wA,I}]
        \square(0,-400)|mlmm|/->`->`->`/<1100,400>[
          cwA\otimes cwA`
          cwA\otimes(I\otimes cwA)`
          (cwA\otimes I)\otimes cwA`
          cwA\otimes(cwA\otimes(I\otimes cwA));
          id_{cwA}\otimes\lambda_{cwA}^{-1}`
          \rho_{cwA}^{-1}\otimes id_{cwA}`
          id_{cwA}\otimes\cR{wA,I}`]
        \dtriangle(1100,-400)|mrm|/<-`->`->/<1700,400>[
          cwA\otimes((cwA\otimes I)\otimes cwA)`
          cwA\otimes(cwA\otimes(I\otimes cwA))`
          cwA\otimes(cwA\otimes cwA);
          id_{cwA}\otimes\alpha_{cwA,I,cwA}`
          id_{cwA}\otimes(\rho_{cwA}\otimes id_{cwA})`
          id_{cwA}\otimes(id_{cwA}\otimes\lambda_{cwA})]
        \square(0,-800)/`->`<-`->/<2800,400>[
          (cwA\otimes I)\otimes cwA`
          cwA\otimes(cwA\otimes cwA)`
          ((cwA\otimes I)\otimes cwA)\otimes cwA`
          (cwA\otimes cwA)\otimes cwA;
          `
          \cL{wA,I}\otimes id_{cwA}`
          \alpha_{cwA,cwA,cwA}`
          (\rho_{cwA}\otimes id_{cwA})\otimes id_{cwA}]
        \ptriangle(300,150)/``/<100,100>[(1)``;``]
        \ptriangle(1950,0)/``/<100,100>[(2)``;``]
        \ptriangle(2400,-350)/``/<100,100>[(3)``;``]
        \ptriangle(1400,-750)/``/<100,100>[(4)``;``]
      \efig
      \end{mathpar}

      (1) commutes trivially and we would not expand $\c{}$ for
      simplicity. (2) and (4) commute because $(\cat{L},c,\cL{},\cR{})$
      is a Lambek category with contraction. (3) commutes because
      $\cat{L}$ is monoidal.

    \item[] \textbf{Case 2:}
      \begin{mathpar}
      \bfig
      \Atrianglepair/->`->`->`<-`->/<1000,400>[
        cwA`
        I\otimes cwA`
        cwA\otimes cwA`
        cwA\otimes I;
        \lambda^{-1}`
        \c{A}`
        \rho^{-1}`
        \w{A}\otimes id_{cwA}`
        id_{cwA}\otimes\w{A}]
      \efig
      \end{mathpar}
      The diagram above commutes by the following diagram chasing.
      \begin{mathpar}
      \bfig
        \square/<-`<-`<-`/<2200,1200>[
          I\otimes cwA`
          wA\otimes cwA`
          cwA`
          cwA\otimes cwA;
          \w{A}^w\otimes id_{cwA}`
          \lambda_{cwA}^{-1}`
          \varepsilon_{wA}^c\otimes cwA`]
        \square(0,-1200)/`->`->`<-/<2200,1200>[
          cwA`
          cwA\otimes cwA`
          cwA\otimes I`
          cwA\otimes wA;
          `
          \rho_{cwA}^{-1}`
          id_{cwA}\otimes\varepsilon_{wA}^c`
          id_{cwA}\otimes\w{A}]
        \Ctriangle(0,-400)|mmm|/<-``->/<500,400>[
          I\otimes cwA`
          cwA`
          cwA\otimes I;
          \lambda_{cwA}^{-1}``
          \rho_{cwA}^{-1}]
        \morphism(500,800)|m|<-500,400>[
          I\otimes(I\otimes cwA)`
          I\otimes cwA;
          id_I\otimes\lambda_{cwA}]
        \square(500,400)|mmmm|/<-`<-`<-`->/<1200,400>[
          I\otimes(I\otimes cwA)`
          wA\otimes(I\otimes cwA)`
          I\otimes cwA`
          cwA\otimes(I\otimes cwA);
          \w{A}^w\otimes id_{I\otimes cwA}`
          \lambda_{I\otimes cwA}^{-1}`
          \varepsilon_{wA}^c\otimes id_{I\otimes cwA}`
          \cR{wA,I}]
        \morphism(1700,800)|m|<500,400>[
          wA\otimes(I\otimes cwA)`
          wA\otimes cwA;
          id_{wA}\otimes\lambda_{cwA}]
        \Dtriangle(1700,-400)|mmm|/`->`<-/<500,400>[
          cwA\otimes(I\otimes cwA)`
          cwA\otimes cwA`
          (cwA\otimes I)\otimes cwA;
          `
          id_{cwA}\otimes\lambda_{cwA}`
          \rho_{cwA}\otimes id_{cwA}]
        \square(500,-800)|mmmm|/->`->`->`<-/<1200,400>[
          cwA\otimes I`
          (cwA\otimes I)\otimes cwA`
          (cwA\otimes I)\otimes I`
          (cwA\otimes I)\otimes wA;
          \cL{wA,I}`
          \rho_{cwA}^{-1}`
          id_{cwA\otimes I}\otimes\varepsilon_{wA}^c`
          id_{cwA\otimes I}\otimes\w{A}^w]
        \morphism(500,-800)|m|<-500,-400>[
          (cwA\otimes I)\otimes I`
          cwA\otimes I;
          \rho_{cwA}\otimes id_I]
        \morphism(1700,-800)|m|<500,-400>[
          (cwA\otimes I)\otimes wA`
          cwA\otimes wA;
          \rho_{cwA}\otimes id_{wA}]
        \ptriangle(200,500)/``/<100,100>[(1)``;``]
        \ptriangle(1100,900)/``/<100,100>[(2)``;``]
        \ptriangle(2000,500)/``/<100,100>[(3)``;``]
        \ptriangle(1100,500)/``/<100,100>[(4)``;``]
        \ptriangle(1100,-100)/``/<100,100>[(5)``;``]
        \ptriangle(200,-700)/``/<100,100>[(6)``;``]
        \ptriangle(1100,-1100)/``/<100,100>[(7)``;``]
        \ptriangle(2000,-700)/``/<100,100>[(8)``;``]
        \ptriangle(1100,-700)/``/<100,100>[(9)``;``]
      \efig
      \end{mathpar}
    (1), (2) and (3) commute by the functionality of $\lambda$. (6), (7)
    and (8) commute by the functionality of $\rho$. (4) and (9) are
    conditions of the Lambek category with $cw$. And (5) is the
    definition of $\c{}$.

    \end{itemize}
    
    % Condition 3
    \item[3.] $\w{}$ and $\c{}$ are coalgebra morphisms.
      \begin{itemize}
      \item[] \textbf{Case 1:}
        \begin{mathpar}
        \bfig
          \square/->`->`->`->/<1000,400>[
          cwA`
          I`
          cwcwA`
          cwI;
          \w{A}`
          \delta{A}`
          \q{I}`
          cw\w{A}]
        \efig
        \end{mathpar}

        The previous diagram commutes by the diagram below. (1) commutes by
        the naturality of $\delta^c$. (2) commutes by the condition of
        $dist_{wA}$. (3), (5) and (6) commute because $c$ is a monoidal
        comonad. (4) commutes because $(\cat{L},w,\w{}^w)$ is a Lambek
        category with weakening. (7) commutes because $c$ and $w$ are
        monoidal comonads.
        \begin{mathpar}
        \bfig
          \square/->`->``/<800,400>[
            cwA`
            cI`
            c^2wA`;
            c\w{A}^w`
            \delta_{wA}^c``]
          \morphism<0,-400>[c^2wA`c^2w^2A;c^2\delta_A^w]
          \morphism(0,400)|m|<800,-800>[cwA`cw^2A;c\delta_A^w]
          \square(0,-800)|almb|/<-`->`=`->/<800,400>[
            c^2w^2A`
            cw^2A`
            cwcwA`
            cw^2A;
            \delta_{w^2A}^c`
            cdist_{wA}``
            cw\varepsilon_{wA}^c]
          \morphism(0,-400)|m|<800,-400>[
            c^2w^2A`cw^2A;c\varepsilon_{w^A}^c]
          \Vtriangle(800,0)|amm|/->`<-`=/<400,400>[
            cI`
            I`
            I;
            \epsilon_I^c`
            \q{I}^c`]
          \Ctriangle(1200,-400)|arm|/`->`<-/<400,400>[
            I`I`cI;
            `
            \q{I}^c`
            \varepsilon_I^c]
          \morphism(1600,-400)|r|<0,-400>[cI`cwI;c\q{I}^w]
          \btriangle(800,-800)|amb|/`->`<-/<800,1200>[
            cI`
            cw^2A`
            cwI;
            `
            c\q{I}^w`
            cw\w{A}^w]
          \ptriangle(300,-250)/``/<100,100>[(1)``;``]
          \ptriangle(150,-750)/``/<100,100>[(2)``;``]
          \ptriangle(500,-600)/``/<100,100>[(3)``;``]
          \ptriangle(800,-200)/``/<100,100>[(4)``;``]
          \ptriangle(1200,100)/``/<100,100>[(5)``;``]
          \ptriangle(1450,-100)/``/<100,100>[(6)``;``]
          \ptriangle(1450,-500)/``/<100,100>[(7)``;``]
        \efig
        \end{mathpar}
      
      \item[] \textbf{Case 2:}
        \begin{mathpar}
        \bfig
          \square/->`->``/<1050,400>[
            cwA`
            cwA\otimes cwA`
            cwcwA`;
            \c{A}`
            \delta_A``]
          \square(1050,0)|ammb|/->``->`/<1050,400>[
            cwA\otimes cwA`
            cwcwA\otimes cwcwA``
            cw(cwA\otimes cwA);
            \delta_A\otimes\delta_A``
            \q{cwA,cwA}`]
            \morphism(0,0)|b|<2100,0>[cwcwA`cw(cwA\otimes cwA);cw\c{A}]
        \efig
        \end{mathpar}

        To prove the previous diagram commute, we first expand it, Then we
        divide it into five parts as shown belovee, and prove each part commutes.

        \begin{mathpar}
        \bfig
          \square|almm|/->`->`<-`/<600,400>[
            cwA`
            cwA\otimes I`
            cw^2A`
            w(cwA\otimes I);
            \rho_{cwA}^{-1}`
            c\delta_A^w`
            \varepsilon_{cwA\otimes I}^w`]
          \qtriangle(600,0)/->``->/<2000,400>[
            cwA\otimes I`
            (cwA\otimes I)\otimes cwA`
            cwA\otimes cwA;
            \cL{wA,I}``
            \rho_{cwA}\otimes id_{cwA}]
          \morphism<0,-400>[cw^2A`c^2w^2A;\delta_{w^2A}^c]
          \morphism(0,-400)<0,-400>[c^2w^2A`cwcwA;cdist_{wA}]
          \morphism(0,-800)<0,-400>[cwcwA`cw(cwA\otimes I);cw\rho_{cwA}]
          \btriangle(0,-1600)/->``->/<1300,400>[
            cw(cwA\otimes I)`
            cw((cwA\otimes I)\otimes cwA)`
            cw(cwA\otimes cwA);
            cw\cL{wA,I}``
            cw(\rho_{cwA}\otimes id_{cwA})]
          \morphism(0,-1200)|m|<600,1200>[
            cw(cwA\otimes I)`w(cwA\otimes I);
            \varepsilon_{w(cwA\otimes I)}^c]
          \square(1300,-1600)|mmmb|/->`<-``<-/<1300,1600>[
            w(cwA\otimes cwA)`
            cwA\otimes cwA`
            cw(cwA\otimes cwA)`
            c(wcwA\otimes wcwA);
            \varepsilon_{cwA\otimes cwA}^w`
            c\varepsilon_{cwA\otimes cwA}^w`
            `
            c\q{cwA\otimes cwA}^w]
          \Ctrianglepair(2600,-800)|mrmmr|/<-`->`=`<-`->/<900,400>[
            cwA\otimes cwA`
            cw^2A\otimes cw^2A`
            cw^2A\otimes cw^2A`
            c^2w^2A\otimes c^2w^2A;
            c\varepsilon_{wA}^c\otimes c\varepsilon_{wA}^c`
            c\delta_A^w\otimes c\delta_A^w``
            \varepsilon_{cw^2A}^c\otimes\varepsilon_{cw^2A}^c`
            \delta_{w^2A}^c\otimes\delta_{w^2A}^c]
          \morphism(2600,-800)|m|<0,-400>[
            c^2w^2A\otimes c^2w^2A`
            cwcwA\otimes cwcwA;
            cdist_{wA}\otimes cdist_{wA}]
          \morphism(2600,-1200)|m|<0,-400>[
            cwcwA\otimes cwcwA`
            c(wcwA\otimes wcwA);
            \q{wcwA\otimes wcwA}^c]
          \ptriangle(2400,-350)/``/<100,100>[(a)``;``]
          \ptriangle(2400,-650)/``/<100,100>[(b)``;``]
          \ptriangle(300,100)/``/<100,100>[(c)``;``]
          \ptriangle(650,-1000)/``/<100,100>[(d)``;``]
          \ptriangle(1800,-1100)/``/<100,100>[(e)``;``]
        \efig
        \end{mathpar}

        Part (a) and (b) are comonad laws.
        
        Part (c) commutes by the following diagram chase. (1) is equality.
        (2) is the comonad law for $w$. (3) is the comonad law for $c$.
        (4) commutes by the naturality of $\varepsilon^c$. (5) is one of
        the conditions for $dist_{wA}$. (6) commutes by the naturality of
        $\varepsilon^w$. And (7) commutes by the naturality of
        $\varepsilon^c$.
        \begin{mathpar}
        \bfig
          \btriangle|lmm|/->`->`=/<600,600>[
            cwA`
            cw^2A`
            cw^2A;
            c\delta_A^w`
            c\delta_A^w`]
          \morphism(0,600)|m|/=/<1200,-600>[cwA`cwA;]
          \ptriangle(0,-600)|mlm|/`->`<-/<600,600>[
            cw^2A`
            cw^2A`
            c^2w^2A;
            `
            \delta_{w^2A}^c`
            \varepsilon_{cw^2A}^c]
          \ptriangle(600,-600)|mmm|/->`->`<-/<600,600>[
            cw^2A`
            cwA`
            wcwA;
            c\varepsilon_{wA}^w`
            dist_{wA}`
            \varepsilon_{cwA}^w]
          \ptriangle(0,-1000)|mlm|/`->`<-/<600,400>[
            c^2w^2A`
            wcwA`
            cwcwA;
            `
            cdist_{wA}`
            \varepsilon_{wcwA}^c]
          \square(0,-600)/->``<-`/<1800,1200>[
            cwA`
            cwA\otimes I``
            w(cwA\otimes I);
            \rho_{cwA}^{-1}``
            \varepsilon_{cwA\otimes I}^w`]
          \square(600,-1000)|mmrm|/->``<-`/<1200,400>[
            wcwA`
            w(cwA\otimes I)``
            cw(cwA\otimes I);
            w\rho_{cwA}^{-1}``
            \varepsilon_{w(cwA\otimes I)}^c`]
          \morphism(0,-1000)|b|<1800,0>[
            cwcwA`
            cw(cwA\otimes I);
            cw\rho_{cwA}^{-1}]
          \ptriangle(200,100)/``/<100,100>[(1)``;``]
          \ptriangle(650,50)/``/<100,100>[(2)``;``]
          \ptriangle(200,-300)/``/<100,100>[(3)``;``]
          \ptriangle(350,-650)/``/<100,100>[(4)``;``]
          \ptriangle(800,-300)/``/<100,100>[(5)``;``]
          \ptriangle(1500,-100)/``/<100,100>[(6)``;``]
          \ptriangle(1100,-900)/``/<100,100>[(7)``;``]
        \efig
        \end{mathpar}

        Part (d) commutes by the following diagram chase. The upper two
        squares both commute by the naturality of $\varepsilon^w$, and the
        lower two squares commute by the naturality of $\varepsilon^c$.
        \begin{mathpar}
        \bfig
          \square|almm|/->`<-`<-`->/<1200,400>[
            cwA\otimes I`
            (cwA\otimes I)\otimes cwA`
            w(cwA\otimes I)`
            w((cwA\otimes I)\otimes cwA);
            \cL{wA,I}`
            \varepsilon_{cwA\otimes I}^w`
            \varepsilon_{(cwA\otimes I)\otimes cwA}^w`
            w\cL{wA,I}]
          \square(0,-400)|mlmb|/`<-`<-`->/<1200,400>[
            w(cwA\otimes I)`
            w((cwA\otimes I)\otimes cwA)`
            cw(cwA\otimes I)`
            cw((cwA\otimes I)\otimes cwA);
            `
            \varepsilon_{w(cwA\otimes I)}^c`
            \varepsilon_{w((cwA\otimes I)\otimes A)}^c`
            cw\cL{wA,I}]
          \square(1200,0)|ammm|/->``<-`->/<1350,400>[
            (cwA\otimes I)\otimes cwA`
            cwA\otimes cwA`
            w((cwA\otimes I)\otimes cwA)`
            w(cwA\otimes cwA);
            \rho_{cwA}\otimes id_{cwA}``
            \varepsilon_{cwA\otimes cwA}^w`
            w(\rho_{cwA}\otimes id_{cwA})]
          \square(1200,-400)|mmrb|/``<-`->/<1350,400>[
            w((cwA\otimes I)\otimes cwA)`
            w(cwA\otimes cwA)`
            cw((cwA\otimes I)\otimes cwA)`
            cw(cwA\otimes cwA);
            ``
            \varepsilon_{w(cwA\otimes cwA)}^c`
            cw(\rho_{cwA}\otimes id_{cwA})]
        \efig
        \end{mathpar}

        Part (e) commutes by the following diagram. (1) commutes by the
        condition of $dist_{wA}$. (2) and (4) commute by the naturality of
        $\varepsilon^c$. (3) and (5) commute because $w$ and $c$ are
        monoidal comonads.
        \begin{mathpar}
        \bfig
          \qtriangle|amm|/<-`<-`->/<1200,400>[
            cwA\otimes cwA`
            cw^2A\otimes cw^2A`
            wcwA\otimes wcwA;
            c\varepsilon_{wA}^w\otimes c\varepsilon_{wA}^w`
            \varepsilon_{cwA}^w\otimes\varepsilon_{cwA}^w`
            dist_{wA}\otimes dist_{wA}]
          \square(1200,0)|amrm|/<-``->`<-/<1200,400>[
            cw^2A\otimes cw^2A`
            c^2w^2A\otimes c^2w^2A`
            wcwA\otimes wcwA`
            cwcwA\otimes cwcwA;
            \varepsilon_{cw^2A}^c\otimes\varepsilon_{cw^2A}^c``
            cdist_{wA}\otimes cdist_{wA}`
            \varepsilon_{wcwA}^c\otimes\varepsilon_{wcwA}^c]
          \morphism(0,-400)|l|<0,800>[
            w(cwA\otimes cwA)`
            cwA\otimes cwA;
            \varepsilon_{cwA\otimes cwA}^w]
          \Atriangle(0,-400)|mmm|/->`<-`/<1200,400>[
            wcwA\otimes wcwA`
            w(cwA\otimes cwA)`
            c(wcwA\otimes wcwA);
            \q{cwA,cwA}^w`
            \varepsilon_{wcwA\otimes wcwA}^c`]
          \dtriangle(1200,-400)/`->`<-/<1200,400>[
            cwcwA\otimes cwcwA`
            cw(cwA\otimes cwA)`
            c(wcwA\otimes wcwA);
            `
            \q{wcwA\otimes wcwA}^c`
            c\q{cwA\otimes cwA}]
          \morphism(0,-400)|b|/<-/<1200,0>[
            w(cwA\otimes cwA)`
            cw(cwA\otimes cwA);
            \varepsilon_{w(cwA\otimes cwA)}^c]
          \ptriangle(900,150)/``/<100,100>[(1)``;``]
          \ptriangle(1800,100)/``/<100,100>[(2)``;``]
          \ptriangle(500,-100)/``/<100,100>[(3)``;``]
          \ptriangle(1200,-300)/``/<100,100>[(4)``;``]
          \ptriangle(2150,-250)/``/<100,100>[(5)``;``]
        \efig
        \end{mathpar}
      \end{itemize}
    
    % Condition 4
    \item[4.] Any coalgebra morphism $f:(cwA,\delta_A)\mto (cwB,\delta_B)$
      between free coalgebras preserves the comonoid structure given
      by $\w{}$ and $\c{}$.

      \begin{itemize}
      \item[] \textbf{Case 1:}
        This coherence diagram is given in the definition of the Lambek
        category with $cw$.
        \begin{mathpar}
        \bfig
          \Vtriangle/->`->`->/<500,400>[
            cwA`
            cwB`
            I;
            f`
            \w{A}`
            \w{B}]
        \efig
        \end{mathpar}

      \item[] \textbf{Case 2:}
        \begin{mathpar}
        \bfig
          \square/->`->`->`->/<800,400>[
          cwA`
          cwA\otimes cwA`
          cwB`
          cwB\otimes cwB;
          \c{A}`
          f`
          f\otimes f`
          \c{B}]
        \efig
        \end{mathpar}

        The square commutes by the diagram chasing below, which commutes by
        the naturality of $\rho$ and $\cL{}$.

        \begin{mathpar}
        \bfig
          \square|almb|/->`->`->`->/<600,500>[
            cwA`
            cwA\otimes I`
            cwB`
            cwB\otimes I;
            \rho_{cwA}^{-1}`
            cwf`
            cwf\otimes id_I`
            \rho_{cwB}^{-1}]
          \square(600,0)|ammb|/->``->`->/<1000,500>[
            cwA\otimes I`
            (cwA\otimes I)\otimes cwA`
            cwB\otimes I`
            (cwB\otimes I)\otimes cwB;
            \cL{wA,I}``
            (cwf\otimes id_I)\otimes cwf`
            \cL{wB,I}]
          \square(1600,0)|amrb|/->``->`->/<1000,500>[
            (cwA\otimes I)\otimes cwA`
            cwA\otimes cwA`
            (cwB\otimes I)\otimes cwB`
            cwB\otimes cwB;
            \rho_{cwA}\otimes id_{cwA}``
            cwf\otimes cwf`
            \rho_{cwB}\otimes id_{cwB}]
        \efig
        \end{mathpar}
      \end{itemize}
  \end{itemize}
% subsection proof_of_conditions_of_lambek_with_cw_(lemma~\ref{lem:compose-cw-2}) (end)
% section proofs (end)



























% section appendix (end)

\end{document}

%%% Local Variables: 
%%% mode: latex
%%% TeX-master: t
%%% End: 

