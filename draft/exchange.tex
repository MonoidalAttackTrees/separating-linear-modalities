\begin{definition}
  \label{def:exchange}
  A \textbf{Lambek category with exchange}, $(\cat{L},e,\e{})$, is a
  Lambek category equipped with a monoidal comonad
  $(e,\varepsilon,\delta)$ on $\cat{L}$, and a monoidal natural
  transformation $\e{A,B}:eA \otimes eB \mto eB \otimes eA$.  We
  require $\e{}$ to be a coalgebra morphism, and the following
  diagrams must commute:
  \[
  \begin{array}{cccc}
    \bfig
    \qtriangle|amm|/->`=`->/<1000,500>[
      eA \otimes eB`
      eB \otimes eA`
      eA \otimes eB;
      \e{A,B}``
      \e{B,A}]
    \efig
    &
    \quad
    &
    \bfig
    \square|amma|<1000,500>[
      e^2A \otimes e^2B`
      e^2B \otimes e^2A`
      e(eA \otimes eB)`
      e(eB \otimes eA);
      \e{eA,eB}`
      \q{eA,eB}`
      \q{eB,eA}`
      e\e{A,B}]
    \efig          
  \end{array}
  \]
  \[
  \begin{array}{cccccccc}
    \bfig
    \square|amma|/->`->``/<1100,400>[
      (eA \otimes eB) \otimes eC`
      eA \otimes (eB \otimes eC)`
      (eB \otimes eA) \otimes eC`;
      \alpha_{eA,eB,eC}`
      \e{A,B} \otimes \id_{eC}``]
    
    \morphism(0,0)|m|/->/<0,-800>[
      (eB \otimes eA) \otimes eC`
      eB \otimes (eA \otimes eC);
      \alpha_{eB,eA,eC}]

    \morphism(0,-800)|a|/->/<1100,0>[
      eB \otimes (eA \otimes eC)`
      eB \otimes (eC \otimes eA);
      \id_{eB} \otimes \e{A,C}]
    
    \morphism(1100,400)|a|/->/<1150,0>[
      eA \otimes (eB \otimes eC)`
      eA \otimes (e^2B \otimes e^2C);
      \id_{eA} \otimes (\delta_{B} \otimes \delta_C)]
    \morphism(2250,400)|m|/->/<0,-400>[
      eA \otimes (e^2B \otimes e^2C)`
      eA \otimes e(eB \otimes eC);
      \id_{eA} \otimes \q{eB,eC}]
    \morphism(2250,0)|m|/->/<0,-400>[
      eA \otimes e(eB \otimes eC)`
      e(eB \otimes eC) \otimes eA;
      \e{eA,eB \otimes eC}]
    \morphism(2250,-400)|m|/->/<0,-400>[
      e(eB \otimes eC) \otimes eA`
      (eB \otimes eC) \otimes eA;
      \varepsilon_{eB \otimes eC} \otimes \id_{eA}]
    \morphism(2250,-800)|a|/->/<-1150,0>[
      (eB \otimes eC) \otimes eA`
      eB \otimes (eC \otimes eA);
      \alpha_{eB,eC,eA}]
    \efig
  \end{array}
  \]
  Furthermore, for any coalgebra morphisms $f : (eA, \delta) \mto
  (eB,\delta)$ and $g : (eC,\delta) \mto (eD,\delta)$ between free
  coalgebras the following diagram must commute:
  \[
  \bfig
  \square(0,-500)|amma|<800,500>[
    eA \otimes eC`
    eB \otimes eD`
    eC \otimes eA`
    eD \otimes eB;
    f \otimes g`
    \e{A,C}`
    \e{B,D}`
    g \otimes f]      
  \efig
  \]
  The morphism $\q{A,B} : eA \otimes eB \mto e(A \otimes B)$ makes
  $(e,\q{})$ a monoidal functor.
\end{definition}
The first diagram in the previous definition makes $\e{}$ an
involution, and the second and third diagrams are required in the
proof that the Eilenberg-Moore category is symmetric; see the proofs of
Lemma~\ref{lemma:beta-coalgebra-morph} and
Lemma~\ref{lemma:the_eilenberg-moore_category_is_symmetric_monoidal}.

\begin{definition}
  \label{def:eilenberg-moore-cat}
  Suppose $(\cat{L},e,\e{})$ is a Lambek category with exchange.  Then
  the \textbf{Eilenberg Moore category}, $\cat{L}^e$, of the comonad
  $(e, \varepsilon, \delta)$ has as objects all the e-coalgebras $(A,
  h_A : A \mto eA)$, and as morphisms all the coalgebra morphisms.  We
  call $h_A$ the action of the coalgebra.  Furthermore, the following
  (action) diagrams must commute:
  \begin{mathpar}
    \bfig
    \square[A`eA`eA`e^2A;h_A`h_A`eh_A`\delta_A]    
    \efig
    \and
    \bfig
    \btriangle/->`=`->/[A`eA`A;h_A``\varepsilon_A]
    \efig
  \end{mathpar}
\end{definition}

\begin{lemma}[The Eilenberg Moore Category is Monoidal]
  \label{lemma:the_eilenberg_moore_category_is_monoidal}
  Then the category $\cat{L}^e$ is monoidal.
\end{lemma}
\begin{proof}
  We must first define the unitors, and then the associator.  Then we
  show that they respect the symmetry monoidal coherence diagrams.
  Throughout this proof we will make use of the coalgebra $(A,h_A)$,
  $(B,h_B)$, and $(C,h_C)$.

  The tensor product of $(A, h_A)$ and $(B, h_b)$ is $(A \otimes
  B,q_{A,B} \circ (h_A \otimes h_B))$, and the unit of the tensor
  product is $(I, q_I)$; both actions are easily shown to satisfies
  the action diagrams of the Eilenberg Moore category. The left and
  right unitors are $\lambda : I \otimes A \mto A$ and $\rho : A
  \otimes I \mto A$, because they are indeed coalgebra morphisms.

  \ \\
  \noindent
  The respective diagram for the right unitor is as follows:
  \begin{center}
    \begin{math}
      \begin{array}{lll}
        \bfig
        \square|amma|/->`->``/<650,500>[A \otimes I`eA \otimes I`A`;h_A \otimes \id{}`\rho``]
        \morphism(650,500)<650,0>[eA \otimes I`eA \otimes eI;\id{} \otimes \q{I}]
        \square(1300,0)|amma|/->``->`/<650,500>[eA \otimes eI`e(A \otimes I)``eA;\q{A,I}``e\rho`]

        \morphism(650,500)<1300,-500>[eA \otimes I`eA;\rho]
        \morphism<1950,0>[A`eA;h_A]
        \efig
      \end{array}
    \end{math}
  \end{center}
  The left diagram commutes by naturality of $\rho$, the right diagram
  commutes by the fact that $e$ is a monoidal functor.  Showing the
  left unitor is a coalgebra morphism is similar.

  The unitors are natural and isomorphisms, because they are
  essentially inherited from the underlying Lambek category.

  The associator $\alpha : (A \otimes B) \otimes C \mto A \otimes (B
  \otimes C)$ is also a coalgebra morphism.  First, notice that:
  \[\q{A \otimes B,C} \circ ((\q{A,B} \circ (h_A \otimes h_B)) \otimes h_c) = \q{A \otimes B,C} \circ (\q{A,B} \otimes \id{}) \circ ((h_A \otimes h_B) \otimes h_C)\]
  where the left-hand side is the action of the coalgebra $(A \otimes B)
  \otimes C$. Similarly, the following is the action of the coalgebra
  $A \otimes (B \otimes C)$:
  \[
  \q{A,B \otimes C} \circ (h_A \otimes (\q{B,C} \circ (h_B \otimes h_C))) = \q{A,B \otimes C} \circ (\id{} \otimes \q{B,C}) \circ (h_A \otimes (h_B \otimes h_C))
  \]
  The following diagram must commute:
  \begin{center}
    \rotatebox{90}{$    
      \bfig
      \square|amma|<1200,500>[
        (A \otimes B) \otimes C`
        (eA \otimes eB) \otimes eC`
        A \otimes (B \otimes C)`
        eA \otimes (eB \otimes eC);
        (h_A \otimes h_B) \otimes h_C`
        \alpha`
        \alpha`
        h_A \otimes (h_B \otimes h_C)]

      \square(1200,0)|amma|/->`->``->/<1200,500>[
        (eA \otimes eB) \otimes eC`
        e(A \otimes B) \otimes eC`
        eA \otimes (eB \otimes eC)`
        eA \otimes e(B \otimes C);
        \q{} \otimes \id{}`
        \alpha``
        \id{} \otimes \q{}]

      \square(2400,0)|amma|/->``->`->/<1200,500>[
        e(A \otimes B) \otimes eC`
        e((A \otimes B) \otimes C)`
        eA \otimes e(B \otimes C)`
        e(A \otimes (B \otimes C));
        \q{}``
        e\alpha`
       \q{}]
      \efig
      $}
  \end{center}
  The left diagram commutes by naturality of $\alpha$, and the right
  diagram commutes because $e$ is a monoidal functor.

  Composition in $\cat{L}^e$ is the same as $\cat{L}$, and thus, the
  monoidal coherence diagrams hold in $\cat{L}^e$ as well.  Thus,
  $\cat{L}^e$ is monoidal.  We now show that it is symmetric.    
\end{proof}

\begin{lemma}
  \label{lemma:pseudo-braided}
  In $\cat{L}^e$ there is a natural transformation $\beta_{A,B} : A \otimes B \mto B \otimes A$.
\end{lemma}
\begin{proof}
  We define $\beta$ as follows:
  \[
  \beta_{A,B} := A \otimes B \mto^{h_A \otimes h_B} eA \otimes eB \mto^{ex_{A,B}} eB \otimes eA \mto^{\varepsilon_B \otimes \varepsilon_A} B \otimes A
  \]
  Suppose $f : A \mto A'$ and $g : B \mto B'$ are two coalgebra
  morphisms.  Then the following diagram shows that $\beta_{A,B}$ is a
  natural transformation:
  \[
  \bfig
  \square|amma|<900,500>[
    A \otimes B`
    eA \otimes eB`
    A' \otimes B'`
    eA' \otimes eB';
    h_A \otimes h_B`
    f \otimes g`
    ef \otimes eg`
    h_{A'} \otimes h_{B'}]

  \square(900,0)|amma|<900,500>[
    eA \otimes eB`
    eB \otimes eA`
    eA' \otimes eB'`
    eB' \otimes eA';
    \e{A,B}`
    ef \otimes eg`
    eg \otimes ef`
    \e{A',B'}]

  \square(1800,0)|amma|<900,500>[
    eB \otimes eA`
    B \otimes A`
    eB' \otimes eA'`
    B' \otimes A';
    \varepsilon_{B} \otimes \varepsilon_{A}`
    eg \otimes ef`
    g \otimes f`
    \varepsilon_{B'} \otimes \varepsilon_{A'}]
  \efig
  \]
  The left diagram commutes because $f$ and $g$ are both coalgebra
  morphisms, the middle diagram commutes because $\e{A,B}$ is a
  natural transformation, and the right diagram commutes by naturality
  of $\varepsilon$.
\end{proof}

\begin{corollary}
  \label{corollary:ex-simple}
  For any coalgebras $(A,h_A)$ and $(B,h_B)$ the following commutes:
    \[
    \bfig
    \square|aaaa|/->`=``->/<700,500>[
      A \otimes B`
      eA \otimes eB`
      A \otimes B`
      eA \otimes eB;
      h_A \otimes h_B```
      h_A \otimes h_B]

    \square(700,0)|aaaa|/->```->/<700,500>[
      eA \otimes eB`
      eB \otimes eA`
      eA \otimes eB`
      eB \otimes eA;
      \e{A,B}```
      \e{A,B}]

    \square(1400,0)|aama|/->``->`=/<700,500>[
      eB \otimes eA`
      B \otimes A`
      eB \otimes eA`
      eB \otimes eA;
      \varepsilon_B \otimes \varepsilon_A``
      h_B \otimes h_A`]
      \efig
    \]
\end{corollary}
\begin{proof}
  This proof follows by the fact that the following diagram commutes:
  \[
  \bfig
  \square|mmmm|/=`->`->`=/<900,500>[
    A \otimes B`
    A \otimes B`
    eA \otimes eB`
    eA \otimes eB;`
    h_A \otimes h_B`
    h_A \otimes h_B`]

  \qtriangle(0,-500)|mmm|/=`<-`->/<900,500>[
    eA \otimes eB`
    eA \otimes eB`
    e^2A \otimes e^2B;`
    \varepsilon_{eA} \otimes \varepsilon_{eB}`
    h_{eA} \otimes h_{eB}]

  \dtriangle(0,-1000)|mmm|/`->`<-/<900,500>[
    e^2A \otimes e^2B`
    eB \otimes eA`
    e^2B \otimes e^2A;`
    \e{eA,eB}`
    \varepsilon_{eB} \otimes \varepsilon_{eA}]

  \morphism(0,0)<0,-1000>[eA \otimes eB`eB \otimes eA;\e{A,B}]

  \square(900,0)|mmmm|/=`->`->`/<900,500>[
    A \otimes B`
    A \otimes B`
    eA \otimes eB`
    eA \otimes eB;`
    h_A \otimes h_B`
    h_A \otimes h_B`]

  \dtriangle(900,-1500)|mmm|/`->`<-/<900,1500>[
    eA \otimes eB`
    B \otimes A`
    eB \otimes eA;`
    \e{A,B}`
    \varepsilon_{B} \otimes \varepsilon_A]

  \morphism(900,-1500)|m|<-900,500>[B \otimes A`eB \otimes eA;h_B \otimes h_A]
  \efig
  \]
  The diagram on the right commutes because $\beta_{A,B}$ is a natural
  transformation, and the other diagrams commute either because
  $\cat{L}$ is a Lambek category with exchange, or by the action
  diagrams.
\end{proof}

\begin{definition}
  \label{def:cofork}
  Given two parallel arrows $f,g : B \mto C$ in a category $\cat{C}$,
  a \textbf{cofork} is a morphism $c : A \mto B$ such that
  the following diagram commutes:
  \[
  A \mto^c B \two^f_g C
  \]
  That is, $f \circ c = g \circ c$.
\end{definition}

\begin{lemma}
  \label{lemma:cofork-for-ex}
  The morphism $\e{A,B} \circ (\h{A} \otimes \h{B})$ is a cofork of
  the morphisms $(\h{B} \otimes \h{A}) \circ (\varepsilon_B \otimes
  \varepsilon_A)$ and $(e\varepsilon_B \otimes e\varepsilon_A) \circ
  (\delta_B \otimes \delta_A)$.
\end{lemma}
\begin{proof}
  We prove this by equational reasoning as follows:
  \[
  \small
  \begin{array}{lll}
    (\h{B} \otimes \h{A}) \circ (\varepsilon_B \otimes \varepsilon_A) \circ \e{A,B} \circ (\h{A} \otimes \h{B})
    \\\,\,\,\,\,\,\,\,\,\,\,
    = (\h{B} \otimes \h{A}) \circ (\varepsilon_B \otimes \varepsilon_A) \circ (\h{B} \otimes \h{A}) \circ \beta_{A,B}
    & \text{(Corollary~\ref{corollary:ex-simple})}\\
    \,\,\,\,\,\,\,\,\,\,\,= (\h{B} \otimes \h{A}) \circ ((\varepsilon_B \circ \h{B}) \otimes (\varepsilon_A \circ \h{A})) \circ \beta_{A,B}
    & \text{}\\
    \,\,\,\,\,\,\,\,\,\,\,= (\h{B} \otimes \h{A}) \circ (\id_B \otimes \id_A) \circ \beta_{A,B}
    & \text{(Action diagrams)}\\
    \,\,\,\,\,\,\,\,\,\,\,= (\h{B} \otimes \h{A}) \circ \beta_{A,B}
    & \text{}\\
    \,\,\,\,\,\,\,\,\,\,\,= \e{A,B} \circ (\h{A} \otimes \h{B})
    & \text{(Corollary~\ref{corollary:ex-simple})}\\
    \,\,\,\,\,\,\,\,\,\,\,= (\id_B \otimes \id_A) \circ \e{A,B} \circ (\h{A} \otimes \h{B})
    & \text{}\\
    \,\,\,\,\,\,\,\,\,\,\,= ((e\varepsilon_B \circ \delta_B) \otimes (e\varepsilon_A \circ \delta_A)) \circ \e{A,B} \circ (\h{A} \otimes \h{B})
    & \text{(Monoidal Comonad)}\\
    \,\,\,\,\,\,\,\,\,\,\,= (e\varepsilon_B \otimes e\varepsilon_A) \circ (\delta_B \otimes \delta_A) \circ \e{A,B} \circ (\h{A} \otimes \h{B})
    & \text{}\\
  \end{array}
  \]
\end{proof}

\begin{lemma}
  \label{lemma:beta-coalgebra-morph}
  In $\cat{L}^e$, $\beta$ is a coalgebra morphism.
\end{lemma}
\begin{proof}
  The proof follows from the commutativity of the following diagram:
  \[
  \small
  \bfig
  \square|amma|<800,500>[
    A \otimes B`
    eA \otimes eB`
    eA \otimes eB`
    e^2A \otimes e^2B;
    h_A \otimes h_B`
    h_A \otimes h_B`
    \delta_A \otimes \delta_B`
    eh_A \otimes eh_B]

  \square(800,0)|amma|<800,500>[
    eA \otimes eB`
    eB \otimes eA`
    e^2A \otimes e^2B`
    e^2B \otimes e^2A;
    \e{A,B}`
    \delta_A \otimes \delta_B`
    \delta_B \otimes \delta_A`
    \e{eA,eB}]

  \square(1600,0)|amma|<800,500>[
    eB \otimes eA`
    B \otimes A`
    e^2B \otimes e^2A`
    eA \otimes eB;
    \varepsilon_B \otimes \varepsilon_A`
    \delta_B \otimes \delta_A`
    h_B \otimes h_A`
    e\varepsilon_B \otimes e\varepsilon_A]

  \square(0,-500)|amma|<800,500>[
    eA \otimes eB`
    e^2A \otimes e^2B`
    e(A \otimes B)`
    e(eA \otimes eB);
    eh_A \otimes eh_B`
    \q{A,B}`
    \q{eA,eB}`
    e(h_A \otimes h_B)]

  \square(800,-500)|amma|<800,500>[
    e^2A \otimes e^2B`
    e^2B \otimes e^2A`
    e(eA \otimes eB)`
    e(eB \otimes eA);
    \e{eA,eB}`
    \q{eA,eB}`
    \q{eB,eA}`
    e\e{A,B}]

  \square(1600,-500)|amma|<800,500>[
    e^2B \otimes e^2A`
    eA \otimes eB`
    e(eB \otimes eA)`
    e(B \otimes A);
    e\varepsilon_B \otimes e\varepsilon_A`
    \q{eB,eA}`
    \q{B,A}`
    e(\varepsilon_B \otimes \varepsilon_A)]

  \place(400,250)[\text{(1)}]
  \place(1200,250)[\text{(2)}]
  \place(2000,250)[\text{(3)}]
  \place(400,-250)[\text{(4)}]
  \place(1200,-250)[\text{(5)}]
  \place(2000,-250)[\text{(6)}]
  \efig 
  \]
  Diagram one commutes by the action diagrams for the coalgebras
  $(A,h_A)$ and $(B,h_B)$, diagram two commutes because $\cat{L}$ is a
  Lambek category with exchange, diagram three does not commute, but
  holds by Lemma~\ref{lemma:cofork-for-ex}, diagram four and six
  commute by naturality of $\q{}$, and diagram five commutes because
  $\cat{L}$ is a Lambek category with exchange.
\end{proof}

\begin{lemma}[The Eilenberg-Moore Category is Symmetric Monoidal]
  \label{lemma:the_eilenberg-moore_category_is_symmetric_monoidal}
  The category $\cat{L}^e$ is symmetric monoidal.
\end{lemma}
\begin{proof}
  The following diagram shows that $\beta_{B,A} \circ \beta_{A,B} = \id_{A \otimes B}$:
  \[
  \bfig
  \square|amma|<800,500>[
    A \otimes B`
    eA \otimes eB`
    eA \otimes eB`
    e^2A \otimes e^2 B;
    h_A \otimes h_B`
    h_A \otimes h_B`
    \delta_A \otimes \delta_B`
    \delta_A \otimes \delta_B]

  \square(0,-500)|amma|<800,500>[
    eA \otimes eB`
    e^2A \otimes e^2B`
    eB \otimes eA`
    e^2B \otimes e^2A;
    \delta_A \otimes \delta_B`
    \e{A,B}`
    \e{eA,eB}`
    \delta_B \otimes \delta_A]

  \square(0,-1000)|ammm|<800,500>[
    eB \otimes eA`
    e^2B \otimes e^2A`
    B \otimes A`
    eB \otimes eA;
    \delta_B \otimes \delta_A`
    \varepsilon_B \otimes \varepsilon_A`
    e\varepsilon_{B} \otimes e\varepsilon_{A}`
    h_B \otimes h_A]

  \square(800,-1000)|ammm|<1200,500>[
    e^2B \otimes e^2A`
    e^2A \otimes e^2B`
    eB \otimes eA`
    eA \otimes eB;
    \e{eB,eA}`
    e\varepsilon_{B} \otimes e\varepsilon_{A}`
    e\varepsilon_{A} \otimes e\varepsilon_{B}`
    \e{B,A}]

  \btriangle(800,-500)|mmm|/`=`/<1200,500>[
    e^2A \otimes e^2B`
    e^2B \otimes e^2A`
    e^2A \otimes e^2B;``]

  \morphism(2000,-1000)<600,0>[
    eA \otimes eB`
    A \otimes B;
    \varepsilon_A \otimes \varepsilon_B]

  \morphism(0,500)/{@{=}@/^10em/}/<2600,-1500>[
    A \otimes B`
    A \otimes B;]

  \place(400,250)[(1)]
  \place(400,-250)[(2)]
  \place(400,-750)[(3)]
  \place(1100,-300)[(4)]
  \place(1400,-750)[(5)]
  \place(1400,250)[(6)]
  \efig
  \]
  Diagram one trivially commutes, diagram two, four, and five commute
  because $\cat{L}$ is a Lambek category with exchange, diagram three
  does not commute, but holds by Lemma~\ref{lemma:cofork-for-ex},
  diagrams six, seven, and eight commute by the fact that
  $(e,\varepsilon,\delta)$ is a comonad and the action diagrams of the
  Eilenberg Moore category.
  
  At this point we must verify that $\beta$ respects the coherence
  diagrams of a symmetric monoidal category; see
  Definition~\ref{def:sym-monoidal-category}.  Thus, we must show that
  each of the following diagrams hold:
  \begin{description}
  \item[Case]
    \[
    \bfig
      \hSquares|aammmaa|/->`->`->``->`->`->/[
        (A \otimes B) \otimes C`
        A \otimes (B \otimes C)`
        (B \otimes C) \otimes A`
        (B \otimes A) \otimes C`
        B \otimes (A \otimes C)`
        B \otimes (C \otimes A);
        \alpha_{A,B,C}`
        \beta_{A,B \otimes C}`
        \beta_{A,B} \otimes \id_C``
        \alpha_{B,C,A}`
        \alpha_{B,A,C}`
        \id_B \otimes \beta_{A,C}]
      \efig      
      \]
      We can show that this diagram commutes, by reducing it to the
      corresponding diagram on free coalgebras which we know holds by
      the assumption that $\cat{L}$ is a Lambek category with
      exchange.  This reduction is as follows (due to the size of the
      diagram it is broken up into three diagrams that can be
      straightforwardly composed):
      \begin{enumerate}
      \item[] Diagram 1:
        \[
        \bfig
        \square|amma|<1000,500>[
          (eB \otimes eA) \otimes C`
          (e^2B \otimes e^2A) \otimes C`
          (B \otimes A) \otimes C`
          (eB \otimes eA) \otimes C;
          (\delta \otimes \delta) \otimes \id`
          (\varepsilon \otimes \varepsilon) \otimes \id`
          (e\varepsilon \otimes e\varepsilon) \otimes \id`
          (h_B \otimes h_A) \otimes \id]


        \square|amma|/{@{->}@/^2em/}`->`->`/<2000,500>[
          (eB \otimes eA) \otimes C`
          (eB \otimes eA) \otimes eC`
          (B \otimes A) \otimes C`
          eB \otimes (eA \otimes eC);
          \id \otimes h_C`
          (\varepsilon \otimes \varepsilon) \otimes \id`
          \alpha`]

        \morphism(1000,0)|m|<1000,500>[
          (eB \otimes eA) \otimes C`
          (eB \otimes eA) \otimes eC;
          \id \otimes h_C]

        \square(0,-500)|amma|/`=`<-`->/<2000,500>[
          (B \otimes A) \otimes C`
          eB \otimes (eA \otimes eC)`
          (B \otimes A) \otimes C`
          B \otimes (A \otimes C);``
          h_B \otimes (h_A \otimes h_C)`
          \alpha]

        \square(0,-1000)|amma|/->`=`->`/<2000,500>[
          (B \otimes A) \otimes C`
          B \otimes (A \otimes C)`
          (B \otimes A) \otimes C`
          B \otimes (eA \otimes eC);
          \alpha``
          \id \otimes (h_A \otimes h_C)`]

        \morphism(0,-1000)<1000,0>[
          (B \otimes A) \otimes C`
          B \otimes (A \otimes C);
          \alpha]

        \morphism(1000,-1000)<1000,0>[
          B \otimes (A \otimes C)`
          B \otimes (eA \otimes eC);
          \id \otimes (h_A \otimes h_C)]

        \square(0,500)|amma|/->`->`->`/<2000,500>[
          (eA \otimes eB) \otimes C`
          (eA \otimes eB) \otimes eC`
          (eB \otimes eA) \otimes C`
          (eB \otimes eA) \otimes eC;
          \id \otimes h_C`
          \e{} \otimes \id`
          \e{} \otimes \id`]

        \square(0,1000)|amma|/=`->`->`->/<2000,500>[
          (A \otimes B) \otimes C`
          (A \otimes B) \otimes C`
          (eA \otimes eB) \otimes C`
          (eA \otimes eB) \otimes eC;`
          (h_A \otimes h_B) \otimes \id`
          (h_A \otimes h_B) \otimes h_C`
          \id \otimes h_C]

        \place(500,250)[(2)]
        \place(1450,450)[(1)]
        \place(1000,-250)[(3)]
        \efig
        \]        
        Diagram one commutes because $(e,\varepsilon,\delta)$ is a
        comonad, diagram two does not commute, but holds by
        Lemma~\ref{lemma:cofork-for-ex}, and diagram 3 commutes by
        naturality of $\alpha$.

      \item[] Diagram 2:
      \item[] Diagram 3:        
      \end{enumerate}

    \item[Case]
      \[
      \bfig
      \btriangle[
        A \otimes B`
        B \otimes A`
        A \otimes B;
        \beta_{A,B}`
        \id_{A \otimes B}`
        \beta_{B,A}]
      \efig
      \]

    \item[Case]
      \[
      \bfig
      \Vtriangle[
        \top \otimes A`
        A \otimes \top`
        A;
        \beta_{\top,A}`
        \lambda_A`
        \rho_A]
      \efig
      \]
  \end{description}
\end{proof}

\begin{definition}
  \label{def:cokleisli-exchange}
  Let $(\cat{L},e,\e{})$ be a Lambek category with exchange. The
  \textbf{coKleisli Category of $e$}, $\cat{L}_e$, is a category with the
  same objects as $\cat{L}$. There is an arrow $\hat{f}:A\mto B$ in
  $\cat{L}_e$ if there is an arrow $f:eA\mto B$ in $\cat{L}$. The
  identity arrow $\hat{id_A}:A\mto A$ is the arrow
  $\varepsilon_A:eA\mto A$ in $\cat{L}$. Given $\hat{f}:A\mto B$ and
  $\hat{g}:B\mto C$ in $\cat{L}_e$, which are arrows
  $f:eA\mto B$ and $g:eB\mto C$ in $\cat{L}$, the composition
  $\hat{g}\circ\hat{f}:A\mto C$ is defined as $g\circ ef\circ\delta_A$.
\end{definition}
