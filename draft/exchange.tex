\begin{definition}
  \label{def:exchange}
  A \textbf{Lambek category with exchange}, $(\cat{L},e,\e{})$, is a
  Lambek category equipped with a monoidal comonad
  $(e,\varepsilon,\delta)$ on $\cat{L}$, and a monoidal natural
  transformation $\e{A,B}:eA \otimes eB \mto eB \otimes eA$.  We
  require $\e{}$ to be a coalgebra morphism, and the following
  diagrams must commute:
  \[
  \small
  \begin{array}{cccccccc}
    \bfig
    \square|amma|<700,500>[
      e^2A \otimes e^2B`
      e^2B \otimes e^2A`
      e(eA \otimes eB)`
      e(eB \otimes eA);
      \e{eA,eB}`
      \q{eA,eB}`
      \q{eB,eA}`
      e\e{A,B}]
    \efig
    &
    \quad
    &    
    \bfig
    \qtriangle|amm|/->`=`->/<700,500>[
      eA \otimes eB`
      eB \otimes eA`
      eA \otimes eB;
      \e{A,B}``
      \e{B,A}]
    \efig
    &
    \quad
    &
    \bfig
    \square|amma|/->`->`->`/<800,400>[
      eI \otimes eA`
      eA \otimes eI`
      I \otimes eA`
      eA \otimes I;
      \e{I,A}`
      \varepsilon_I \otimes \id`
      \id \otimes \varepsilon_I`]

    \Vtriangle(0,-200)/`->`->/<400,200>[
      I \otimes eA`
      eA \otimes I`
      eA;`
      \lambda_{eA}`
      \rho_{eA}]
    \efig
  \end{array}
  \]
  \[
  \begin{array}{cccccccc}
    \bfig
    \square|amma|/->`->``/<1100,400>[
      (eA \otimes eB) \otimes eC`
      eA \otimes (eB \otimes eC)`
      (eB \otimes eA) \otimes eC`;
      \alpha_{eA,eB,eC}`
      \e{A,B} \otimes \id_{eC}``]
    
    \morphism(0,0)|m|/->/<0,-400>[
      (eB \otimes eA) \otimes eC`
      eB \otimes (eA \otimes eC);
      \alpha_{eB,eA,eC}]

    \morphism(0,-400)|a|/->/<2100,0>[
      eB \otimes (eA \otimes eC)`
      eB \otimes (eC \otimes eA);
      \id_{eB} \otimes \e{A,C}]
    
    \morphism(1100,400)|a|/->/<1150,0>[
      eA \otimes (eB \otimes eC)`
      eA \otimes (e^2B \otimes e^2C);
      \id_{eA} \otimes (\delta_{B} \otimes \delta_C)]
    \morphism(2250,400)|a|/->/<1000,0>[
      eA \otimes (e^2B \otimes e^2C)`
      eA \otimes e(eB \otimes eC);
      \id_{eA} \otimes \q{eB,eC}]
    \morphism(3250,400)|m|/->/<0,-400>[
      eA \otimes e(eB \otimes eC)`
      e(eB \otimes eC) \otimes eA;
      \e{eA,eB \otimes eC}]
    \morphism(3250,0)|m|/->/<0,-400>[
      e(eB \otimes eC) \otimes eA`
      (eB \otimes eC) \otimes eA;
      \varepsilon_{eB \otimes eC} \otimes \id_{eA}]
    \morphism(3250,-400)|a|/->/<-1150,0>[
      (eB \otimes eC) \otimes eA`
      eB \otimes (eC \otimes eA);
      \alpha_{eB,eC,eA}]
    \efig
  \end{array}
  \]
  Furthermore, for any coalgebra morphisms $f : (eA, \delta) \mto
  (eB,\delta)$ and $g : (eC,\delta) \mto (eD,\delta)$ between free
  coalgebras the following diagram must commute:
  \begin{center}
    \begin{math}
      \bfig
      \square(0,-500)|amma|<800,500>[
        eA \otimes eC`
        eB \otimes eD`
        eC \otimes eA`
        eD \otimes eB;
        f \otimes g`
        \e{A,C}`
        \e{B,D}`
        g \otimes f]      
      \efig
    \end{math}
  \end{center}
  The morphism $\q{A,B} : eA \otimes eB \mto e(A \otimes B)$ makes
  $(e,\q{})$ a monoidal functor.
\end{definition}
The first diagram in the previous definition makes $e : \cat{L} \mto
\cat{L}$ a symmetric monoidal functor, and the second, third, and
forth diagrams make the category of free coalgebras (the Kleisli
category) symmetric monoidal.

\begin{definition}
  \label{def:eilenberg-moore-cat}
  Suppose $(\cat{L},e,\e{})$ is a Lambek category with exchange.  Then
  the \textbf{Eilenberg Moore category}, $\cat{L}^e$, of the comonad
  $(e, \varepsilon, \delta)$ has as objects all the e-coalgebras $(A,
  h_A : A \mto eA)$, and as morphisms all the coalgebra morphisms.  We
  call $h_A$ the action of the coalgebra.  Furthermore, the following
  (action) diagrams must commute:
  \begin{mathpar}
    \bfig
    \square[A`eA`eA`e^2A;h_A`h_A`eh_A`\delta_A]    
    \efig
    \and
    \bfig
    \btriangle/->`=`->/[A`eA`A;h_A``\varepsilon_A]
    \efig
  \end{mathpar}
\end{definition}

\begin{lemma}[The Eilenberg Moore Category of the comonad $e$ is Monoidal]
  \label{lemma:the_eilenberg_moore_category_is_monoidal}
  The category $\cat{L}^e$ is monoidal.
\end{lemma}
\begin{proof}
  For the complete proof see
  Appendix~\ref{subsec:proof_of_the_eilenberg-moore_category_is_monoidal_lemma:the_eilenberg_moore_category_is_monoidal}.
\end{proof}

\begin{lemma}
  \label{lemma:pseudo-braided}
  In $\cat{L}^e$ there is a natural transformation $\beta_{A,B} : A \otimes B \mto B \otimes A$.
\end{lemma}
\begin{proof}
  We define $\beta$ as follows:
  \begin{center}
    \begin{math}
      \beta_{A,B} := A \otimes B \mto^{h_A \otimes h_B} eA \otimes eB \mto^{ex_{A,B}} eB \otimes eA \mto^{\varepsilon_B \otimes \varepsilon_A} B \otimes A
    \end{math}
  \end{center}
For the proof that it is natural please see Appendix~\ref{subsec:proof_of_lemma:pseudo-braided}.
\end{proof}

\begin{corollary}
  \label{corollary:ex-simple}
  For any coalgebras $(A,h_A)$ and $(B,h_B)$ the followings commute:
  \begin{center}
    \begin{math}
      \bfig
      \square|aaaa|/->`=``->/<700,500>[
        A \otimes B`
        eA \otimes eB`
        A \otimes B`
        eA \otimes eB;
        h_A \otimes h_B```
        h_A \otimes h_B]

      \square(700,0)|aaaa|/->```->/<700,500>[
        eA \otimes eB`
        eB \otimes eA`
        eA \otimes eB`
        eB \otimes eA;
        \e{A,B}```
        \e{A,B}]

      \square(1400,0)|aama|/->``->`=/<700,500>[
        eB \otimes eA`
        B \otimes A`
        eB \otimes eA`
        eB \otimes eA;
        \varepsilon_B \otimes \varepsilon_A``
        h_B \otimes h_A`]
      \efig
    \end{math}
  \end{center}
  \begin{center}
    \begin{math}
      \bfig
       \square<800,400>[
         eA\otimes eB`eB\otimes eA`e(A\otimes B)`e(B\otimes A);
         \e{A,B}`\q{A,B}`\q{B,A}`e\beta_{A,B}]
      \efig
    \end{math}
  \end{center}
\end{corollary}
\begin{proof}
For the complete proof please see
Appendix~\ref{subsec:proof_of_corollary:ex-simple}.
\end{proof}


\begin{definition}
  \label{def:cofork}
  Given two parallel arrows $f,g : B \mto C$ in a category $\cat{C}$,
  a \textbf{cofork} is a morphism $c : A \mto B$ such that the diagram
  $A \mto^c B \two^f_g C$ commutes.  That is, $f \circ c = g \circ c$.
\end{definition}

\begin{lemma}
  \label{lemma:cofork-for-ex}
  The morphism $\e{A,B} \circ (\h{A} \otimes \h{B})$ is a cofork of
  the morphisms $(\h{B} \otimes \h{A}) \circ (\varepsilon_B \otimes
  \varepsilon_A)$ and $(e\varepsilon_B \otimes e\varepsilon_A) \circ
  (\delta_B \otimes \delta_A)$.
\end{lemma}
\begin{proof}
  This proof holds by straightforward equational reasoning.  For the
  complete proof please see
  Appendix~\ref{subsec:proof_of_lemma:cofork-for-ex}.
\end{proof}

\begin{lemma}
  \label{lemma:beta-coalgebra-morph}
  In $\cat{L}^e$, $\beta$ is a coalgebra morphism.
\end{lemma}
\begin{proof}
  For the complete proof see
  Appendix~\ref{subsec:proof_of_lemmabeta-coalgebra-morph}.
\end{proof}

\begin{lemma}[The Eilenberg-Moore Category of the comonad $e$ is Symmetric Monoidal]
  \label{lemma:the_eilenberg-moore_category_is_symmetric_monoidal}
  The category $\cat{L}^e$ is symmetric monoidal.
\end{lemma}
\begin{proof}
  For the complete proof please see
  Appendix~\ref{subsec:proof_of_the_eilenberg-moore_category_is_symmetric_lemma:the_eilenberg-moore_category_is_symmetric_monoidal}.
\end{proof}

