We now turn to how adjoint resources models correspond to various
substructural logics.  The definition of the model along with
Benton's~\cite{Benton:1994} LNL adjoint logic suggests a clear design
of each adjoint resource logic.

Every resource adjoint logic begins with the base logic consisting of
the Lambek calculus potentially extended with one or more of the
structural rules: exchange, weakening, or contraction.  Then we
connect this base logic with each other logic corresponding to the
categories in a composition of structure.  Each of these logics are
then connected by syntactic versions of the adjoint functors.

Now we give several example logics with increasing complexity, and
then conclude with a discussion for a general system for resource
tracking.  We begin with the most basic resource adjoint logic.

\subsection{Lambek Calculus with Itself}
\label{subsec:lambek_calculus_with_itself}
\input{lambek-itself-ott}
% subsection lambek_calculus_with_itself (end)

\subsection{Lambek Calculus with Exchange}
\label{subsec:lambek_calculus_with_exchange}
\input{lambek-exchange-ott}
% subsection lambek_calculus_with_exchange (end)

\subsection{Lambek Calculus with Exchange and Commutative Affine Logic}
\label{subsec:linear_logic_with_affine_logic}

% subsection linear_logic_with_affine_logic (end)

\subsection{Linear Logic with Intuitionistic Logic, Affine Logic, and Contraction Logic}
\label{subsec:linear_logic_with_affine_logic_and_contraction_logic}

% subsection linear_logic_with_affine_logic_and_contraction_logic (end)

\subsection{Complete Resource Tracking}
\label{subsec:complete_resource_tracking}

% subsection complete_resource_tracking (end)

%%% Local Variables: 
%%% mode: latex
%%% TeX-master: main.tex
%%% End: 
