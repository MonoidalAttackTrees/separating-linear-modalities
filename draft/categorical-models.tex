We develop a categorical framework in which many different
intuitionistic substructural logics may be modeled.  The locus of this
framework is an adjunction.  We initially take a monoidal category,
$\cat{L}$, as a base, and then extend it with one or more structural
morphisms -- a morphism corresponding to a structural rule in logic --
to obtain a second category $\hat{\cat{L}}$.  Then we form a monoidal
adjunction $\hat{\cat{L}} : F \dashv G : \cat{L}$ just as
Benton~\cite{Benton:1994} did for intuitionistic linear logic.
Depending on which structural morphisms we add to $\hat{\cat{L}}$ we
will obtain different models.  In particular, each model will come
endowed with a comonad on $\cat{L}$ which equips $\cat{L}$ with the
ability to track the corresponding structural rule(s).

We will show that by adding the morphisms for either weakening,
contraction, or exchange, to $\cat{L}$ will yield an adjoint model of
non-commutative relevance logic/linear logic, non-commutative
contraction logic/linear logic, and commutative/non-commutative linear
logic.  The latter model will come with a monoidal comonad $e :
\cat{L} \mto \cat{L}$ such that there is a symmetry $\e{A,B} : eA \seq
eB \mto eB \seq eA$, where $- \seq -$ denotes a non-commutative tensor
product.  In fact, this is the first adjoint model of the Lambek
calculus with the exchange comonad.

At this point we will have adjoint models for each individual
structural rule.  What if we want more than one structural rule?
There are a few different choices that one can choose from depending
on the scenario.  First, if $\hat{\cat{L}}$ contains more than one
structural rule, then $\cat{L}$ will have a single comonad that adds
all of those structural rules to $\cat{L}$.  For example, if
$\hat{\cat{L}}$ contains weakening, contraction, and exchange, then
$\hat{\cat{L}}$ is cartesian closed and $\cat{L}$ will have the usual
$! : \cat{L} \mto \cat{L}$ comonad.  The second scenario is when
$\cat{L}$ also contains some structural rules.  For example, if
$\hat{\cat{L}}$ contains exchange and weakening and $\cat{L}$ contains
exchange, then $\cat{L}$ will have a comonad, $r : \cat{L} \mto
\cat{L}$, which combines linear logic with relevance logic.  Thus, how
we instantiate the two categories in the adjunction influences which
logic one may model.

What if we want multiple comonads tracking different logics?  In this
scenario the different comonads would allow us to mix the different
logics in interesting ways.  Suppose $\cat{L}$ has no structural rules
and $\cat{E}$ is $\cat{L}$ with exchange and $\cat{EW}$ is $\cat{E}$
with weakening.  Then we can form two adjunctions $\cat{E} : F \dashv
G : \cat{L}$ and $\cat{EW} : H \dashv J : \cat{L}$, but the categories
$\cat{E}$ and $\cat{EW}$ have a structural rule in common.  So
instead, we form the adjunction $\cat{EW} : H \dashv J : \cat{E} : F
\dashv G : \cat{L}$.  Thus, $\cat{L}$ has the exchange comonad $e = FG
: \cat{L} \mto \cat{L}$ as well as the relevance logic comonad $r =
FHJG: \cat{L} \mto \cat{L}$.  Additionally, there is a comonad $w = JH
: \cat{E} \mto \cat{E}$ adding weakening to $\cat{E}$.


\subsection{Lambek Categories}
\label{subsec:lambek_categories}
\begin{definition}
  \label{def:Lambek-category}
  A \textbf{monoidal category}, $(\cat{L}, \otimes, I, \lambda,
  \rho)$, is a category, $\cat{L}$, equipped with a bifunctor,
  $\otimes : \cat{L} \times \cat{L} \mto \cat{L}$, called the tensor
  product, a distinguished object $I$ of $\cat{L}$ called the unit,
  and three natural isomorphisms $\lambda_A : I \otimes A \mto A$,
  $\rho_A : A \otimes I \mto A$, and $\alpha_{A,B,C} : A \otimes (B
  \otimes C) \mto (A \otimes B) \otimes C$ called the left and right
  unitors and the associator respectively.  Finally, these are subject
  to the following coherence diagrams:
  \begin{mathpar}
    \bfig
    \hSquares|aammmma|/->`->`->``->``/<400>[
      ((A \otimes B) \otimes C) \otimes D`
      (A \otimes (B \otimes C)) \otimes D`
      A \otimes ((B \otimes C) \otimes D)`
      (A \otimes B) \otimes (C \otimes D)``
      A \otimes (B \otimes (C \otimes D));
      \alpha_{A,B,C} \otimes \id_D`
      \alpha_{A,B \otimes C,D}`
      \alpha_{A \otimes B,C,D}``
      \id_A \otimes \alpha_{B,C,D}``]

    \morphism(-200,0)<2700,0>[
      (A \otimes B) \otimes (C \otimes D)`
      A \otimes (B \otimes (C \otimes D));
      \alpha_{A,B,C \otimes D}]
    \efig
    \and
    \bfig
    \Vtriangle[
      (A \otimes I) \otimes B`
      A \otimes (I \otimes B)`
      A \otimes B;
      \alpha_{A,I,B}`
      \rho_{A}\otimes id_B`
      id_A\otimes\lambda_{B}]
    \efig
  \end{mathpar}
\end{definition}

\begin{definition}
  \label{def:Lambek-category}
  A \textbf{Lambek category} is a monoidal category $(\cat{L},
  \otimes, I, \lambda, \rho,\alpha)$ equipped with two bifunctors
  $\lto : \catop{L} \times \cat{L} \mto \cat{L}$ and $\rto : \cat{L}
  \times \catop{L} \mto \cat{L}$ that are both right adjoint to the
  tensor product.  That is, the following natural bijections hold:
  \begin{center}
    \begin{math}
      \begin{array}{lll}
        \Hom{L}{X \otimes A}{B} \cong \Hom{L}{X}{A \lto B} & \quad\quad\quad\quad & 
        \Hom{L}{A \otimes X}{B} \cong \Hom{L}{X}{B \rto A}\\
      \end{array}
    \end{math}
  \end{center}
\end{definition}
One might call Lambek categories biclosed monoidal categories, but we
name them in homage to Lambek for his contributions to non-commutative
linear logic.

\begin{definition}
  \label{def:sym-monoidal-category}
  A monoidal category $(\cat{L},\otimes,I,\lambda,\rho,\alpha)$ is
  \textbf{symmetric} if there is a natural transformation $\beta_{A,B}
  : A \otimes B \mto B \otimes A$ such that $\beta_{B,A} \circ
  \beta_{A,B} = \id_{A \otimes B}$ and the following commute:
  \begin{center}
    \begin{math}
      \small
      \begin{array}{lll}
        \bfig
        \hSquares|aammmaa|/->`->`->``->`->`->/[
        (A \otimes B) \otimes C`
        A \otimes (B \otimes C)`
        (B \otimes C) \otimes A`
        (B \otimes A) \otimes C`
        B \otimes (A \otimes C)`
        B \otimes (C \otimes A);
        \alpha_{A,B,C}`
        \beta_{A,B \otimes C}`
        \beta_{A,B} \otimes \id_C``
        \alpha_{B,C,A}`
        \alpha_{B,A,C}`
        \id_B \otimes \beta_{A,C}]
        \efig
        & \quad &
        \bfig
          \Vtriangle[
            I \otimes A`
            A \otimes I`
            A;
            \beta_{I,A}`
            \lambda_A`
            \rho_A]
          \efig
      \end{array}
    \end{math}
  \end{center}
\end{definition}

\begin{definition}
  \label{def:sym-monoidal-closed}
  A symmetric monoidal category $(\cat{L}, \otimes, I, \lambda, \rho,
  \alpha, \beta)$ is \textbf{closed} if it comes equipped with a
  bifunctor $\limp : \catop{L} \times \cat{L} \mto \cat{L}$ that is
  right adjoint to the tensor product.  That is, the following natural
  bijection $\Hom{L}{X \otimes A}{B} \cong \Hom{L}{X}{A \limp B}$ holds.
\end{definition}

\begin{definition}
  \label{def:weakening}
  A \textbf{Lambek category with weakening}, $(\cat{L}, \otimes, I,
  \lambda, \rho, \alpha, \w{})$, is a Lambek category $(\cat{L}, \otimes, I,
  \lambda, \rho,\alpha)$ equipped with a natural transformation
  $\w{A} : A \mto I$.
\end{definition}

\begin{definition}
  \label{def:contraction}
  A \textbf{Lambek category with contraction}, $(\cat{L}, \otimes, I,
  \lambda, \rho, \alpha, \cL{},\cR{})$, is a Lambek category $(\cat{L}, \otimes, I,
  \lambda, \rho,\alpha)$ equipped with natural transformations:
  \[
  \begin{array}{lll}
    \cL{A,B} : (A \otimes B) \mto (A \otimes B) \otimes A & \quad &
    \cR{A,B} : (B \otimes A) \mto A \otimes (B \otimes A)\\
  \end{array}
  \]
  Furthermore, the following diagrams must commute:
    \begin{mathpar}
      \bfig
      \square/<-`->``/<1050,400>[
	A\otimes I`
        A`
        (A\otimes I)\otimes A`;
	\rho_{A}^{-1}`
	\cL{A,I}``]
      \square(1050,0)/->``->`/<1050,400>[
        A`
        I\otimes A``
        A\otimes(I\otimes A);
        \lambda_{A}^{-1}``
	\cR{A,I}`]
        \morphism(0,0)|b|<2100,0>[(A\otimes I)\otimes A`A\otimes(I\otimes A);\alpha_{A,I,A}]
      \efig
    \end{mathpar}
    \begin{mathpar}
    \bfig
      \square/->`->``->/<1300,800>[
        A\otimes A`
        A\otimes(A\otimes I)`
        (I\otimes A)\otimes A`
        (A\otimes(I\otimes A))\otimes A;
        id_{A}\otimes\rho_{A}^{-1}`
        \lambda_{A}^{-1}\otimes id_{A}``
        \cR{A,I}\otimes id_{A}]
      \qtriangle(1300,400)/->``->/<1300,400>[
        A\otimes(A\otimes I)`
        A\otimes((A\otimes I)\otimes A)`
        A\otimes(A\otimes A);
        id_{A}\otimes\cL{A,I}``
        id_{A}\otimes(\rho_{A}\otimes id_{A})]
      \dtriangle(1300,0)/`<-`->/<1300,400>[
        A\otimes(A\otimes A)`
        (A\otimes(I\otimes A))\otimes A`
        (A\otimes A)\otimes A;
        `
        \alpha_{A,A,A}`
        (id_{A}\otimes\lambda_{A})\otimes id_{A}]
    \efig
    \end{mathpar}
\end{definition}

\begin{definition}
  \label{def:exchange}
  A \textbf{Lambek category with exchange}, $(\cat{L}, \otimes, I,
  \lambda, \rho, \alpha, \e{})$, is a Lambek category, $(\cat{L},
  \otimes, I, \lambda, \rho, \alpha)$, such that $\cat{L}$ is
  symmetric monoidal, where $\e{A,B} : A \otimes B \mto B \otimes A$
  is the symmetry.
\end{definition}

% subsection lambek_categories (end)
