\subsection{Symmetric Monoidal Categories}
\label{subsec:symmetric_monoidal_categories}

\begin{definition}
  \label{def:monoidal-category}
  A \textbf{monoidal category} is a category, $\cat{M}$,
  with the following data:
  \begin{itemize}
  \item An object $\top$ of $\cat{M}$,
  \item A bi-functor $\otimes : \cat{M} \times \cat{M} \mto \cat{M}$,
  \item The following natural isomorphisms:
    \[
    \begin{array}{lll}
      \lambda_A : \top \otimes A \mto A\\
      \rho_A : A \otimes \top \mto A\\      
      \alpha_{A,B,C} : (A \otimes B) \otimes C \mto A \otimes (B \otimes C)\\
    \end{array}
    \]
  \item Subject to the following coherence diagrams:
    \begin{mathpar}
      \bfig
      \vSquares|ammmmma|/->`->```->``<-/[
        ((A \otimes B) \otimes C) \otimes D`
        (A \otimes (B \otimes C)) \otimes D`
        (A \otimes B) \otimes (C \otimes D)``
        A \otimes (B \otimes (C \otimes D))`
        A \otimes ((B \otimes C) \otimes D);
        \alpha_{A,B,C} \otimes \id_D`
        \alpha_{A \otimes B,C,D}```
        \alpha_{A,B,C \otimes D}``
        \id_A \otimes \alpha_{B,C,D}]      
      
      \morphism(1185,1000)|m|<0,-1000>[
        (A \otimes (B \otimes C)) \otimes D`
        A \otimes ((B \otimes C) \otimes D);
        \alpha_{A,B \otimes C,D}]
      \efig
    \end{mathpar}
    \begin{mathpar}
      \bfig
      \Vtriangle[
        (A \otimes \top) \otimes B`
        A \otimes (\top \otimes B)`
        A \otimes B;
        \alpha_{A,\top,B}`
        \rho_{A}\otimes id_B`
        id_A\otimes\lambda_{B}]
      \efig
    \end{mathpar}
  \end{itemize}
\end{definition}

% Definition: symmetric monoidal category
\begin{definition}
  \label{def:sym-monoidal-category}
  A \textbf{symmetric monoidal category (SMC)} is a category, $\cat{M}$,
  with the following data:
  \begin{itemize}
  \item An object $\top$ of $\cat{M}$,
  \item A bi-functor $\otimes : \cat{M} \times \cat{M} \mto \cat{M}$,
  \item The following natural isomorphisms:
    \[
    \begin{array}{lll}
      \lambda_A : \top \otimes A \mto A\\
      \rho_A : A \otimes \top \mto A\\      
      \alpha_{A,B,C} : (A \otimes B) \otimes C \mto A \otimes (B \otimes C)\\
    \end{array}
    \]
  \item A symmetry natural isomorphism:
    \[
    \beta_{A,B} : A \otimes B \mto B \otimes A
    \]
  \item Subject to the following coherence diagrams:
    \begin{mathpar}
      \bfig
      \vSquares|ammmmma|/->`->```->``<-/[
        ((A \otimes B) \otimes C) \otimes D`
        (A \otimes (B \otimes C)) \otimes D`
        (A \otimes B) \otimes (C \otimes D)``
        A \otimes (B \otimes (C \otimes D))`
        A \otimes ((B \otimes C) \otimes D);
        \alpha_{A,B,C} \otimes \id_D`
        \alpha_{A \otimes B,C,D}```
        \alpha_{A,B,C \otimes D}``
        \id_A \otimes \alpha_{B,C,D}]      
      
      \morphism(1185,1000)|m|<0,-1000>[
        (A \otimes (B \otimes C)) \otimes D`
        A \otimes ((B \otimes C) \otimes D);
        \alpha_{A,B \otimes C,D}]
      \efig
      \and
      \bfig
      \hSquares|aammmaa|/->`->`->``->`->`->/[
        (A \otimes B) \otimes C`
        A \otimes (B \otimes C)`
        (B \otimes C) \otimes A`
        (B \otimes A) \otimes C`
        B \otimes (A \otimes C)`
        B \otimes (C \otimes A);
        \alpha_{A,B,C}`
        \beta_{A,B \otimes C}`
        \beta_{A,B} \otimes \id_C``
        \alpha_{B,C,A}`
        \alpha_{B,A,C}`
        \id_B \otimes \beta_{A,C}]
      \efig      
    \end{mathpar}
    \begin{mathpar}
      \bfig
      \Vtriangle[
        (A \otimes \top) \otimes B`
        A \otimes (\top \otimes B)`
        A \otimes B;
        \alpha_{A,\top,B}`
        \rho_{A}\otimes id_B`
        id_A\otimes\lambda_{B}]
      \efig
      \and
      \bfig
      \btriangle[
        A \otimes B`
        B \otimes A`
        A \otimes B;
        \beta_{A,B}`
        \id_{A \otimes B}`
        \beta_{B,A}]
      \efig
      \and
      \bfig
      \Vtriangle[
        \top \otimes A`
        A \otimes \top`
        A;
        \beta_{\top,A}`
        \lambda_A`
        \rho_A]
      \efig
    \end{mathpar}    
  \end{itemize}
\end{definition}

% Definition: monoidal biclosed category
\begin{definition}
  \label{def:monoidal-biclosed-category}
  A \textbf{monoidal biclosed category} is a monoidal category
  $(\cat{M},\top,\otimes)$, such that, for any object $B$ of $\cat{M}$,
  each of the functors $-\otimes B:\cat{M}\mto\cat{M}$ and
  $B\otimes -:\cat{M}\mto\cat{M}$ has a specified right adjoint. Hence,
  for any object $A$ and $C$ of $\cat{M}$, there are two objects
  $C\leftharpoonup B$ and $B\rightharpoonup C$ of $\cat{M}$ and two
  natural bijections:
  \begin{align*}
  \Hom{\cat{M}}{A\otimes B}{C} &\cong
    \Hom{\cat{M}}{A}{C\leftharpoonup B} \\
  \Hom{\cat{M}}{B \otimes A}{C} &\cong
    \Hom{\cat{M}}{A}{B \rightharpoonup C}
  \end{align*}
\end{definition}

% Definition: symmetric monoidal closed category
\begin{definition}
  \label{def:SMCC}
  A \textbf{symmetric monoidal closed category (SMCC)} is a symmetric
  monoidal category, $(\cat{M},\top,\otimes)$, such that, for any object
  $B$ of $\cat{M}$, the functor $- \otimes B : \cat{M} \mto \cat{M}$
  has a specified right adjoint.  Hence, for any objects $A$ and $C$
  of $\cat{M}$ there is an object $B \limp C$ of $\cat{M}$ and a
  natural bijection:
  \[
  \Hom{\cat{M}}{A \otimes B}{C} \cong \Hom{\cat{M}}{A}{B \limp C}
  \]
  We call the functor $\limp : \cat{M} \times \cat{M} \mto \cat{M}$
  the internal hom of $\cat{M}$.
\end{definition}

\begin{definition}
  \label{def:MCFUN}
  Suppose we are given two monoidal categories
  $(\cat{M}_1,\top_1,\otimes_1,\alpha_1,\lambda_1,\rho_1,\beta_1)$ and
  $(\cat{M}_2,\top_2,\otimes_2,\alpha_2,\lambda_2,\rho_2,\beta_2)$.  Then a
  \textbf{monoidal functor} is a functor $F : \cat{M}_1 \mto
  \cat{M}_2$, a map $m_{\top_1} : \top_2 \mto F\top_1$ and a natural transformation
  $m_{A,B} : FA \otimes_2 FB \mto F(A \otimes_1 B)$ subject to the
  following coherence conditions:
  \begin{mathpar}
    \bfig
    \vSquares|ammmmma|/->`->`->``->`->`->/[
      (FA \otimes_2 FB) \otimes_2 FC`
      FA \otimes_2 (FB \otimes_2 FC)`
      F(A \otimes_1 B) \otimes_2 FC`
      FA \otimes_2 F(B \otimes_1 C)`
      F((A \otimes_1 B) \otimes_1 C)`
      F(A \otimes_1 (B \otimes_1 C));
      {\alpha_2}_{FA,FB,FC}`
      m_{A,B} \otimes \id_{FC}`
      \id_{FA} \otimes m_{B,C}``
      m_{A \otimes_1 B,C}`
      m_{A,B \otimes_1 C}`
      F{\alpha_1}_{A,B,C}]
    \efig
    \end{mathpar}
  \begin{mathpar}
    \bfig
    \square|amma|/->`->`<-`->/<1000,500>[
      \top_2 \otimes_2 FA`
      FA`
      F\top_1 \otimes_2 FA`
      F(\top_1 \otimes_1 A);
      {\lambda_2}_{FA}`
      m_{\top_1} \otimes \id_{FA}`
      F{\lambda_1}_{A}`
      m_{\top_1,A}]
    \efig
    \and
    \bfig
    \square|amma|/->`->`<-`->/<1000,500>[
      FA \otimes_2 \top_2`
      FA`
      FA \otimes_2 F\top_1`
      F(A \otimes_1 \top_1);
      {\rho_2}_{FA}`
      \id_{FA} \otimes m_{\top_1}`
      F{\rho_1}_{A}`
      m_{A,\top_1}]
    \efig
    \end{mathpar}
  Need to notice that the composition of monoidal functors is also monoidal,
  subject to the above coherence conditions.

\end{definition}

\begin{definition}
  \label{def:SMCFUN}
  Suppose we are given two symmetric monoidal closed categories\\
  $(\cat{M}_1,\top_1,\otimes_1,\alpha_1,\lambda_1,\rho_1,\beta_1)$ and
  $(\cat{M}_2,\top_2,\otimes_2,\alpha_2,\lambda_2,\rho_2,\beta_2)$.  Then a
  \textbf{symmetric monoidal functor} is a functor $F : \cat{M}_1 \mto
  \cat{M}_2$, a map $m_{\top_1} : \top_2 \mto F\top_1$ and a natural
  transformation $m_{A,B} : FA \otimes_2 FB \mto F(A \otimes_1 B)$ subject
  to the following coherence conditions:
  \begin{mathpar}
    \bfig
    \vSquares|ammmmma|/->`->`->``->`->`->/[
      (FA \otimes_2 FB) \otimes_2 FC`
      FA \otimes_2 (FB \otimes_2 FC)`
      F(A \otimes_1 B) \otimes_2 FC`
      FA \otimes_2 F(B \otimes_1 C)`
      F((A \otimes_1 B) \otimes_1 C)`
      F(A \otimes_1 (B \otimes_1 C));
      {\alpha_2}_{FA,FB,FC}`
      m_{A,B} \otimes \id_{FC}`
      \id_{FA} \otimes m_{B,C}``
      m_{A \otimes_1 B,C}`
      m_{A,B \otimes_1 C}`
      F{\alpha_1}_{A,B,C}]
    \efig
    \end{mathpar}
%    \and
\begin{mathpar}
    \bfig
    \square/->`->`<-`->/<1000,500>[
      \top_2 \otimes_2 FA`
      FA`
      F\top_1 \otimes_2 FA`
      F(\top_1 \otimes_1 A);
      {\lambda_2}_{FA}`
      m_{\top_1} \otimes \id_{FA}`
      F{\lambda_1}_{A}`
      m_{\top_1,A}]
    \efig
    \and
    \bfig
    \square/->`->`<-`->/<1000,500>[
      FA \otimes_2 \top_2`
      FA`
      FA \otimes_2 F\top_1`
      F(A \otimes_1 \top_1);
      {\rho_2}_{FA}`
      \id_{FA} \otimes m_{\top_1}`
      F{\rho_1}_{A}`
      m_{A,\top_1}]
    \efig
     \end{mathpar}
     
      \begin{mathpar}
    \bfig
    \square/->`->`->`->/<1000,500>[
      FA \otimes_2 FB`
      FB \otimes_2 FA`
      F(A \otimes_1 B)`
      F(B \otimes_1 A);
      {\beta_2}_{FA,FB}`
      m_{A,B}`
      m_{B,A}`
      F{\beta_1}_{A,B}]
    \efig
  \end{mathpar}
\end{definition}

\begin{definition}
  \label{def:MCNAT}
  Suppose $(\cat{M}_1,\top_1,\otimes_1)$ and $(\cat{M}_2,\top_2,\otimes_2)$
  are monoidal categories, and $(F,m)$ and $(G,n)$ are monoidal functors
  between $\cat{M}_1$ and $\cat{M}_2$.  Then a \textbf{
    monoidal natural transformation} is a natural transformation,
  $f : F \mto G$, subject to the following coherence diagrams:
  \begin{mathpar}
    \bfig
    \square<1000,500>[
      FA \otimes_2 FB`
      F(A \otimes_1 B)`
      GA \otimes_2 GB`
      G(A \otimes_1 B);
      m_{A,B}`
      f_A \otimes_2 f_B`
      f_{A \otimes_1 B}`
      n_{A,B}]
    \efig
    \and
    \bfig
    \Vtriangle/->`<-`<-/[
      F\top_1`
      G\top_1`
      \top_2;
      f_{\top_1}`
      m_{\top_1}`
      n_{\top_1}]
    \efig
  \end{mathpar}  
\end{definition}

\begin{definition}
  \label{def:SMCNAT}
  Suppose $(\cat{M}_1,\top_1,\otimes_1)$ and $(\cat{M}_2,\top_2,\otimes_2)$
  are SMCs, and $(F,m)$ and $(G,n)$ are symmetric monoidal functors
  between $\cat{M}_1$ and $\cat{M}_2$.  Then a \textbf{symmetric
    monoidal natural transformation} is a natural transformation,
  $f : F \mto G$, subject to the following coherence diagrams:
  \begin{mathpar}
    \bfig
    \square<1000,500>[
      FA \otimes_2 FB`
      F(A \otimes_1 B)`
      GA \otimes_2 GB`
      G(A \otimes_1 B);
      m_{A,B}`
      f_A \otimes_2 f_B`
      f_{A \otimes_1 B}`
      n_{A,B}]
    \efig
    \and
    \bfig
    \Vtriangle/->`<-`<-/[
      F\top_1`
      G\top_1`
      \top_2;
      f_{\top_1}`
      m_{\top_1}`
      n_{\top_1}]
    \efig
  \end{mathpar}  
\end{definition}

\begin{definition}
  \label{def:MCADJ}
  Suppose $(\cat{M}_1,\top_1,\otimes_1)$ and $(\cat{M}_2,\top_2,\otimes_2)$
  are monoidal categories, and $(F,m)$ is a monoidal functor between
  $\cat{M}_1$ and $\cat{M}_2$ and $(G,n)$ is a monoidal
  functor between $\cat{M}_2$ and $\cat{M}_1$.  Then a
  \textbf{monoidal adjunction} is an ordinary adjunction
  $\cat{M}_1 : F \dashv G : \cat{M}_2$ such that the unit,
  $\eta_A : A \to GFA$, and the counit, $\varepsilon_A : FGA \to A$, are
  monoidal natural transformations.  Thus, the following
  diagrams must commute:
  \begin{mathpar}
    \bfig
    \square/->`->`->`<-/<1000,500>[
      FGA \otimes_2 FGB`
      F(GA \otimes_1 GB)`
      A \otimes_2 B`
      FGA \otimes_2 FGB;
      m_{GA,GB}`
      \varepsilon_A \otimes_1 \varepsilon_B`
      Fn_{A,B}`
      \varepsilon_{A \otimes_1 B}]
    \efig
    \and
    \bfig
    %% \Vtriangle|amm|/->`<-`=/[
    %%   FG\top_1`
    %%   \top_1`
    %%   \top_1;
    %%   \varepsilon_{\top_1}`
    %%   \q{\top_1}`]
    \square/->`<-`->`=/<1000,500>[
      F\top_1`
      FG\top_2`
      \top_2`
      \top_2;
      Fn_{\top_2}`
      m_{\top_1}`
      \varepsilon_{\top_1}`]    
    \efig
    \and
    \bfig
    %% \dtriangle|mmb|<1000,500>[
    %%   A \otimes_2 B`
    %%   GFA \otimes_2 GFB`
    %%   GF(A \otimes_2 B);
    %%   \eta_A \otimes_2 \eta_B`
    %%   \eta_{A \otimes_2 B}`
    %%   \p{A,B}]
    \square/<-`->`->`->/<1000,500>[
      GFA \otimes_1 GFB`
      A \otimes_1 B`
      G(FA \otimes_2 FB)`
      GF(A \otimes_1 B);
      \eta_A \otimes_1 \eta_B`
      n_{FA,FB}`
      \eta_{A \otimes_1 B}`
      m_{A,B}]
    \efig
    \and
    \bfig
    %% \Vtriangle|amm|/->`=`<-/[
    %%   \top_1`
    %%   GF\top_1`
    %%   \top_1;
    %%   \eta_{\top_1}``
    %%   p_{\top_1}]
    \square/->`<-`<-`=/<1000,500>[
      G\top_2`
      GF\top_1`
      \top_1`
      \top_1;
      Gm_{\top_1}`
      n_{\top_2}`
      \eta_{\top_1}`]      
    \efig
  \end{mathpar} 
\end{definition}

\begin{definition}
  \label{def:SMCADJ}
  Suppose $(\cat{M}_1,\top_1,\otimes_1)$ and $(\cat{M}_2,\top_2,\otimes_2)$
  are SMCs, and $(F,m)$ is a symmetric monoidal functor between
  $\cat{M}_1$ and $\cat{M}_2$ and $(G,n)$ is a symmetric monoidal
  functor between $\cat{M}_2$ and $\cat{M}_1$.  Then a
  \textbf{symmetric monoidal adjunction} is an ordinary adjunction
  $\cat{M}_1 : F \dashv G : \cat{M}_2$ such that the unit,
  $\eta_A : A \to GFA$, and the counit, $\varepsilon_A : FGA \to A$, are
  symmetric monoidal natural transformations.  Thus, the following
  diagrams must commute:
  \begin{mathpar}
    \bfig
    \square/->`->`->`<-/<1000,500>[
      FGA \otimes_2 FGB`
      F(GA \otimes_1 GB)`
      A \otimes_2 B`
      FGA \otimes_2 FGB;
      m_{GA,GB}`
      \varepsilon_A \otimes_1 \varepsilon_B`
      Fn_{A,B}`
      \varepsilon_{A \otimes_1 B}]
    \efig
    \and
    \bfig
    %% \Vtriangle|amm|/->`<-`=/[
    %%   FG\top_1`
    %%   \top_1`
    %%   \top_1;
    %%   \varepsilon_{\top_1}`
    %%   \q{\top_1}`]
    \square/->`<-`->`=/<1000,500>[
      F\top_1`
      FG\top_2`
      \top_2`
      \top_2;
      Fn_{\top_2}`
      m_{\top_1}`
      \varepsilon_{\top_1}`]    
    \efig
    \and
    \bfig
    %% \dtriangle|mmb|<1000,500>[
    %%   A \otimes_2 B`
    %%   GFA \otimes_2 GFB`
    %%   GF(A \otimes_2 B);
    %%   \eta_A \otimes_2 \eta_B`
    %%   \eta_{A \otimes_2 B}`
    %%   \p{A,B}]
    \square/<-`->`->`->/<1000,500>[
      GFA \otimes_1 GFB`
      A \otimes_1 B`
      G(FA \otimes_2 FB)`
      GF(A \otimes_1 B);
      \eta_A \otimes_1 \eta_B`
      n_{FA,FB}`
      \eta_{A \otimes_1 B}`
      m_{A,B}]
    \efig
    \and
    \bfig
    %% \Vtriangle|amm|/->`=`<-/[
    %%   \top_1`
    %%   GF\top_1`
    %%   \top_1;
    %%   \eta_{\top_1}``
    %%   p_{\top_1}]
    \square/->`<-`<-`=/<1000,500>[
      G\top_2`
      GF\top_1`
      \top_1`
      \top_1;
      Gm_{\top_1}`
      n_{\top_2}`
      \eta_{\top_1}`]      
    \efig
  \end{mathpar} 
\end{definition}

\begin{definition}
  \label{def:monoidal-comonad}
  A \textbf{monoidal comonad} on a monoidal
  category $\cat{C}$ is a triple $(T,\varepsilon, \delta)$, where
  $(T,\m{})$ is a monoidal endofunctor on $\cat{C}$,
  $\varepsilon_A : TA \mto A$ and $\delta_A : TA \to T^2 A$ are
  monoidal natural transformations, which make the following
  diagrams commute:
  \begin{mathpar}
    \bfig
    \square<600,600>[
      TA`
      T^2A`
      T^2A`
      T^3A;
      \delta_A`
      \delta_A`
      T\delta_A`
      \delta_{TA}]
    \efig
    \and
    \bfig
    \Atrianglepair/=`->`=`<-`->/<600,600>[
      TA`
      TA`
      T^2 A`
      TA;`
      \delta_A``
      \varepsilon_{TA}`
      T\varepsilon_A]
    \efig
  \end{mathpar}
  The assumption that $\varepsilon$ and $\delta$ are 
  monoidal natural transformations amount to the following diagrams
  commuting:
  \begin{mathpar}
    \bfig
    \qtriangle/->`->`->/<1000,600>[
      TA \otimes TB`
      T(A \otimes B)`
      A \otimes B;
      \m{A,B}`
      \varepsilon_A \otimes \varepsilon_B`
    \varepsilon_{A \otimes B}]
    \efig
    \and
    \bfig
    \Vtriangle/<-`->`=/<600,600>[
      T\top`
      \top`
      \top;
      \m{\top}`
      \varepsilon_\top`]
    \efig    
  \end{mathpar}
  \begin{mathpar}
    \bfig
    \square|alab|/`->``->/<1050,600>[
      TA \otimes TB``
      T^2A \otimes T^2B`
      T(TA \otimes TB);`
      \delta_A \otimes \delta_B``
      \m{TA,TB}]
    \square(1050,0)|mmrb|/``->`->/<1050,600>[`
      T(A \otimes B)`
      T(TA \otimes TB)`
      T^2(A \otimes B);``
      \delta_{A \otimes B}`
      T\m{A,B}]
    \morphism(0,600)<2100,0>[TA \otimes TB`T(A \otimes B);\m{A,B}]
    \efig
    \and
    \bfig
    \square<600,600>[
      \top`
      T\top`
      T\top`
      T^2\top;
      \m{\top}`
      \m{\top}`
      \delta_\top`
      T\m{\top}]
    \efig
  \end{mathpar}
\end{definition}


\begin{definition}
  \label{def:symm-monoidal-comonad}
  A \textbf{symmetric monoidal comonad} on a symmetric monoidal
  category $\cat{C}$ is a triple $(T,\varepsilon, \delta)$, where
  $(T,\m{})$ is a symmetric monoidal endofunctor on $\cat{C}$,
  $\varepsilon_A : TA \mto A$ and $\delta_A : TA \to T^2 A$ are
  symmetric monoidal natural transformations, which make the following
  diagrams commute:
  \begin{mathpar}
    \bfig
    \square<600,600>[
      TA`
      T^2A`
      T^2A`
      T^3A;
      \delta_A`
      \delta_A`
      T\delta_A`
      \delta_{TA}]
    \efig
    \and
    \bfig
    \Atrianglepair/=`->`=`<-`->/<600,600>[
      TA`
      TA`
      T^2 A`
      TA;`
      \delta_A``
      \varepsilon_{TA}`
      T\varepsilon_A]
    \efig
  \end{mathpar}
  The assumption that $\varepsilon$ and $\delta$ are symmetric
  monoidal natural transformations amount to the following diagrams
  commuting:
  \begin{mathpar}
    \bfig
    \qtriangle/->`->`->/<1000,600>[
      TA \otimes TB`
      T(A \otimes B)`
      A \otimes B;
      \m{A,B}`
      \varepsilon_A \otimes \varepsilon_B`
    \varepsilon_{A \otimes B}]
    \efig
    \and
    \bfig
    \Vtriangle/<-`->`=/<600,600>[
      T\top`
      \top`
      \top;
      \m{\top}`
      \varepsilon_\top`]
    \efig    
  \end{mathpar}
  \begin{mathpar}
    \bfig
    \square|alab|/`->``->/<1050,600>[
      TA \otimes TB``
      T^2A \otimes T^2B`
      T(TA \otimes TB);`
      \delta_A \otimes \delta_B``
      \m{TA,TB}]
    \square(1050,0)|mmrb|/``->`->/<1050,600>[`
      T(A \otimes B)`
      T(TA \otimes TB)`
      T^2(A \otimes B);``
      \delta_{A \otimes B}`
      T\m{A,B}]
    \morphism(0,600)<2100,0>[TA \otimes TB`T(A \otimes B);\m{A,B}]
    \efig
    \and
    \bfig
    \square<600,600>[
      \top`
      T\top`
      T\top`
      T^2\top;
      \m{\top}`
      \m{\top}`
      \delta_\top`
      T\m{\top}]
    \efig
  \end{mathpar}
\end{definition}

\section{Proofs}
\label{sec:proofs}
\subsection{Proof of Composition of Weakening and Contraction (Lemma~\ref{lem:compose-cw})}
\label{subsec:proof_of_composition_of_weakening_and_contraction_lem:compose-cw}
Since by definition $w:\cat{L} \mto \cat{L}$ and $c:\cat{L} \mto
\cat{L}$ are monoidal functors we know that their composition
$cw:\cat{L} \mto \cat{L}$ is a monoidal functor:
\[
\begin{array}{ll}
  \q{A,B} : cwA\otimes cwB\mto cw(A\otimes B)   \\
  \q{A,B} = c\q{A,B}^w\circ\q{wA,wB}^c        \\
  \q{I} : I\mto cwI                             \\
  \q{I} = c\q{I}^w\circ\q{I}^c
\end{array}
\]

We must now define both $\varepsilon_A:cwA\mto A$ and
$\delta_A:cwA\mto cwcwA$, and then show that they are monoidal
natural transformations subject to the comonad laws. Since we are
composing two comonads each of $\varepsilon$ and $\delta$ can be
given two definitions, but they are equivalent:
\begin{itemize}
\item $\varepsilon_A:cwA\mto A$ is defined as in the diagram
  below, which commutes by the naturality of $\varepsilon^c$.
  \begin{mathpar}
    \bfig
    \square(1050,0)/->`->`->`->/<1050,600>[
      cwA`
      wA`
      cA`
      A;
      \varepsilon_{wA}^c`
      c\varepsilon_A^w`
      \varepsilon_A^w`
      \varepsilon_A^c]
    \efig
  \end{mathpar}

\item $\delta_A:cwA\mto cwcwA$ is defined as in the diagram:
  \begin{mathpar}
    \bfig
    \square|almb|/->`->`->`->/<1050,600>[
      cwA`
      cw^2A`
      c^2wA`
      c^2w^2A;
      c\delta_A^w`
      \delta_{wA}^c`
      \delta_{w^2A}^c`
      c^2\delta_A^w]
    \square(1050,0)|amrb|/->``->`->/<1050,600>[
      cw^2A`
      c^2w^2A`
      c^2w^2A`
      cwcwA;
      \delta_{w^2A}^c``
      cdist_{wA}`
      cdist_{wA}]
    \efig
  \end{mathpar}
  The left part of the diagram commutes by the naturality
  of $\delta^c$ and the right part commutes trivially.
\end{itemize}

The remainder of the proof shows that the comonad laws hold.

\begin{itemize}
\item[] \textbf{Case 1:}
  \begin{mathpar}
    \bfig
    \square/->`->`->`->/<1050,600>[
      cwA`
      cwcwA`
      cwcwA`
      cwcwcwA;
      \delta_A`
      \delta_A`
      cw\delta_A`
      \delta_{cwA}]
    \efig
  \end{mathpar}

  The previous diagram commutes because the following one does.

  \begin{mathpar}
    \bfig
    \ptriangle/->`->`=/<700,400>[
      cwA`
      cwcwA`
      cwcwA;
      \delta_A`
      \delta_A`]
    \square(700,0)|amm|/->`->`->`/<900,400>[
      cwcwA`
      cwcw^2A`
      c^2wcwA`
      c^2wcw^2A;
      cwc\delta_A^w`
      \delta_{wcwA}^c`
      \delta_{wcw^2A}^c`]
    \ptriangle(1600,0)|amm|/->``<-/<1100,400>[
      cwcw^2A`
      cwc^2w^2A`
      c^2wcw^2A;
      cw\delta_{w^2A}^c`
      `
      cdist_{cw^2A}]
    \qtriangle(700,-600)|mmm|/->`->`->/<900,600>[
      c^2wcwA`
      c^2wcw^2A`
      c^2w^2cwA;
      c^2wc\delta_A^w`
      c^2\delta_{cwA}^w`
      c^2wdist_{wA}]
    \btriangle(0,-600)/->``->/<1600,600>[
      cwcwA`
      cw^2cwA`
      c^2w^2cwA;
      c\delta_{cwA}^w`
      `
      \delta_{w^2cwA}^c]
    \dtriangle(1600,-600)/`->`->/<1100,1000>[
      cwc^2w^2A`
      c^2w^2cwA`
      cwcwcwA;
      `
      cwcdist_{wA}`
      cdist_{wcwA}]
    % To show texts in each subdiagram:
    \ptriangle(200,-150)/``/<400,400>[(1)``;``]
    \ptriangle(1000,-200)/``/<400,400>[(2)``;``]
    \ptriangle(1300,-600)/``/<400,400>[(3)``;``]
    \ptriangle(500,-700)/``/<400,400>[(4)``;``]
    \ptriangle(1850,-150)/``/<400,400>[(5)``;``]
    \ptriangle(2100,-700)/``/<400,400>[(6)``;``]
    \efig
  \end{mathpar}

  (1) commutes by equality and we will not expand $\delta_A$ for
  simplicity. (2) and (4) commutes by the naturality of $\delta^c$. (3),
  (5) commutes by the conditions of $dist$. (6) commutes by the naturality of
  $dist$.

\item[] \textbf{Case 2}:
  \begin{mathpar}
    \bfig
    \qtriangle/->`=`->/<600,600>[
      cwA`
      cwcwA`
      cwA;
      \delta_A``
      cw\varepsilon_A]
    \efig
  \end{mathpar}

  The triangle commutes because of the following diagram chasing.

  \begin{mathpar}
    \bfig
    \qtriangle|amm|/->`<-`=/<1200,600>[
      cwA`
      cw^2A`
      cw^2A;
      c\delta_A^w`
      c\varepsilon_{wA}^w`]
    \ptriangle(1200,0)|amm|/->``->/<600,600>[
      cw^2A`
      c^2w^2A`
      cw^2A;
      \delta_{w^2A}^c``
      c\varepsilon_{w^2A}^c]
    \btriangle(0,-1200)/=``<-/<1200,1800>[
      cwA`
      cwA`
      wcwA;
      ``\varepsilon_{cwA}^w]
    \btriangle(1200,-1200)|mmb|/`<-`<-/<600,1200>[
      cw^2A`
      wcwA`
      cwcwA;
      `
      cw\varepsilon_{wA}^c`
      \varepsilon_{wcwA}^c]
    \dtriangle(600,-600)|mmm|/`->`<-/<600,600>[
      cw^2A`
      wA`
      w^2A;
      `
      \varepsilon_{w^2A}^c`
      \varepsilon_{wA}^w]
    \morphism(0,600)|m|<600,-1200>[cwA`wA;\varepsilon_{wA}^c]
    \morphism(0,-1200)|m|<600,600>[cwA`wA;\varepsilon_{wA}^c]
    \morphism(1200,-1200)|m|<0,600>[wcwA`w^2A;w\varepsilon_{wA}^c]
    \morphism(1800,600)|r|<0,-1800>[c^2w^2A`cwcwA;cdist_{wA}]
    \ptriangle(800,300)/``/<100,100>[(1)``;``]
    \ptriangle(1350,300)/``/<100,100>[(2)``;``]
    \ptriangle(700,-300)/``/<100,100>[(3)``;``]
    \ptriangle(1600,-300)/``/<100,100>[(4)``;``]
    \ptriangle(300,-700)/``/<100,100>[(5)``;``]
    \ptriangle(700,-1000)/``/<100,100>[(6)``;``]
    \ptriangle(1450,-1000)/``/<100,100>[(7)``;``]
    \efig
  \end{mathpar}
  (1) commutes by the comonad law for $w$ with components $\delta_A^w$
  and $\varepsilon_{wA}^w$. (2) commutes by the comonad law for $c$ with
  components $\delta_{w^2A}^c$ and $\varepsilon_{w^2A}^c$. (3) and (7)
  commute by the naturality of $\varepsilon^c$. (4) commutes by the condition
  of $dist$. (5) commutes trivially. And (6) commutes by the naturality of
  $\varepsilon^w$.
  
\item[] \textbf{Case 3}:
  \begin{mathpar}
    \bfig
    \btriangle/->`=`->/<600,600>[
      cwA`
      cwcwA`
      cwA;
      \delta_A``
      \varepsilon_{cwA}]
    \efig
  \end{mathpar}

  The previous triangle commutes because the following diagram chasing
  does.

  \begin{mathpar}
    \bfig
    \qtriangle|amm|/->`->`/<800,400>[
      cwA`
      cw^2A`
      c^2wA;
      c\delta_A^w`
      \delta_{wA}^c`]
    \morphism(0,400)|m|/<-/<800,-800>[cwA`c^2wA;c\varepsilon_{wA}^c]
    \ptriangle(800,0)|amm|/->``<-/<800,400>[
      cw^2A`
      c^2w^2A`
      c^2wA;
      \delta_{w^2A}^c``
      c^2\delta_A^w]
    \morphism(800,-400)|m|/<-/<800,800>[c^2wA`c^2w^2A;c^2w\varepsilon_A^w]
    \morphism(800,0)/=/<0,-400>[c^2wA`c^2wA;]
    \btriangle(0,-800)/=``<-/<800,1200>[
      cwA`
      cwA`
      cwcA;
      ``
      cw\varepsilon_A^c]
    \dtriangle(800,-800)/`->`<-/<800,1200>[
      c^2w^2A`
      cwcA`
      cwcwA;
      `
      cdist_{wA}`
      cwc\varepsilon_A^w]
    \morphism(800,-400)|m|/->/<0,-400>[c^2wA`cwcA;cdist_A]
    \ptriangle(800,100)/``/<100,100>[(1)``;``]
    \ptriangle(600,-100)/``/<100,100>[(2)``;``]
    \ptriangle(1000,-100)/``/<100,100>[(3)``;``]
    \ptriangle(400,-600)/``/<100,100>[(4)``;``]
    \ptriangle(1200,-600)/``/<100,100>[(5)``;``]
    \efig
  \end{mathpar}

  (1) commutes by the naturality of $\delta^c$. (2) is the comonad law
  for $c$ with components $\delta_{wA}^c$ and $\varepsilon_{wA}^c$. (3)
  is the comonad law for $w$ with components $\delta_A^w$ and
  $\varepsilon_A^w$. (4) commutes by the condition of $dist$. And (5)
  commute by the naturality of $dist$.

\end{itemize}
% subsection proof_of_composition_of_weakening_and_contraction_(lemma~\ref{lem:compose-cw}) (end)
% section proofs (end)
