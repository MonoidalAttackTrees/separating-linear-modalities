Now we turn to making our model precise.
\begin{definition}
  \label{def:model}  
  Suppose $\cat{L}_0, \ldots, \cat{L}_n$ is a family of Lambek
  categories with zero or more structural morphisms where $\cat{L}_i$
  is a full subcategory of $\cat{L}_{i+1}$.  Then a \textbf{structural
    adjoint model}, \\ $\overrightarrow{\cat{L}_n : F_n \dashv G_n : \cat{L}_{n-1}}$, is a composition of monoidal adjunctions:
  \[ \cat{L}_n : F_n \dashv G_n : \cat{L}_{n-1} : F_{n-1} \dashv G_{n-1} : \cat{L}_{n-2} : \cdots : \cat{L}_1 : F \dashv G : \cat{L}_0. \]
\end{definition}
This definition is an extension -- or perhaps a simplification due to
the isolation of exchange -- of the models discussed by
Melli\'es~\cite{MELLIES2004202}.

Our structural adjoint model subsumes Benton's~\cite{Benton:1994}
linear/non-linear model (LNL).  Simply take the sequence of Lambek
categories to be $\cat{L}_0$, a Lambek category with exchange, and
$\cat{L}_1$, a Lambek category with weakening, contraction, and
exchange, and thus, $\cat{L}_1$ is cartesian closed.  However, our
model is a lot more flexible and expressive.

\begin{lemma}[Comonads in a Structural Adjoint Model]
  \label{lemma:comonads_in_a_structural_adjoint_model}
  Suppose $\overrightarrow{\cat{L}_n : F_n \dashv G_n : \cat{L}_{n-1}}$ is a structural adjoint model.  Then
  there are the following comonads:  
  \begin{itemize}
  \item $(\cat{L}_0, F_0G_0, \varepsilon^0, \delta^0), \ldots, (\cat{L}_{n-1}, F_nG_n, \varepsilon^{n}, \delta^{n})$
  \item $(\cat{L}_0, F_0F_1G_1G_0, \varepsilon^0, \delta^0), \ldots, (\cat{L}_{n-1}, F_{n-1}F_nG_nG_{n-1}, \varepsilon^{n}, \delta^{n})$
  \item $(\cat{L}_0, F_0F_1F_2G_2G_1G_0, \varepsilon^0, \delta^0), \ldots, (\cat{L}_{n-1}, F_{n-2}F_{n-1}F_nG_nG_{n-1}G_{n-2}, \varepsilon^{n}, \delta^{n})$\\
    $\vdots$
  \item $(\cat{L}_0, F_0 \cdots F_nG_n \cdots G_0, \varepsilon^0, \delta^0)$
  \end{itemize} 
\end{lemma}
\begin{proof}
  This proof easily follows from the well-known fact that adjoints
  induce comonads -- as well as monads -- and composition of adjoints.
\end{proof}
The previous lemma shows that a Lambek category $\cat{L}_i$ in the
sequence is endowed with all of the structure found in each of the
categories above it, but this structure is explicitly tracked using
the various comonads.  That is, the Eilenberg-Moore category of each of the
comonads mentioned in the previous lemma has the corresponding
structural rule as morphisms.

\begin{lemma}
  \label{lemma:kleisli_category_gen}
  Suppose $\cat{L}_l : F \dashv G : \cat{L}_{r}$ is a monoidal
  adjunction in the structural adjoint model
  $\overrightarrow{\cat{L}_n : F_n \dashv G_n : \cat{L}_{n-1}}$.  Then
  the Eilenberg-Moore category, $\cat{L}^E_{r}$, contains all of the
  structural morphisms from both $\cat{L}_{r}$ and $\cat{L}_{l}$.
\end{lemma}
\begin{proof}
  This result holds similarly to Benton's~\cite{Benton:1994} proof
  that the Eilenberg-Moore category for the of-course comoand is
  cartesian closed.  So we omit the details.
\end{proof}

\subsection{Example Structural Adjoint Models}
\label{subsec:example_structural_adjoint_models}

We can give a number of example adjoint structures that are of
interest to the research community.

\textbf{Exchange.}  The first is a model that reveals how to combine
the Lambek Calculus with the exchange comonad and Girard's of-course
comonad.  This model is of interest to the linguistics community
\cite{?}, because they often only want exchange in very controlled
instances.  Valeria de Paiva~\cite{?} was the first to show that this
is possible using Dialectica Categories and Reedy's~\cite{?}  model.
However, she uses a comonad with the natural transformation $\e{A,B} :
A \seq eB \mto eB \seq A$, but we feel this goes against the standard
view of algebraic binary operations.  In addition, while Dialectica
categories are extremely useful, but rather complex, we are interested
in simpler models.  Thus, we prefer an adjoint model with a comonad
which has the natural transformation $\e{A,B} : eA \seq eB \mto eB
\seq eA$.

As we have said in the introduction there are many security
applications where one must have both a commutative and a
non-commutative tensor product within the same logic.  For example,
when reasoning about process trees in threat analysis.

\begin{definition}
  \label{def:LC-adjoint-structure}
  Suppose $\cat{L}_{ewc}$ is a Lambek category with exchange,
  weakening, and contraction, $\cat{L}_e$ is a Lambek category with
  exchange, and $\cat{L}$ is a Lambek category.  Then a \textbf{LC
    adjoint model} is the structural adjoint model
  $\cat{L}_{ewc} : H \dashv J : \cat{L}_e : F \dashv G : \cat{L}$.
\end{definition}

We must now show that $\cat{L}$ in a LC adjoint model has two comonads
$e : \cat{L} \mto \cat{L}$ adding exchange to $\cat{L}$, and $! :
\cat{L} \mto \cat{L}$ -- Girard's of-course modality -- adding
weakening, contraction, and exchange.  We first have the following
corollary to Lemma~\ref{lemma:comonads_in_a_structural_adjoint_model}.

\begin{corollary}
  \label{corollary:LC-comonads}
  Suppose $\cat{L}_{ewc} : H \dashv J : \cat{L}_e : F \dashv G : \cat{L}$ is a LC adjoint model.
  Then there are comonads 
  $(e : \cat{L} \mto \cat{L}, \varepsilon^e, \delta^e)$, $(! : \cat{L} \mto \cat{L}, \varepsilon^!, \delta^!)$,
  and $(!_e : \cat{L}_e \mto \cat{L}_e, \varepsilon^{!_e}, \delta^{!_e})$.
\end{corollary}
\begin{proof}
  We only show how each of the tuples are defined:
  \begin{itemize}
  \item $eA = FGA$, $\varepsilon^{e}_A : eA \mto A$ is the
    counit of the adjunction, and $\delta^{e}_A = F\eta^e_{GA} : eA
    \mto eeA$, where $\eta^e_A : A \mto GFA$ is the unit of the
    adjunction.
    
  \item $!_eA = HJA$, $\varepsilon^{!_e}_A : !_eA \mto A$ is the
    counit of the adjunction, and $\delta^{!_e}_A = H\eta^{!_e}_{JA} : !_e A
    \mto !_e!_eA$, where $\eta^{!_e}_A : A \mto JHA$ is the unit of the
    adjunction.  

  \item $!A = FHJGA$, $\varepsilon^{!}_A : !A \mto A$ is the
    counit of the adjunction, and $\delta^{!}_A = FH\eta^{!}_{JGA} : ! A
    \mto !!A$, where $\eta^!_A : A \mto JGFHA$ is the unit of the
    adjunction.
  \end{itemize}  
\end{proof}

\noindent
As a corollary to Lemma~\ref{lemma:kleisli_category_gen} we show that
the Eilenberg-Moore categories contain the required structure.
\begin{corollary}
  \label{corollary:EM-LC-adjoint-model}
  Suppose $\cat{L}_{ewc} : H \dashv J : \cat{L}_e : F \dashv G : \cat{L}$ is a LC adjoint model. Then
  the Eilenberg-Moore Categories associated with the comonads
  $(e : \cat{L} \mto \cat{L}, \varepsilon^e, \delta^e)$, $(! : \cat{L} \mto \cat{L}, \varepsilon^!, \delta^!)$,
  and $(!_e : \cat{L}_e \mto \cat{L}_e, \varepsilon^{!_e}, \delta^{!_e})$
  have the structure:
  $\cat{L}^E_e$ is symmetric monoidal, and
  $\cat{L}^E_!$ and $\cat{L}^E_{!_e}$ are cartesian closed.
\end{corollary}
\begin{proof}
  The proof that $\cat{L}^E_{!_e}$ is cartesian closed follows from
  Benton~\cite{Benton:1994}, because $\cat{L}_e$ is symmetric
  monoidal, and $\cat{L}_{ewc}$ is cartesian closed, and hence
  $\cat{L}_{ewc} : H \dashv J : \cat{L}_e$ is a LNL model.  So we only
  give proofs of the other two categories.
  
  TODO: Jiaming
\end{proof}

Prove the Eilenberg-Moore category is symmetric monoidal.

\textbf{Contraction.}

Prove the Klesli category has contractions.

\textbf{Contraction and Exchange.}

Prove the Klesli category has weakening.

% subsection example_structural_adjoint_models (end)

%%% Local Variables: 
%%% mode: latex
%%% TeX-master: main.tex
%%% End: 
