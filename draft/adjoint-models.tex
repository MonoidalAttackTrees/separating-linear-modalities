Now we turn to making our model precise.
\begin{definition}
  \label{def:model}  
  Suppose $\cat{L}_0, \ldots, \cat{L}_n$ is a family of Lambek
  categories with zero or more structural morphisms where $\cat{L}_0$
  is a full subcategory of each $\cat{L}_{i}$ for $0 < i \leq n$.  Then a \textbf{composition of structure},
  $\overleftarrow{\cat{L}_n : F_{n-1} \dashv G_{n-1} : \cat{L}_{n-1}}$, is a composition of monoidal adjunctions:
  \[ \cat{L}_n : F_{n-1} \dashv G_{n-1} : \cat{L}_{n-1} : F_{n-2} \dashv G_{n-2} : \cat{L}_{n-2} : \cdots : \cat{L}_1 : F_0 \dashv G_0 : \cat{L}_0. \]
  We call $\cat{L}_0$ the base of the composition.
\end{definition}
This definition is an extension -- or perhaps a simplification due to
the isolation of exchange -- of the models discussed by
Melli\'es~\cite{MELLIES2004202}.

Our composition of structure subsumes Benton's~\cite{Benton:1994}
linear/non-linear model (LNL).  Simply take the sequence of Lambek
categories to be $\cat{L}_0$, a Lambek category with exchange, and
$\cat{L}_1$, a Lambek category with weakening, contraction, and
exchange, and thus, $\cat{L}_1$ is cartesian closed.  However, our
model is a lot more flexible and expressive.

\begin{lemma}[Comonads in a Composition of Structure]
  \label{lemma:comonads_in_a_composition_of_structure}
  Suppose $\overleftarrow{\cat{L}_n : F_{n-1} \dashv G_{n-1} : \cat{L}_{n-1}}$ is a composition of structure.  Then
  there are the following comonads:  
  \begin{itemize}
  \item $(\cat{L}_0, F_0G_0, \varepsilon^0_0, \delta^0_0), \ldots, (\cat{L}_{n-1}, F_nG_n, \varepsilon^0_{n-1}, \delta^0_{n-1})$
  \item $(\cat{L}_0, F_0F_1G_1G_0, \varepsilon^1_0, \delta^1_0), \ldots, (\cat{L}_{n-1}, F_{n-1}F_nG_nG_{n-1}, \varepsilon^1_{n-1}, \delta^1_{n-1})$
  \item $(\cat{L}_0, F_0F_1F_2G_2G_1G_0, \varepsilon^2_0, \delta^2_0), \ldots, (\cat{L}_{n-1}, F_{n-2}F_{n-1}F_nG_nG_{n-1}G_{n-2}, \varepsilon^2_{n-1}, \delta^2_{n-1})$\\
    $\vdots$
  \item $(\cat{L}_0, F_0 \cdots F_nG_n \cdots G_0, \varepsilon^n_0, \delta^n_0)$
  \end{itemize} 
\end{lemma}
\begin{proof}
  This proof easily follows from the well-known fact that adjoints
  induce comonads -- as well as monads -- and composition of adjoints.
\end{proof}
The previous lemma shows that a Lambek category $\cat{L}_i$ in the
sequence is endowed with all of the structure found in each of the
categories above it, but this structure is explicitly tracked using
the various comonads.  That is, the Eilenberg-Moore category of each of the
comonads mentioned in the previous lemma has the corresponding
structural rule as morphisms.

\begin{lemma}
  \label{lemma:kleisli_category_gen}
  Suppose $\overleftarrow{\cat{L}_n : F_{n-1} \dashv G_{n-1} :
    \cat{L}_{n-1}}$ is a composition of structure containing
  the adjunction
  $\cat{L}_j : F_{j-1} \dashv G_{j-1} : \cat{L}_{j-1} : \cdots : \cat{L}_{i+1} : F_i \dashv G_i : \cat{L}_i$
  for $0 \leq i < n$ and $0 < j \leq n$.  If
  $(\cat{L}_i, M, \varepsilon, \delta)$ is the comonad
  defined by $M A = F_i\cdots F_{j-1}G_{j-1} \cdots G_iA $, then
  the Eilenberg-Moore category, $\cat{L}^E_M$, contains every structural morphism
  in the categories $\cat{L}_i,\ldots,\cat{L}_j$.
\end{lemma}
\begin{proof}
  This result holds similarly to Benton's~\cite{Benton:1994} proof
  that the Eilenberg-Moore category for the of-course comoand is
  cartesian closed.  So we omit the details.
\end{proof}

Multiple composition of structure can have the same Eilenberg-Moore
category associated with the comonad induced by the model.  For
example, the comonad associated with $\cat{L}_e : F \dashv G :
\cat{L}_{c}$ is the exchange comonad given below, and hence, its
Eilenberg-Moore category contains both contraction and exchange, which
is a model of strict logic. However, $\cat{L}_c : F \dashv G :
\cat{L}_{e}$ induces the contraction comonad, but its Eilenberg-Moore
category is the same.  The difference between these two models is
whats being explicitly tracked.  In the first model contraction is a
first-class citizen, but exchange is being tracked by the comonad,
while in the second example exchange is a first-class citizen and
contraction is being explicitly tracked by the comonad.

\subsection{Example Compositions of Structure}
\label{subsec:example_compositions_of_structure}

We give a number of example adjoint structures that are of
interest to the research community.

\textbf{Lambek Calculus with Exchange.}  The first is a model that
reveals how to combine the Lambek Calculus with the exchange comonad
and Girard's of-course comonad.  This model is of interest to the
linguistics community \cite{?}, because they often only want exchange
in very controlled instances.  Valeria de Paiva~\cite{?} was the first
to show that this is possible using Dialectica Categories and
Reedy's~\cite{?}  model.  However, she uses a comonad with the natural
transformation $\e{A,B} : A \seq eB \mto eB \seq A$, but we feel this
goes against the standard view of algebraic binary operations.  In
addition, while Dialectica categories are extremely useful, but rather
complex, we are interested in simpler models.  Thus, we prefer an
adjoint model with a comonad which has the natural transformation
$\e{A,B} : eA \seq eB \mto eB \seq eA$.

As we have said in the introduction there are many security
applications where one must have both a commutative and a
non-commutative tensor product within the same logic.  For example,
when reasoning about process trees in threat analysis.

\begin{definition}
  \label{def:LC-adjoint-structure}
  Suppose $\cat{L}_{ewc}$ is a Lambek category with exchange,
  weakening, and contraction, $\cat{L}_e$ is a Lambek category with
  exchange, and $\cat{L}$ is a Lambek category.  Then a \textbf{LC
    adjoint model} is the composition of structure
  $\cat{L}_{ewc} : H \dashv J : \cat{L}_e : F \dashv G : \cat{L}$.
\end{definition}

We must now show that $\cat{L}$ in a LC adjoint model has two comonads
$e : \cat{L} \mto \cat{L}$ adding exchange to $\cat{L}$, and $! :
\cat{L} \mto \cat{L}$ -- Girard's of-course modality -- adding
weakening, contraction, and exchange.  We first have the following
corollary to Lemma~\ref{lemma:comonads_in_a_composition_of_structure}.

\begin{corollary}
  \label{corollary:LC-comonads}
  Suppose $\cat{L}_{ewc} : H \dashv J : \cat{L}_e : F \dashv G : \cat{L}$ is a LC adjoint model.
  Then there are comonads 
  $(e : \cat{L} \mto \cat{L}, \varepsilon^e, \delta^e)$, $(! : \cat{L} \mto \cat{L}, \varepsilon^!, \delta^!)$,
  and $(!_e : \cat{L}_e \mto \cat{L}_e, \varepsilon^{!_e}, \delta^{!_e})$.
\end{corollary}
\begin{proof}
  We only show how each of the tuples are defined:
  \begin{itemize}
  \item $eA = FGA$, $\varepsilon^{e}_A : eA \mto A$ is the
    counit of the adjunction, and $\delta^{e}_A = F\eta^e_{GA} : eA
    \mto eeA$, where $\eta^e_A : A \mto GFA$ is the unit of the
    adjunction.
    
  \item $!_eA = HJA$, $\varepsilon^{!_e}_A : !_eA \mto A$ is the
    counit of the adjunction, and $\delta^{!_e}_A = H\eta^{!_e}_{JA} : !_e A
    \mto !_e!_eA$, where $\eta^{!_e}_A : A \mto JHA$ is the unit of the
    adjunction.  

  \item $!A = FHJGA$, $\varepsilon^{!}_A : !A \mto A$ is the
    counit of the adjunction, and $\delta^{!}_A = FH\eta^{!}_{JGA} : ! A
    \mto !!A$, where $\eta^!_A : A \mto JGFHA$ is the unit of the
    adjunction.
  \end{itemize}  
\end{proof}

\noindent
As a corollary to Lemma~\ref{lemma:kleisli_category_gen} we show that
the Eilenberg-Moore categories contain the required structure.
\begin{corollary}
  \label{corollary:EM-LC-adjoint-model}
  Suppose $\cat{L}_{ewc} : H \dashv J : \cat{L}_e : F \dashv G : \cat{L}$ is a LC adjoint model. Then
  the Eilenberg-Moore Categories associated with the comonads
  $(e : \cat{L} \mto \cat{L}, \varepsilon^e, \delta^e)$, $(! : \cat{L} \mto \cat{L}, \varepsilon^!, \delta^!)$,
  and $(!_e : \cat{L}_e \mto \cat{L}_e, \varepsilon^{!_e}, \delta^{!_e})$
  have the structure:
  $\cat{L}^E_e$ is symmetric monoidal, and
  $\cat{L}^E_!$ and $\cat{L}^E_{!_e}$ are cartesian closed.
\end{corollary}
\begin{proof}
  \begin{itemize}
    \item \textbf{$\cat{L}^E_{!_e}$ is cartesian closed}: The proof that $\cat{L}^E_{!_e}$
      is cartesian closed follows from Benton~\cite{Benton:1994}, because $\cat{L}_e$ is
      symmetric monoidal, and $\cat{L}_{ewc}$ is cartesian closed, and hence
      $\cat{L}_{ewc} : H \dashv J : \cat{L}_e$ is a LNL model.
    \item \textbf{$\cat{L}^E_e$ is symmetric monoidal}: The Eilenberg-Moore category,
      $\cat{L}^E_e$, of the comonad $(e,\epsilon^e,\delta^e)$ has as objects all the
      $e$-coalgebras $(A, h_A:A\mto eA)$, and as morphisms all the coalgebra morphisms
      such that the following (action) diagrams commute:
      \begin{mathpar}
        \bfig
        \square[A`eA`eA`e^2A;h_A`h_A`eh_A`\delta_A]    
        \efig
        \and
        \bfig
        \btriangle/->`=`->/[A`eA`A;h_A``\varepsilon_A]
        \efig
      \end{mathpar}
      We first prove $\cal{L}^E_e$ is monoidal. The tensor product of $(A, h_A)$ and
      $(B, h_B)$ is \\
      $(A\otimes B, \q{A,B}\circ(h_A\otimes h_B)): A\otimes B\mto e(A\otimes B)$ and the
      unit of the tensor product is $(I, \q{I}:I\mto eI)$. Obviously, they both satisfy the
      action diagrams above. The left unitor is $\hat\lambda:I\otimes A\mto A$, which is the
      morphism $I\otimes A\mto eA$ in $\cat{L}$ defined as
      $e\lambda_A\circ\q{I,A}\circ(\q{I}\otimes id_{eA})\circ(id_I\otimes h_A)$, where
      $\lambda$ is the left unitor in $\cat{L}$. The right unitor is
      $\hat\rho:A\otimes I\mto A$, which is the morphism $A\otimes I\mto A$ in $\cat{L}$
      defined as $e\rho_A\circ\q{A,I}\circ(id_{eA}\otimes\q{I})\circ(h_A\otimes id_I)$,
      where $\rho$ is the right unitor in $\cat{L}$. And the associator is
      $\hat\alpha:(A\otimes B)\otimes C\mto A\otimes (B\otimes C)$, which is the
      morphism $(A\otimes B)\otimes C\mto e(A\otimes(B\otimes C))$ in $\cat{L}$ defined
      as $e\e{A,B}\circ\q{A,B}\circ(id_{eA}\otimes h_B)\circ(h_A\otimes id_B)$, where
      $\alpha$ is the associator in $\cat{L}$.
      
      
      The left and
      right unitors are $\lambda:I\otimes A\mto A$ and $\rho:A\otimes I\mto A$, and the
      associator is $\alpha:(A\otimes B)\otimes C\mto A\otimes(B\otimes C)$. These left and
      right unitors and the associator are coalgebra morphisms because the following
      diagrams commute:
      \begin{mathpar}
        \bfig
        \morphism<450,0>[I\otimes A`I\otimes eA;id_I\otimes h_A]
        \morphism(450,0)<550,0>[I\otimes eA`eI\otimes eA;\q{I}\otimes id_{eA}]
        \morphism(1000,0)<450,0>[eI\otimes eA`e(I\otimes A);\q{I,A}]
        \morphism<0,-400>[I\otimes A`A;\lambda_A]
        \morphism(1450,0)<0,-400>[e(I\otimes A)`eA;e\lambda_A]
        \morphism(450,0)<1000,-400>[I\otimes eA`eA;\lambda_{eA}]
        \morphism(0,-400)<1450,0>[A`eA;h_A]
        \efig
        \and
        \bfig
        \morphism<450,0>[A\otimes I`eA\otimes I;h_A\otimes id_I]
        \morphism(450,0)<550,0>[eA\otimes I`eA\otimes eI;id_{eA}\otimes \q{I}]
        \morphism(1000,0)<450,0>[eA\otimes eI`e(A\otimes I);\q{A,I}]
        \morphism<0,-400>[A\otimes I`A;\rho_A]
        \morphism(1450,0)<0,-400>[e(A\otimes I)`eA;e\rho_A]
        \morphism(450,0)<1000,-400>[eA\otimes I`eA;\rho_{eA}]
        \morphism(0,-400)<1450,0>[A`eA;h_A]
        \efig
        \and
        \bfig
        \square<900,400>[
          (A\otimes B)\otimes C`
          (eA\otimes eB)\otimes eC`
          A\otimes(B\otimes C)`
          eA\otimes(eB\otimes eC);
          (h_A\otimes h_B)\otimes h_C`
          \alpha_{A,B,C}`
          \alpha_{eA,eB,eC}`
          h_A\otimes(h_B\otimes h_C)]
        \morphism(900,400)<900,0>[(eA\otimes eB)\otimes eC`e(A\otimes B)\otimes eC;\q{A,B}\otimes id_{eC}]
        \morphism(900,0)<900,0>[eA\otimes(eB\otimes eC)`eA\otimes e(B\otimes C);id_{eA}\otimes\q{B,C}]
        \square(1800,0)/->``->`->/<900,400>[
          e(A\otimes B)\otimes eC`
          e((A\otimes B)\otimes C)`
          eA\otimes e(B\otimes C)`
          e(A\otimes(B\otimes C));
          \q{A\otimes B,C}``
          e\alpha_{A,B,C}`
          \q{A,B\otimes C}]
        \efig
      \end{mathpar}
      Composition in $\cat{L}^E_e$ is the same as in $\cat{L}_e$, and thus, the monoidal
      coherence diagrams hold in $\cat{L}^E_e$ as well. Thus, $\cat{L}^E_e$ is monoidal. We
      now show that it is symmetric. \\
      The symmetry is $\e{A,B}:A\otimes B\mto B\otimes A$, which is a coalgebra morphism
      by the following commutative diagram:
      \begin{mathpar}
        \bfig
        \square|allb|<600,400>[
          A\otimes B`eA\otimes B`B\otimes A`eB\otimes A;
          h_A\otimes id_B`\e{A,B}`\e{eA,B}`h_B\otimes id_A]
        \morphism(600,400)<600,0>[eA\otimes B`eA\otimes eB;id_{eA}\otimes h_B]
        \morphism(600,0)|b|<600,0>[eB\otimes A`eB\otimes eA;id_{eB}\otimes h_A]
        \square(1200,0)<600,400>[
          eA\otimes eB`e(A\otimes B)`eB\otimes eA`e(B\otimes A);
          \q{A,B}`\e{eA,eB}`e\e{A,B}`\q{B,A}]
        \efig
      \end{mathpar}
      Similarly, the coherence diagrams for symmetry commute. This concludes the proof that
      $\cat{L}^E_e$ is symmetric monoidal.
    \item \textbf{$\cat{L}^E_!$ is cartesian closed}: Similar with the proof that
      $\cat{L}^E_e$ is symmetric monoidal, $\cat{L}^E_!$ is also symmetric monoidal, with the
      unit being $(I,\q{I}:I\mto!I)$, the right unitor being $\hat\lambda:I\otimes A\mto A$,
      the right unitor being $\hat\rho:A\otimes I\mto A$, and the associator being
      $\hat\alpha:(A\otimes B)\otimes C\mto A\otimes(B\otimes C)$. \\
      (NOT FINISHED)
  \end{itemize}
\end{proof}
\noindent
The previous result shows that $\cat{L}$ has the following structural morphisms:
\[
\begin{array}{lllllll}
  \e{A,B} : eA \seq eB \mto eB \seq eA & \quad & 
  \w{A} : !A \mto I & \quad &
  \c{A} : !A \seq !A \mto !A\\
\end{array}
\]
In addition, the category $\cat{L}_e$ has the following structural morphisms:
\[
\begin{array}{llllll}
  \e{A,B} : A \otimes B \mto B \otimes A & \quad &
  \w{A} : !_eA \mto I & \quad &
  \c{A} : !_eA \otimes !_eA \mto !_eA\\
\end{array}
\]
However, notice that exchange is a first class citizen, and hence, is not tracked by a comonad.

There are a number of additional models with similar features to the
LC adjoint model.  Suppose $r_1,r_2 \in \{w,c\}$.  Then the set of
additional models of the Lambek calculus with exchange and the
of-course modality can be modeled by the compositions of structure
defined as follows:
\[
\begin{array}{lll}
  \cat{L}_{r_2} : K \dashv M : \cat{L}_{r_1} : H \dashv J : \cat{L}_{e} : F \dashv G : \cat{L}\\
\end{array}
\]
Here we have decomposed $\cat{L}_{ewc}$ into an adjunction
$\cat{L}_{r_2} : K \dashv M : \cat{L}_{r_1}$ where the Eilenberg-Moore
category of the comonad induced by this adjunction is
$\cat{L}_{r_1r_2}$, and thus, when this adjunction is composed with
the adjunction $\cat{L}_{r_1} : H \dashv J : \cat{L}_{e} : F \dashv G
: \cat{L}$ it yields a comonad whose Eilenberg-Moore category
$\cat{L}_{er_1r_2}$ is isomorphic to $\cat{L}_{ewc}$.

However, each of these models are in fact different. They each track
different structural rules in different ways. The LC adjoint model
induces only two comonads, but the models just introduced induces six
different comonads.  Thus, depending what one wants to track and in
which category determines which model is best.


\textbf{Affine Logic.} Affine logic has applications in verification
of security protocols \cite{Bugliesi:2015:ART:2807424.2743018}.
Compositions of structure provide a means to combine intuitionistic
linear logic with affine logic.
\begin{definition}
  \label{def:LC-adjoint-structure}
  Suppose $\cat{L}_{ew}$ is a Lambek category with exchange and
  weakening, and  $\cat{L}_e$ is a Lambek category with
  exchange.  Then an \textbf{affine adjoint model} is the composition of structure
  $\cat{L}_{ew} : F \dashv G : \cat{L}_e$.
\end{definition}
Just as above this model comes with a comonad $(w : \cat{L}_e \mto
\cat{L}_e, \varepsilon^w, \delta^w)$ defined in the same way as $!_eA$
from above.  This then equips $\cat{L}_e$ with the natural
transformation $\w{A} : wA \mto I$.  Finally, the Eilenberg-Moore
category $\cat{L}^E_w$ models intuitionistic affine logic.

In this example we made both categories symmetric, but this is not
strictly necessary.  One could also model non-commutative affine logic
as well.  In fact, commutativity -- as well as all of the other
structural rules -- is optional in every logic we discuss in this
paper.

\textbf{Strict Logic.}  Similarly to affine logic we can model strict
logic -- sometimes referred to as contraction logic -- as well.
\begin{definition}
  \label{def:LC-adjoint-structure}
  Suppose $\cat{L}_{ec}$ is a Lambek category with exchange and
  contraction, and  $\cat{L}_e$ is a Lambek category with
  exchange.  Then a \textbf{strict adjoint model} is the composition of structure
  $\cat{L}_{ec} : F \dashv G : \cat{L}_e$.
\end{definition}
Just as above this model comes with a comonad $(c : \cat{L}_e \mto
\cat{L}_e, \varepsilon^c, \delta^c)$ defined in the same way as $!_eA$
and $wA$ from above.  This then equips $\cat{L}_e$ with the natural
transformation $\c{A} : cA \otimes cA \mto cA$.  Finally, the
Eilenberg-Moore category $\cat{L}^E_c$ models intuitionistic strict
logic.

% subsection example_compositions_of_structure (end)

\subsection{Resource Adjoint Models}
\label{subsec:resource_adjoint_models}

The main idea behind a composition of structure is that when we
chain multiple Lambek categories together we can think of it as
composing several comonads together resulting in a comonad which
embodies the structure of each category in the sequence of
adjunctions.  How do we add several different types of
comonads to a base category which do not arise as a composition of
adjunctions?

As an example, suppose $\cat{L}$ is a Lambek category, and we want to
track exchange, affine logic, and strict logic as different comonads
in $\cat{L}$.  Then a model of this logic would be the following set
of compositions of structure:
\[
\begin{array}{lll}
  \cat{L}_w : F_1 \dashv G_1 : \cat{L}_e : F_0 \dashv G_0 : \cat{L}\\
  \cat{L}_{ec} : F_2 \dashv G_2 : \cat{L}\\
\end{array}
\]
The first composition of structure endows $\cat{L}$ with two comonads
that track exchange and affine logic -- by composing the exchange
comonad with the weakening comonad -- respectively.  The second
adjunction adds in the comonad that tracks strict logic, because
$\cat{L}_{ec}$ contains both exchange and contraction.  Again, there
are a few other sets of compositions of structure that will also allow
one to model this situation, which is best depends on ones needs.  The
key point is that by combining multiple compositions of structure on
the same base we increase the diversity of the kinds of resources that
can be tracked.

We have now arrived at a new model of resource-conscious logics based
on compositions of structure.
\begin{definition}
  \label{def:resource-adjoint-model}
  Suppose $\cat{L}_0$ is a Lambek category called the base of the
  model.  Then a \textbf{resource adjoint model},
  $\{\overleftarrow{\cat{L}_n : F_{n-1} \dashv G_{n-1} : \cat{L}_{n-1}}\}$,
  is a set of compositions of structure whose bases are all $\cat{L}_0$.
\end{definition}


% subsection combining_models (end)

%%% Local Variables: 
%%% mode: latex
%%% TeX-master: main.tex
%%% End: 
