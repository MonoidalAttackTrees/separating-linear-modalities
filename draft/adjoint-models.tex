Now we turn to making our model precise.
\begin{definition}
  \label{def:model}  
  Suppose $\cat{L}_0, \ldots, \cat{L}_n$ is a family of Lambek
  categories with zero or more structural morphisms where $\cat{L}_i$
  is a full subcategory of $\cat{L}_{i+1}$.  Then a \textbf{structural
    adjoint model}, \\ $\overrightarrow{\cat{L}_n : F_n \dashv G_n : \cat{L}_{n-1}}$, is a composition of monoidal adjunctions:
  \[ \cat{L}_n : F_n \dashv G_n : \cat{L}_{n-1} : F_{n-1} \dashv G_{n-1} : \cat{L}_{n-2} : \cdots : \cat{L}_1 : F \dashv G : \cat{L}_0. \]
\end{definition}
This definition is an extension -- or perhaps a simplification due to
the isolation of exchange -- of the models discussed by
Melli\'es~\cite{MELLIES2004202}.

Our structural adjoint model subsumes Benton's~\cite{Benton:1994}
linear/non-linear model (LNL).  Simply take the sequence of Lambek
categories to be $\cat{L}_0$, a Lambek category with exchange, and
$\cat{L}_1$, a Lambek category with weakening, contraction, and
exchange, and thus, $\cat{L}_1$ is cartesian closed.  However, our
model is a lot more flexible and expressive.

\begin{lemma}[Comonads in a Structural Adjoint Model]
  \label{lemma:comonads_in_a_structural_adjoint_model}
  Suppose $\overrightarrow{\cat{L}_n : F_n \dashv G_n : \cat{L}_{n-1}}$ is a structural adjoint model.  Then
  there are the following comonads:  
  \begin{itemize}
  \item $(\cat{L}_0, F_0G_0, \varepsilon^0, \delta^0), \ldots, (\cat{L}_{n-1}, F_nG_n, \varepsilon^{n}, \delta^{n})$
  \item $(\cat{L}_0, F_0F_1G_1G_0, \varepsilon^0, \delta^0), \ldots, (\cat{L}_{n-1}, F_{n-1}F_nG_nG_{n-1}, \varepsilon^{n}, \delta^{n})$
  \item $(\cat{L}_0, F_0F_1F_2G_2G_1G_0, \varepsilon^0, \delta^0), \ldots, (\cat{L}_{n-1}, F_{n-2}F_{n-1}F_nG_nG_{n-1}G_{n-2}, \varepsilon^{n}, \delta^{n})$\\
    $\vdots$
  \item $(\cat{L}_0, F_0 \cdots F_nG_n \cdots G_0, \varepsilon^0, \delta^0)$
  \end{itemize} 
\end{lemma}
\begin{proof}
  This proof easily follows from the well-known fact that adjoints
  induce comonads -- as well as monads -- and composition of adjoints.
\end{proof}
The previous lemma shows that a Lambek category $\cat{L}_i$ in the
sequence is endowed with all of the structure found in each of the
categories above it, but this structure is explicitly tracked using
the various comonads.  That is, the Eilenberg-Moore category of each of the
comonads mentioned in the previous lemma has the corresponding
structural rule as morphisms.

\begin{lemma}
  \label{lemma:kleisli_category_gen}
  Suppose $\cat{L}_i : F_i \dashv G_i : \cat{L}_{i-1}$ is a monoidal
  adjunction in the structural adjoint model
  $\overrightarrow{\cat{L}_n : F_n \dashv G_n : \cat{L}_{n-1}}$.  Then
  the Eilenberg-Moore category, $\cat{L}^E_{i-1}$, contains all of the
  structural morphisms from both $\cat{L}_{i-1}$ and $\cat{L}_{i}$.
\end{lemma}


\subsection{Example Structural Adjoint Models}
\label{subsec:example_structural_adjoint_models}

\textbf{Exchange.}

Prove the Eilenberg-Moore category is symmetric monoidal.

\textbf{Contraction.}

Prove the Klesli category has contractions.

\textbf{Contraction and Exchange.}

Prove the Klesli category has weakening.

% subsection example_structural_adjoint_models (end)
