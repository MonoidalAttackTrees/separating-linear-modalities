The logic we present here will be the basis of all of the logics
discussed in this section, and can also be seen as detailing the basic
setup of an adjoint logic.  This logic will consist of a single
syntactic adjunction where both sides of the adjunction are identical.
Thus, the induced comonad by the adjunction will not add anything new.

The syntax is defined in Figure~\ref{fig:LC-with-LC:syntax}.
\begin{figure}
  \begin{mdframed}
    \[
    \begin{array}{lll}
      \text{(Source Formulas)} & [[A]],[[B]],[[C]] ::= [[Unit]] \mid [[A (x) B]] \mid [[A -> B]] \mid [[B <- A]]\\
      \text{(Target Formulas)} & [[X]],[[Y]],[[Z]] ::= [[Unit]] \mid [[X (x) Y]] \mid [[X -> Y]] \mid [[Y <- X]]\\
      \text{(Source and Target)} & [[T]] := [[A]] \mid [[X]]\\
      \text{(Source Contexts)} & [[G]],[[D]] ::= [[.]] \mid [[x : A]] \mid [[G,D]]\\
      \text{(Target Contexts)} & [[I]],[[P]] ::= [[.]] \mid [[x : X]] \mid [[I,P]]\\      
      \text{(Source Terms)}    & [[s]] ::= \\
      \text{(Target Terms)}    & [[t]] ::= \\
      \text{(Patterns)}        & [[p]] ::= [[-]] \mid [[x]] \mid [[triv]] \mid [[p (x) p']] \mid [[F p]] \mid [[Gf p]]\\
    \end{array}
    \]
  \end{mdframed}
  \caption{Syntax for the Lambek Calculus with Itself}
  \label{fig:LC-with-LC:syntax}
\end{figure}
