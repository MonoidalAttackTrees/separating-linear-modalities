The bases of all of our models will be what we call Lambek categories.
These are named after Joachim Lambek to pay homage to his work on the
Lambek calculus which can be seen as non-commutative intuitionistic
linear logic \cite{Lambek1968-LAMTMO-5}. Thus, each of our models have
a very basic foundation.

Lambek categories are based on (non-symmetric) monoidal categories.
\begin{definition}
  \label{def:Lambek-category}
  A \textbf{monoidal category}, $(\cat{L}, \seq, I, \lambda,
  \rho)$, is a category, $\cat{L}$, equipped with a bifunctor,
  $\seq : \cat{L} \times \cat{L} \mto \cat{L}$, called the tensor
  product, a distinguished object $I$ of $\cat{L}$ called the unit,
  and three natural isomorphisms $\lambda_A : I \seq A \mto A$,
  $\rho_A : A \seq I \mto A$, and $\alpha_{A,B,C} : (A \seq B)
  \seq C \mto A \seq (B \seq C)$ called the left and right
  unitors and the associator respectively.  Finally, these are subject
  to the following coherence diagrams:
  \begin{mathpar}
    \bfig
    \hSquares|aammmma|/->`->`->``->``/<400>[
      ((A \seq B) \seq C) \seq D`
      (A \seq (B \seq C)) \seq D`
      A \seq ((B \seq C) \seq D)`
      (A \seq B) \seq (C \seq D)``
      A \seq (B \seq (C \seq D));
      \alpha_{A,B,C} \seq \id_D`
      \alpha_{A,B \seq C,D}`
      \alpha_{A \seq B,C,D}``
      \id_A \seq \alpha_{B,C,D}``]

    \morphism(-200,0)<2700,0>[
      (A \seq B) \seq (C \seq D)`
      A \seq (B \seq (C \seq D));
      \alpha_{A,B,C \seq D}]
    \efig
    \and
    \bfig
    \Vtriangle[
      (A \seq I) \seq B`
      A \seq (I \seq B)`
      A \seq B;
      \alpha_{A,I,B}`
      \rho_{A}\seq id_B`
      id_A\seq\lambda_{B}]
    \efig
  \end{mathpar}
\end{definition}
\noindent
A Lambek category adds closure to monoidal categories.
\begin{definition}
  \label{def:Lambek-category}
  A \textbf{Lambek category} is a monoidal category $(\cat{L},
  \seq, I, \lambda, \rho,\alpha)$ equipped with two bifunctors
  $\lto : \catop{L} \times \cat{L} \mto \cat{L}$ and $\rto : \cat{L}
  \times \catop{L} \mto \cat{L}$ that are both right adjoint to the
  tensor product.  That is, the following natural bijections hold:
  \begin{center}
    \begin{math}
      \begin{array}{lll}
        \Hom{L}{X \seq A}{B} \cong \Hom{L}{X}{A \lto B} & \quad\quad\quad\quad & 
        \Hom{L}{A \seq X}{B} \cong \Hom{L}{X}{B \rto A}\\
      \end{array}
    \end{math}
  \end{center}
\end{definition}
An alternative name for Lambek categories is biclosed monoidal
categories.

If we add a symmetry to a Lambek category then we will obtain a
symmetric monoidal closed category.  The following two definitions and
lemma capture this result.
\begin{definition}
  \label{def:sym-monoidal-category}
  A monoidal category $(\cat{L},\seq,I,\lambda,\rho,\alpha)$ is
  \textbf{symmetric} if there is a natural transformation $\e{A,B}
  : A \seq B \mto B \seq A$ such that $\e{B,A} \circ
  \e{A,B} = \id_{A \seq B}$ and the following commute:
  \begin{center}
    \begin{math}
      \small
      \begin{array}{lll}
        \bfig
        \hSquares|aammmaa|/->`->`->``->`->`->/[
        (A \seq B) \seq C`
        A \seq (B \seq C)`
        (B \seq C) \seq A`
        (B \seq A) \seq C`
        B \seq (A \seq C)`
        B \seq (C \seq A);
        \alpha_{A,B,C}`
        \e{A,B \seq C}`
        \e{A,B} \seq \id_C``
        \alpha_{B,A,C}`
        \alpha_{B,A,C}`
        \id_B \seq \e{A,C}]
        \efig
        & \quad &
        \bfig
          \Vtriangle[
            I \seq A`
            A \seq I`
            A;
            \e{I,A}`
            \lambda_A`
            \rho_A]
          \efig
      \end{array}
    \end{math}
  \end{center}
  Throughout this paper when $- \seq -$ is symmetric we denote it by
  $- \otimes -$.
\end{definition}
We call a symmetric Lambek category a Lambek category with exchange,
because the symmetry models the exchange rule.
\begin{definition}
  \label{def:sym-monoidal-closed}
  A symmetric monoidal category $(\cat{L}, \otimes, I, \lambda, \rho,
  \alpha, \beta)$ is \textbf{closed} if it comes equipped with a
  bifunctor $\limp : \catop{L} \times \cat{L} \mto \cat{L}$ that is
  right adjoint to the tensor product.  That is, the following natural
  bijection $\Hom{L}{X \otimes A}{B} \cong \Hom{L}{X}{A \limp B}$ holds.
\end{definition}
\begin{lemma}
  \label{lemma:internal-homs-collapse}
  Let $A$ and $B$ be two objects in a Lambek category with exchange. Then
  $(A \lto B) \cong (B \rto A)$.
\end{lemma}
\begin{proof}
  First, notice that for any object $C$ we have
  \begin{center}
  \begin{math}
  \small
  \begin{array}{lllll}
    Hom[C,A\lto B]
    & \cong & Hom[C\otimes A,B] & \cat{L}\text{ is a Lambek category}\\
    & \cong & Hom[A\otimes C,B] & \text{By the symmetry }\e{C,A}\\
    & \cong & Hom[C,B\rto A]    & \cat{L}\text{ is a Lambek category}
  \end{array}
  \end{math}
  \end{center}  
  Thus, $A\lto B\cong B\rto A$ by the Yoneda lemma.
\end{proof}
\begin{corollary}
  \label{corollary:LC-with-ex-mc}
  A Lambek category with exchange is symmetric monoidal closed.
\end{corollary}

We will also be discussing two other structural rules: weakening and
contraction.  These are defined as follows.
\begin{definition}
  \label{def:weakening}
  A \textbf{Lambek category with weakening}, $(\cat{L}, \otimes, I,
  \lambda, \rho, \alpha, \w{})$, is a Lambek category $(\cat{L}, \otimes, I,
  \lambda, \rho,\alpha)$ equipped with a natural transformation
  $\w{A} : A \mto I$.
\end{definition}
\begin{definition}
  \label{def:contraction}
  A \textbf{Lambek category with contraction}, $(\cat{L}, \otimes, I,
  \lambda, \rho, \alpha, \cL{},\cR{})$, is a Lambek category $(\cat{L}, \otimes, I,
  \lambda, \rho,\alpha)$ equipped with natural transformations:
  \[
  \begin{array}{lll}
    \cL{A,B} : (A \otimes B) \mto (A \otimes B) \otimes A & \quad &
    \cR{A,B} : (B \otimes A) \mto A \otimes (B \otimes A)\\
  \end{array}
  \]
  Furthermore, the following diagrams must commute:
    \begin{mathpar}
      \bfig
      \square/<-`->``/<1050,400>[
	A\otimes I`
        A`
        (A\otimes I)\otimes A`;
	\rho_{A}^{-1}`
	\cL{A,I}``]
      \square(1050,0)/->``->`/<1050,400>[
        A`
        I\otimes A``
        A\otimes(I\otimes A);
        \lambda_{A}^{-1}``
	\cR{A,I}`]
        \morphism(0,0)|b|<2100,0>[(A\otimes I)\otimes A`A\otimes(I\otimes A);\alpha_{A,I,A}]
      \efig
    \end{mathpar}
    \begin{mathpar}
    \bfig
      \square/->`->``->/<1300,800>[
        A\otimes A`
        A\otimes(A\otimes I)`
        (I\otimes A)\otimes A`
        (A\otimes(I\otimes A))\otimes A;
        id_{A}\otimes\rho_{A}^{-1}`
        \lambda_{A}^{-1}\otimes id_{A}``
        \cR{A,I}\otimes id_{A}]
      \qtriangle(1300,400)/->``->/<1300,400>[
        A\otimes(A\otimes I)`
        A\otimes((A\otimes I)\otimes A)`
        A\otimes(A\otimes A);
        id_{A}\otimes\cL{A,I}``
        id_{A}\otimes(\rho_{A}\otimes id_{A})]
      \dtriangle(1300,0)/`<-`->/<1300,400>[
        A\otimes(A\otimes A)`
        (A\otimes(I\otimes A))\otimes A`
        (A\otimes A)\otimes A;
        `
        \alpha_{A,A,A}`
        (id_{A}\otimes\lambda_{A})\otimes id_{A}]
    \efig
    \end{mathpar}
\end{definition}
We call the morphisms:
\[
\begin{array}{lll}
  \e{A,B} : A \otimes B \mto B \otimes A\\
  \w{A} : A \mto I\\
  \cL{A,B} : (A \otimes B) \mto (A \otimes B) \otimes A\\
  \cR{A,B} : (B \otimes A) \mto A \otimes (B \otimes A)\\
\end{array}
\]
structural morphisms, because they all model the various structural
rules in intuitionistic logic.




















