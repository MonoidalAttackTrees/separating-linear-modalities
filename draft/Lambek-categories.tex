\begin{definition}
  \label{def:Lambek-category}
  A \textbf{monoidal category}, $(\cat{L}, \otimes, I, \lambda,
  \rho)$, is a category, $\cat{L}$, equipped with a bifunctor,
  $\otimes : \cat{L} \times \cat{L} \mto \cat{L}$, called the tensor
  product, a distinguished object $I$ of $\cat{L}$ called the unit,
  and three natural isomorphisms $\lambda_A : I \otimes A \mto A$,
  $\rho_A : A \otimes I \mto A$, and $\alpha_{A,B,C} : A \otimes (B
  \otimes C) \mto (A \otimes B) \otimes C$ called the left and right
  unitors and the associator respectively.  Finally, these are subject
  to the following coherence diagrams:
  \begin{mathpar}
    \bfig
    \hSquares|aammmma|/->`->`->``->``/<400>[
      ((A \otimes B) \otimes C) \otimes D`
      (A \otimes (B \otimes C)) \otimes D`
      A \otimes ((B \otimes C) \otimes D)`
      (A \otimes B) \otimes (C \otimes D)``
      A \otimes (B \otimes (C \otimes D));
      \alpha_{A,B,C} \otimes \id_D`
      \alpha_{A,B \otimes C,D}`
      \alpha_{A \otimes B,C,D}``
      \id_A \otimes \alpha_{B,C,D}``]

    \morphism(-200,0)<2700,0>[
      (A \otimes B) \otimes (C \otimes D)`
      A \otimes (B \otimes (C \otimes D));
      \alpha_{A,B,C \otimes D}]
    \efig
    \and
    \bfig
    \Vtriangle[
      (A \otimes I) \otimes B`
      A \otimes (I \otimes B)`
      A \otimes B;
      \alpha_{A,I,B}`
      \rho_{A}\otimes id_B`
      id_A\otimes\lambda_{B}]
    \efig
  \end{mathpar}
\end{definition}

\begin{definition}
  \label{def:Lambek-category}
  A \textbf{Lambek category} is a monoidal category $(\cat{L},
  \otimes, I, \lambda, \rho,\alpha)$ equipped with two bifunctors
  $\lto : \catop{L} \times \cat{L} \mto \cat{L}$ and $\rto : \cat{L}
  \times \catop{L} \mto \cat{L}$ that are both right adjoint to the
  tensor product.  That is, the following natural bijections hold:
  \begin{center}
    \begin{math}
      \begin{array}{lll}
        \Hom{L}{X \otimes A}{B} \cong \Hom{L}{X}{A \lto B} & \quad\quad\quad\quad & 
        \Hom{L}{A \otimes X}{B} \cong \Hom{L}{X}{B \rto A}\\
      \end{array}
    \end{math}
  \end{center}
\end{definition}
One might call Lambek categories biclosed monoidal categories, but we
name them in homage to Lambek for his contributions to non-commutative
linear logic.

\begin{definition}
  \label{def:sym-monoidal-category}
  A monoidal category $(\cat{L},\otimes,I,\lambda,\rho,\alpha)$ is
  \textbf{symmetric} if there is a natural transformation $\beta_{A,B}
  : A \otimes B \mto B \otimes A$ such that $\beta_{B,A} \circ
  \beta_{A,B} = \id_{A \otimes B}$ and the following commute:
  \begin{center}
    \begin{math}
      \small
      \begin{array}{lll}
        \bfig
        \hSquares|aammmaa|/->`->`->``->`->`->/[
        (A \otimes B) \otimes C`
        A \otimes (B \otimes C)`
        (B \otimes C) \otimes A`
        (B \otimes A) \otimes C`
        B \otimes (A \otimes C)`
        B \otimes (C \otimes A);
        \alpha_{A,B,C}`
        \beta_{A,B \otimes C}`
        \beta_{A,B} \otimes \id_C``
        \alpha_{B,C,A}`
        \alpha_{B,A,C}`
        \id_B \otimes \beta_{A,C}]
        \efig
        & \quad &
        \bfig
          \Vtriangle[
            I \otimes A`
            A \otimes I`
            A;
            \beta_{I,A}`
            \lambda_A`
            \rho_A]
          \efig
      \end{array}
    \end{math}
  \end{center}
\end{definition}

\begin{definition}
  \label{def:sym-monoidal-closed}
  A symmetric monoidal category $(\cat{L}, \otimes, I, \lambda, \rho,
  \alpha, \beta)$ is \textbf{closed} if it comes equipped with a
  bifunctor $\limp : \catop{L} \times \cat{L} \mto \cat{L}$ that is
  right adjoint to the tensor product.  That is, the following natural
  bijection $\Hom{L}{X \otimes A}{B} \cong \Hom{L}{X}{A \limp B}$ holds.
\end{definition}
