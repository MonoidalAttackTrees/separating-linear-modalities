\documentclass{article}
\usepackage{amssymb,amsmath}
\usepackage{amsthm}
\usepackage{cmll}
\usepackage{txfonts}
\usepackage{graphicx}
\usepackage{stmaryrd}
\usepackage{todonotes}
\usepackage{mathpartir}
\usepackage{hyperref}
\usepackage{mdframed}
\usepackage[barr]{xy}
\usepackage{comment}
\usepackage{graphicx}
\usepackage[inline]{enumitem}

\usepackage{caption}
\captionsetup[figure]{name=Diagram}

\newtheorem{theorem}{Theorem}
\newtheorem{lemma}[theorem]{Lemma}
\newtheorem{corollary}[theorem]{Corollary}
\newtheorem{definition}[theorem]{Definition}
\newtheorem{proposition}[theorem]{Proposition}
\newtheorem{example}[theorem]{Example}

%% This renames Barr's \to to \mto.  This allows us to use \to for imp
%% and \mto for a inline morphism.
\let\mto\to
\let\to\relax
\newcommand{\to}{\rightarrow}
\newcommand{\ndto}[1]{\to_{#1}}
\newcommand{\ndwedge}[1]{\wedge_{#1}}

% Commands that are useful for writing about type theory and programming language design.
%% \newcommand{\case}[4]{\text{case}\ #1\ \text{of}\ #2\text{.}#3\text{,}#2\text{.}#4}
\newcommand{\interp}[1]{\llbracket #1 \rrbracket}
\newcommand{\normto}[0]{\rightsquigarrow^{!}}
\newcommand{\join}[0]{\downarrow}
\newcommand{\redto}[0]{\rightsquigarrow}
\newcommand{\nat}[0]{\mathbb{N}}
\newcommand{\fun}[2]{\lambda #1.#2}
\newcommand{\CRI}[0]{\text{CR-Norm}}
\newcommand{\CRII}[0]{\text{CR-Pres}}
\newcommand{\CRIII}[0]{\text{CR-Prog}}
\newcommand{\subexp}[0]{\sqsubseteq}
%% Must include \usepackage{mathrsfs} for this to work.

\date{}

\let\b\relax
\let\d\relax
\let\t\relax
\let\r\relax
\let\c\relax
\let\j\relax
\let\wn\relax
\let\H\relax

% Cat commands.
\newcommand{\powerset}[1]{\mathcal{P}(#1)}
\newcommand{\cat}[1]{\mathcal{#1}}
\newcommand{\func}[1]{\mathsf{#1}}
\newcommand{\iso}[0]{\mathsf{iso}}
\newcommand{\H}[0]{\func{H}}
\newcommand{\J}[0]{\func{J}}
\newcommand{\catop}[1]{\cat{#1}^{\mathsf{op}}}
\newcommand{\Hom}[3]{\mathsf{Hom}_{\cat{#1}}(#2,#3)}
\newcommand{\limp}[0]{\multimap}
\newcommand{\colimp}[0]{\multimapdotinv}
\newcommand{\dial}[1]{\mathsf{Dial_{#1}}(\mathsf{Sets^{op}})}
\newcommand{\dialSets}[1]{\mathsf{Dial_{#1}}(\mathsf{Sets})}
\newcommand{\dcSets}[1]{\mathsf{DC_{#1}}(\mathsf{Sets})}
\newcommand{\sets}[0]{\mathsf{Sets}}
\newcommand{\obj}[1]{\mathsf{Obj}(#1)}
\newcommand{\mor}[1]{\mathsf{Mor(#1)}}
\newcommand{\id}[0]{\mathsf{id}}
\newcommand{\lett}[0]{\mathsf{let}\,}
\newcommand{\inn}[0]{\,\mathsf{in}\,}
\newcommand{\cur}[1]{\mathsf{cur}(#1)}
\newcommand{\curi}[1]{\mathsf{cur}^{-1}(#1)}

\newcommand{\w}[1]{\mathsf{weak}_{#1}}
\newcommand{\c}[1]{\mathsf{contra}_{#1}}
\newcommand{\cL}[1]{\mathsf{contraL}_{#1}}
\newcommand{\cR}[1]{\mathsf{contraR}_{#1}}
\newcommand{\e}[1]{\mathsf{ex}_{#1}}

\newcommand{\m}[1]{\mathsf{m}_{#1}}
\newcommand{\n}[1]{\mathsf{n}_{#1}}
\newcommand{\b}[1]{\mathsf{b}_{#1}}
\newcommand{\d}[1]{\mathsf{d}_{#1}}
\newcommand{\h}[1]{\mathsf{h}_{#1}}
\newcommand{\p}[1]{\mathsf{p}_{#1}}
\newcommand{\q}[1]{\mathsf{q}_{#1}}
\newcommand{\t}[0]{\mathsf{t}}
\newcommand{\r}[1]{\mathsf{r}_{#1}}
\newcommand{\s}[1]{\mathsf{s}_{#1}}
\newcommand{\j}[1]{\mathsf{j}_{#1}}
\newcommand{\jinv}[1]{\mathsf{j}^{-1}_{#1}}
\newcommand{\wn}[0]{\mathop{?}}
\newcommand{\codiag}[1]{\bigtriangledown_{#1}}

\newenvironment{changemargin}[2]{%
  \begin{list}{}{%
    \setlength{\topsep}{0pt}%
    \setlength{\leftmargin}{#1}%
    \setlength{\rightmargin}{#2}%
    \setlength{\listparindent}{\parindent}%
    \setlength{\itemindent}{\parindent}%
    \setlength{\parsep}{\parskip}%
  }%
  \item[]}{\end{list}}

\newenvironment{diagram}{
  \begin{center}
    \begin{math}
      \bfig
}{
      \efig
    \end{math}
  \end{center}
}

%% %% Ott
%% \input{BiLNL-inc}

\urldef{\mailsa}\path|{heades}@augusta.edu|

\begin{document}

\title{\vspace{-45px}On Linear Modalities for Exchange, Weakening, and Contraction}
\author{Harley Eades III and Jiaming Jiang}
\date{Computer Science, Augusta University\\Computer Science, North Carolina State University}

\maketitle 

%% - Start with idea: supporting a wide range of substructural logics
%%     - Lambek calculus
%%     - Linear logic
%%     - Affine Logic
\vspace{-15px}
\textbf{Introduction.} In 1987 Girard introduced a new logic
\cite{Girard:1987} that is based on the property that every hypothesis
must be used exactly once.  He called this logic, linear logic.  This
means that the structural rules weakening and contraction are not
admissible.  However, to recover the ability to allow weakening and
contraction without sacrificing linearity completely Girard introduced
the of-course modality, denoted by $!A$, which is defined by the
following rules:
\vspace{-7px}
\[
\small
\begin{array}{cccccccccccccc}
  $$\mprset{flushleft}
  \inferrule* [right=] {
    \Gamma \vdash B
  }{\Gamma,!A \vdash B}
  &
  $$\mprset{flushleft}
  \inferrule* [right=] {
    \Gamma_1,!A,!A,\Gamma_2 \vdash B
  }{\Gamma_1,!A,\Gamma_2 \vdash B}
  &
  $$\mprset{flushleft}
  \inferrule* [right=] {
    !\Gamma \vdash B
  }{!\Gamma \vdash !B}
  &
  $$\mprset{flushleft}
  \inferrule* [right=] {
    \Gamma,A \vdash B
  }{\Gamma,!A \vdash B}
\end{array}
\vspace{-7px}
\]
The first two rules allow formulas under the of-course modality to be
weakened into the context and the second rule allows repeated
hypotheses to be contracted.  Now the last two rules imply that the
of-course modality is a comonad.

Linear logic is also commutative, that is, it contains the exchange
structural rule:
\vspace{-7px}
\[
\small
\inferrule* [right=] {
  \Gamma_1,A,B,\Gamma_2 \vdash C
}{\Gamma_1,B,A,\Gamma_2 \vdash C}
\vspace{-7px}
\]
Non-commutative formulations of linear logic go back to Lambek
\cite{Lambek1958} which happened to be before Girard.  Lambek used a
non-commutative tensor product to model sentence structure in
linguistics.  Today non-commutative linear logic has seen many
applications in computer science and linguistics.  However, some have
posed the question of whether an exchange modality can be added to
non-commutative linear logic in the same way that Girard added
weakening and contraction.

We propose a new logic that has the ability to model several
substructural logics.  The base of this logic is in non-commutative
linear logic.  Then we split the of-course modality into two separate
modalities: $w A$ and $c A$.  The former models weakening and the
latter contraction.  Then using a distributive law we show how to
compose these two modalities in such a way that the logic has the
ability to model logics with both weakening and contraction.  In
addition to these two modalities we add a third, $e A$, which adds
exchange.  In fact, under $e A$ we recover full intuitionistic linear
logic.  Finally, we show how to compose all three of these modalities
to obtain the ability to recover the of-course modality.

\textbf{Categorical model.} Following Bierman \cite{Bierman:1994} we
first show how to construct a categorical model in terms of monoidal
categories.  Suppose $(M, I, \otimes)$ is a monoidal category, where
$\otimes : M \times M \mto M$ is the tensor product with $I$ its unit.
Then equip $(M, I, \otimes)$ with two comonads
$(w,\varepsilon_w,\delta_w)$ and $(c,\varepsilon_c,\delta_c)$ where $w
: M \mto M$ and $c : M \mto M$ are endofunctors, and $\varepsilon_w :
wA \mto A$, $\varepsilon_c : cA \mto A$, $\delta_w : w^2A \mto wA$ and
$\delta_c : c^2A \mto cA$ are all natural transformations.  Then we
add the natural transformations $\w{A} : wA \mto I$, $\cL{A,B} : cA
\otimes B \mto (cA \otimes B) \otimes cA$ and $\cR{A,B} : B \otimes cA
\mto cA \otimes (B \otimes cA)$ where the first models weakening and
the last two model contraction.  All of these are subject to several
coherence diagrams; omitted for brevity.  By adding a distributive law
(natural transformation), $\mathsf{dist}_A : cwA \mto wcA$, we can
construct another comonad $(wc,\varepsilon_{wc},\delta_{wc})$ which
models both weakening and contraction.  Keep in mind that the
Eilenberg-Moore category for $wc : M \mto M$ is not cartesian or even
symmetric monoidal, because the exchange rule defined in terms of
weakening and contraction does not have the necessary properties.  We
add a third comonad $(e,\varepsilon_e,\delta_e)$ with the natural
isomorphism $\e{A,B} : eA \otimes eB \mto eB \otimes eA$; subject
to several coherence diagrams.  The Eilenberg-Moore category for this
comonad is indeed symmetric monoidal.  Thus, by adding another
distributive law (natural isomorphism) $\mathsf{dist}_A : ecwA \mto
cweA$ we can construct the comonad
$(cwe,\varepsilon_{cwe},\delta_{cwe})$ whose Eilenberg-Moore category
is cartesian.  In this setting the two contractions blend together to
form the usual $\c{A} : A \mto A \otimes A$.

\textbf{Logic and type system.} By exploiting the beautiful
Curry-Howard-Lambek correspondence we can form a logic and type system
using the previous model.  The base of the logic is the Lambek
Calculus (non-commutative linear logic) with the various comonads
defined in the same way the of-course modality is defined.  For
example, the rules for the exchange modality are defined as follows:
\vspace{-7px}
\[
\small
\begin{array}{cccccccccccccc}
  $$\mprset{flushleft}
  \inferrule* [right=] {
    \Gamma_1,eA,eB,\Gamma_2 \vdash C
  }{\Gamma_1,eB,eA,\Gamma_2 \vdash C}
  &
  $$\mprset{flushleft}
  \inferrule* [right=] {
    e\Gamma \vdash B
  }{e\Gamma \vdash eB}
  &
  $$\mprset{flushleft}
  \inferrule* [right=] {
    \Gamma,A \vdash B
  }{\Gamma,eA \vdash B}
\end{array}
\vspace{-7px}
\]
At this point we are forced to ask a question, which of the
distributive laws do we choose to add?  There are many.  We choose the
necessary laws to be able to model several different substructural
logics: $\mathsf{dist}_1 : cwA \mto wcA$, $\mathsf{dist}_2 : weA \mto
ewA$, $\mathsf{dist}_3 : ceA \mto ecA$, and $\mathsf{dist}_4 : ecwA
\mto cweA$.  These distributive laws allow for the encoding of the
Lambek calculus with weakening, contraction, and exchange, affine
logic, strict logic, and full intuitionistic linear logic.  The type
system is left for future work.

\textbf{Conclusion.}  The main application we have in mind for the
logic we describe above is to use it has a logical foundation of the
graphical model of threat analysis known as attack trees
\cite{Eades:2016a}.  These types of models require several different
commutative and non-commutative tensor products within the same logic.
We also believe that this type of system will be of interest to the
linear logic community, because it allows for the encoding of
several different forms of commutative and non-commutative
substructural logics.

\vspace{-15px}
\bibliographystyle{plainurl} \bibliography{ref}

\end{document}

%%% Local Variables: 
%%% mode: latex
%%% TeX-master: t
%%% End: 
